\section{Behavioral Refinement and Equivalence} \label{sec:refinement}

We define the QAM equivalence relation, we utilize the trace refinement relation.
The essential idea is that two transition systems are equivalent if the same sequences of actions can be performed
from their respective initial states.
Based on the QAM language defined in \Cref{sec:qam}, we define a labeled transition system as the following.

\begin{definition}\label{def:labeledsystem}\rm[QAM Labeled Transition System]
A \emph{QAM labeled transition system (LTS)} is a triple $\Sigma=(\Cs,A,S)$ where
\begin{itemize}
\item $\Cs$ is a set of configuration defined in \Cref{fig:q-pi-syntax}.
\item $A$ is a set of labels, whose elements have the form $p.c.m$ defined in \Cref{fig:q-pi-syntax}.
\item $S$ is a set of transition rules reflecting the QAM semantics in \Cref{sec:qamsemantics}.
\end{itemize}
\end{definition}

We can also define an initial configuration $C$ for a LTS $\Sigma=(\Cs,A,S)$ to restrict $\Sigma$ to begin at $C$, written as $(\Sigma,C)$.
In a LTS $(\Sigma,C)$, a trace is defined as a sequence of labels by evaluating an initial configuration $C$ by only collecting non-empty labels in sequence defined in \Cref{sec:qamsemantics}. The definition of a trace refinement relation is typically defined on a set of traces, referring to all possible traces by evaluating $C$ as the initial state, with the definition as follows.

\begin{definition}\label{def:traces}\rm[QAM Traces]
Given a LTS $(\Sigma,C)$ with $\Sigma=(\Cs,A,S)$, the set of \emph{traces} for a configuration $C\in \Cs$, written as $traces(\Sigma,C)$,
is the minimal set satisfying:

\begin{itemize}
\item $\epsilon\in traces(\Sigma,C)$, i.e., the empty trace is a member of $traces(\Sigma,C)$,
\item if there is a configuration $C'$, such that $C\longrightarrow C'$, $traces(\Sigma,C)=traces(\Sigma,C')$,
\item if there is a configuration $C'$, such that $C\xrightarrow{p.c.m}C'$ and $\sigma\in traces(\Sigma,C')$,
then the concatenation of $(p.c.m)$ and $\sigma$ is in $traces(\Sigma,C)$, written as $(p.c.m)\sigma \in traces(\Sigma,C)$.
\end{itemize}
\end{definition}

In defining a trace refinement relation, we are comparing two labeled transition systems $(\Sigma_1,C_1)$ and $(\Sigma_2,C_2)$ with the action set to be the same in the two LTS by the definition below.

\begin{definition}\label{def:traceeq}\rm[QAM Trace Refinement]
Given two labeled transition systems $(\Sigma_1,C_1)$ and $(\Sigma_2,C_2)$, we say that $(\Sigma_1,C_1)$ trace refines $(\Sigma_2,C_2)$, written as $(\Sigma_1,C_1) \sqsubseteq (\Sigma_2,C_2)$, iff $traces(\Sigma_1,C_1)\subseteq traces(\Sigma_2,C_2)$

\end{definition}

It is tricky to utilize the above definition to prove inductive properties for quantum abstract machines. To do so, we need to incorporate
two labeled systems with the traces definition in \Cref{def:traces}.
Mainly, we equate a configuration $C_a$ in $(\Sigma_1,C_1)$ with a configuration $C_b$ in $(\Sigma_2,C_2)$ by a configuration equivalence relation $=_C$, written as $C_a =_C C_b$. The inductive version of the trace refinement definition is given as follows.

\begin{definition}\label{def:traceeq1}\rm[QAM Trace Refinement]
Given two labeled transition systems $(\Sigma_1,C_1)$ and $(\Sigma_2,C_2)$, and $C_1 =_C C_2$, we say that $(\Sigma_1,C_1)$ trace refines $(\Sigma_2,C_2)$, written as $(\Sigma_1,C_1) \sqsubseteq (\Sigma_2,C_2)$, iff

\begin{itemize}
\item $(\Sigma_1,C_1)$ and $(\Sigma_2,C_2)$ cannot make a move,
\item there is a configuration $C'_1$, such that $C_1\longrightarrow C'_1$, and $(\Sigma_1,C'_1) \sqsubseteq (\Sigma_2,C_2)$,
\item there is a configuration $C'_2$, such that $C_2\longrightarrow C'_2$, then $(\Sigma_1,C_1) \sqsubseteq (\Sigma_2,C'_2)$,
\item there are configurations $C'_1$ and $C'_2$, such that $C_1\xrightarrow{p.c.m}C'_1$ and $C_2\xrightarrow{p.c.m}C'_2$ and $C_1' =_C C'_2$ and $(\Sigma_1,C'_1) \sqsubseteq (\Sigma_2,C'_2)$.
\end{itemize}
\end{definition}

We utilize \Cref{def:traceeq1} to show that the QPass protocol is a special instance of QCast, by showing that if we treat the statically generated connectivity graph in QCast, then every QPass trace is an instance of a QCast trace. The detail descriptions of viewing QPass and QCast as two labeled transition systems as well as the theorem are given in \Cref{sec:case-study}.



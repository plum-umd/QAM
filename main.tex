%%%%%%%%%%%%%%%%%%%%%%%%%%%%%%%%%%%%%%%%%%%%%%%%%%%%%%%%%%%%%%%
%
% Welcome to Overleaf --- just edit your LaTeX on the left,
% and we'll compile it for you on the right. If you open the
% 'Share' menu, you can invite other users to edit at the same
% time. See www.overleaf.com/learn for more info. Enjoy!
%
%%%%%%%%%%%%%%%%%%%%%%%%%%%%%%%%%%%%%%%%%%%%%%%%%%%%%%%%%%%%%%%
\documentclass[a4paper,UKenglish,cleveref, autoref, thm-restate]{lipics-v2021}
%\usepackage{semantic}
%\usepackage[margin=0.6in]{geometry}
%% Some recommended packages.
\usepackage{booktabs}   %% For formal tables:
                        %% http://ctan.org/pkg/booktabs
\usepackage{subcaption} %% For complex figures with subfigures/subcaptions
                        %% http://ctan.org/pkg/subcaption
\usepackage{bussproofs}
\usepackage[cal=boondoxo]{mathalfa}
\DeclareMathAlphabet{\mathpzc}{OT1}{pzc}{m}{it}
\usepackage{amsmath}
%\newtheorem{theorem}{Theorem}[section]
%\newtheorem{observation}[theorem]{Observation}
\usepackage{color}
\usepackage{xspace}
\input{macros}
\usepackage{bm}
\usepackage{amssymb}
\input{preamble}
\usepackage{pifont}% http://ctan.org/pkg/pifont

\newcommand{\cmark}{\text{\ding{51}}}
\newcommand{\xmark}{\text{\ding{55}}}

% \usepackage{forest}
% Using the geometry package with a small
% page size to create the article graphic

\bibliographystyle{plainurl}

\title{The Quantum Abstract Machine}  

%\titlerunning{Dummy short title} %TODO optional, please use if title is longer than one line

\author{Liyi Li and Le Chang and Mingwei Zhu and Rance Cleaveland}{ }{}{}{}%TODO mandatory, please use full name; only 1 author per \author macro; first two parameters are mandatory, other parameters can be empty. Please provide at least the name of the affiliation and the country. The full address is optional. Use additional curly braces to indicate the correct name splitting when the last name consists of multiple name parts.

%\author{Joan R. Public\footnote{Optional footnote, e.g. to mark corresponding author}}{Department of Informatics, Dummy College, [optional: Address], Country}{joanrpublic@dummycollege.org}{[orcid]}{[funding]}

\authorrunning{L. Li and L. Chang and M. Zhu and R. Cleaveland} %TODO mandatory. First: Use abbreviated first/middle names. Second (only in severe cases): Use first author plus 'et al.'

\Copyright{L. Li and L. Chang and M. Zhu and R. Cleaveland} %TODO mandatory, please use full first names. LIPIcs license is "CC-BY";  http://creativecommons.org/licenses/by/3.0/

\ccsdesc[100]{} %TODO mandatory: Please choose ACM 2012 classifications from https://dl.acm.org/ccs/ccs_flat.cfm 

\keywords{Quantum Computation, Abstract Machine} %TODO mandatory; please add comma-separated list of keywords

\category{} %optional, e.g. invited paper

%\relatedversion{} %optional, e.g. full version hosted on arXiv, HAL, or other respository/website
%\relatedversiondetails[linktext={opt. text shown instead of the URL}, cite=DBLP:books/mk/GrayR93]{Classification (e.g. Full Version, Extended Version, Previous Version}{URL to related version} %linktext and cite are optional

%\supplement{}%optional, e.g. related research data, source code, ... hosted on a repository like zenodo, figshare, GitHub, ...
%\supplementdetails[linktext={opt. text shown instead of the URL}, cite=DBLP:books/mk/GrayR93, subcategory={Description, Subcategory}, swhid={Software Heritage Identifier}]{General Classification (e.g. Software, Dataset, Model, ...)}{URL to related version} %linktext, cite, and subcategory are optional

%\funding{(Optional) general funding statement \dots}%optional, to capture a funding statement, which applies to all authors. Please enter author specific funding statements as fifth argument of the \author macro.

%\acknowledgements{I want to thank \dots}%optional

%\nolinenumbers %uncomment to disable line numbering



%Editor-only macros:: begin (do not touch as author)%%%%%%%%%%%%%%%%%%%%%%%%%%%%%%%%%%
\EventEditors{Liyi Li and Le Chang and Mingwei Zhu and Rance Cleaveland}
\EventNoEds{2}
\EventLongTitle{34th International Conference on Concurrency Theory (Concur 2023)}
\EventShortTitle{Concur 2023}
\EventAcronym{Concur}
\EventYear{2023}
\EventDate{September 18 -- 23, 2023}
\EventLocation{Antwerp, Belgium}
\EventLogo{}
\SeriesVolume{34}
\ArticleNo{23}
%%%%%%%%%%%%%%%%%%%%%%%%%%%%%%%%%%%%%%%%%%%%%%%%%%%%%%
\begin{document}
\maketitle

\begin{abstract}
Quantum Abstract Machine
\end{abstract}

\newcommand{\qsend}[3]{#1\texttt{.}#2\texttt{!<}#3\texttt{>}}
\newcommand{\qsenda}[2]{#1\texttt{!<}#2\texttt{>}}
\newcommand{\qrev}[2]{#1\texttt{?(}#2\texttt{)}}
\newcommand{\qact}[2]{#1.#2}

\newcommand{\qchan}[3]{#1\texttt{[}#2\texttt{]}.#3}
\newcommand{\qchana}[3]{#1\texttt{[}#2\texttt{]}\,@\,#3}

\newcommand{\parp}[2]{#1\,\texttt{@}\,#2}

\newcommand{\parl}[2]{#1\,\texttt{|}\,#2}

\newcommand{\comp}[1]{\texttt{!}#1}

\newcommand{\pcell}[3]{\langle #1 \rangle^{#2}_{#3}}

\newcommand{\parll}[2]{#1\,\texttt{||}\,#2}

% \newcommand{\alchan}[2]{#1\,\triangleleft\,#2}

% \newcommand{\chansol}[1]{\texttt{\{|}#1\texttt{|\}}}

\newcommand{\mcomm}[2]{\texttt{[}#1\texttt{]}#2}

\newcommand{\Ss}{\mathbb{S}}
\newcommand{\Rs}{\mathbb{R}}
\newcommand{\Ps}{\mathbb{P}}
\newcommand{\Ls}{\mathbb{L}}
\newcommand{\Cs}{\mathcal{C}}
\newcommand{\Fs}{\mathcal{F}}
\newcommand{\Es}{\mathcal{E}}
\newcommand{\Os}{\mathcal{O}}
\newcommand{\rules}{\mathcal{R}}
\newcommand{\Path}[1]{<#1> }
\newcommand{\decl}[2]{\texttt{decl}\;#1\texttt{ in }#2}
\newcommand{\leta}[3]{\texttt{let}\;#1\texttt{ = }#2\texttt{ in }#3}
\newcommand{\Cfg}[3]{\langle #1\texttt{,}#2\texttt{,}#3 \rangle}
\newcommand{\Seta}[2]{#1\texttt{@}#2}
\newcommand{\PSet}[1]{\texttt{\{}#1\texttt{\}}}

\newcommand{\qcell}[2]{\langle #1 \rangle_{#2}}
\newcommand{\ccell}[1]{\langle #1 \rangle}
\newcommand{\Cella}[3]{\langle #1 \rangle^{#2}_{#3}}
\newcommand{\Cellb}[3]{\langle #1 \, \;...\rangle^{#2}_{#3}}
\newcommand{\cn}[1]{\texttt{#1}}




\section{Introduction} \label{sec:introduction}

\begin{itemize}
    \item Importance of quantum computing and communication
    \item Need at present to be a quantum-mechanics expert to understand quantum computing
    \item This work is intended to give an operational account of quantum interaction in a process-algebraic style, inspired by the Chemical Abstract Machine
\end{itemize}

Quantum computers offer unique capabilities that can be used to
program substantially faster algorithms compared to those written for
classical computers. For example, Grover's search algorithm \cite{grover1996,grover1997}
can query unstructured data in sub-linear time (compared to linear
time on a classical computer), and Shor's algorithm \cite{shors} can factorize a
number in polynomial time (compared to sub-exponential time for the
best known classical algorithm).
Additionally, quantum computing provides a secured communication system that can transmit
information without the possibility of eavesdropping.
For example, quantum teleportation can safely communicate a qubit information between two parties.

It is more and more essential to build a network based on the combination of quantum computing
and the current classical network facilities to provide extra secured information transformation.
Several classical quantum hybrid network protocols are proposed \cite{10.1145/3387514.3405853,https://doi.org/10.48550/arxiv.2205.08479,8068178,e24101488}.
These protocols represent the first step for developing reliable classical quantum hybrid network (CQHN).
It is necessary to develop a CQHN protocol framework that allow
researchers to create efficient and correct CQHN protocols. 
Many previous frameworks \cite{10.1145/1040305.1040318,9165801} were a quantum imitation of classical process algebra, by incorporating classical process algebra framework with quantum circuit operations.
For example, CQP \cite{10.1145/1040305.1040318} instantiated CSP \cite{Hoare:1985:CSP:3921} with a quantum circuit language.

There are two major issues associated with the previous frameworks.
First, the beauty of process algebra is to define a mathematical model that captures the essence of
multi-threaded program semantics; so that program executions are easily viewed as 
automata transitions -- usually associated with automata based model checking mechanisms.
Incorporating quantum circuit languages significantly complicates the program semantics of the previous frameworks. Thus, the associated verification framework is unnecessarily complicated.
For example, defining a simple quantum teleportation in CQP has even more complicated structure than the original circuit describing quantum teleportation.
Second, quantum circuit semantics is unintuitive, and incorporating quantum semantics with process algebra makes the system even more unintuitive.
Eventually, programmers, who might have a brief idea about quantum computation, need to use a given protocol framework to define CQHN network protocols. If they spent most of the time working out how the unintuitive framework works, why will they use the framework?
Unfortunately, previous CQHN frameworks had unintuitive program semantics because they basically put quantum circuit languages together with multi-threaded process algebra together with no chemistry. 

In this paper, we introduce quantum abstract machine (QAM), permitting the definition of CQHN protocols based on classical chemical abstract machine. The key aspect in QAM's design identifies a set of properties enough to describe quantum network protocols without including quantum program operational semantics in the system.
QAM leverages the fact that quantum network protocols are all based on multi-parties of quantum teleportation.
Instead of defining the operational semantics for describing quantum teleportation, such as creating entanglement and qubit measurmwent, we obverse the mathmetical properties involving in the procedure and define them as conceptual mathemtical terms in our system.
We identify a list of contributions:

\begin{itemize}
    \item Design quantum abstract machine suitable for defining CQHN protocols, which can be easily interpreted as automata.
    \item Design a trace refinement framework based on the QAM language, which is useful for verifying if two protocols are equivalent. 
    \item We utilize QAM to define QPASS and QCAST network protocols \cite{10.1145/3387514.3405853} in a simple automata structure, and we use the trace refinement framework to verify that the QPASS protocol is an instance of the QCAST protocol. 
\end{itemize}

\liyi{move the beginning of section 3 to intro.}
First, there are two main tasks in sending quantum messages, message transmissions and deliveries.
The former uses the quantum swaps to convey a message from a node to another which is closer to the final destination and connectivity is usually represented as a graph structure such as the one in \Cref{fig:q-pi-example}.
During the procedure, the actual quantum swap circuit detail is less interesting than the cost analysis, where each two-node message transmission costs each node one qubit resource, because how swaps are constructed in circuit is almost identical in different protocols.
We also use a quantum teleportation circuit for delivering a message. Again, the more interesting analysis is about different guarantees a network protocol can provide in the message deliveries rather than the circuit detail.

Next, many quantum message sending guarantees are time-sensitive and related to probabilities that are time-sensitive.
Thus, we model in QAM a global clock and each task step, a message transmission or delivery, costs one time slot,
and every message is associated with a probability value defining the success likelihood of delivering the message.
For example, sending a message from \cn{Cat} to \cn{Dan} in \Cref{fig:q-pi-example} needs an intermediate transmission node $r_1$, and the likelihood of successfully delivering the message is obviously lower than the case if we can directly send the message from \cn{Cat} to \cn{Dan} without an intermediate step. In addition, qubits might be decohered after a certain period of time, so a property we can define,
based on the QAM system, is to guarantee that every delivered message is within a threshold time period.


\section{Background}
\label{sec:background}

We begin with some background on quantum computing and quantum algorithms. 

\noindent\textbf{\textit{Quantum States.}} A quantum state consists of one or more quantum bits (\emph{qubits}). A qubit can be expressed as a two dimensional vector $\begin{psmallmatrix} \alpha \\ \beta \end{psmallmatrix}$ where $\alpha,\beta$ are complex numbers such that $|\alpha|^2 + |\beta|^2 = 1$.  The $\alpha$ and $\beta$ are called \emph{amplitudes}. 
%
We frequently write the qubit vector as $\alpha\ket{0} + \beta\ket{1}$ where $\ket{0} = \begin{psmallmatrix} 1 \\ 0 \end{psmallmatrix}$ and $\ket{1} = \begin{psmallmatrix} 0 \\ 1 \end{psmallmatrix}$ are \emph{computational basis states}. When both $\alpha$ and $\beta$ are non-zero, we can think of the qubit as being ``both 0 and 1 at once,'' a.k.a. a \emph{superposition}. For example, $\frac{1}{\sqrt{2}}(\ket{0} + \ket{1})$ is an equal superposition of $\ket{0}$ and $\ket{1}$. 

We can join multiple qubits together to form a larger quantum state with the \emph{tensor product} ($\otimes$) from linear algebra. For example, the two-qubit state $\ket{0} \otimes \ket{1}$ (also written as $\ket{01}$) corresponds to vector $[~0~1~0~0~]^T$. 
Sometimes a multi-qubit state cannot be expressed as the tensor of individual states; such states are called \emph{entangled}. One example is the state $\frac{1}{\sqrt{2}}(\ket{00} + \ket{11})$, known as a \emph{Bell pair}.
Entangled states lead to exponential blowup: A general $n$-qubit state must be described with a $2^n$-length vector, rather than $n$ vectors of length two.

\begin{figure}[t]
{\centering
\hspace*{-1cm}
          \begin{minipage}[b]{.22\textwidth}
            {\qquad
              \footnotesize
              \Qcircuit @C=0.5em @R=0.5em {
                \lstick{\ket{0}} & \gate{H} & \ctrl{1} & \qw &\qw & & \dots & \\
                \lstick{\ket{0}} & \qw & \targ & \ctrl{1} & \qw & &  \dots &  \\
                \lstick{\ket{0}} & \qw & \qw   & \targ & \qw & &  \dots &  \\
                & \vdots &   &  &  & & & \\
                & \vdots &  & \dots & & & \ctrl{1} & \qw  \\
                \lstick{\ket{0}} & \qw & \qw & \qw &\qw &\qw & \targ & \qw
              }
            }
\caption{GHZ}
\label{fig:circuit-example}
\end{minipage}
\hfill
\begin{minipage}[b]{.37\textwidth}
                 \includegraphics[width=1\textwidth]{tele_circuit.png}
            \caption{Teleportation Circuit}
            \label{fig:background-circuit-examplea}
 \end{minipage}
\hfill
\begin{minipage}[b]{.44\textwidth}
                 \includegraphics[width=1\textwidth]{teleportation.png}
            \caption{Teleportation Diagram}
            \label{fig:background-circuit-exampleb}
          \end{minipage}
}
\end{figure}

\begin{figure}[t]
            {

         }
  \label{fig:background-circuit-example}
\end{figure}

\noindent\textbf{\textit{Quantum Computations and the First Quantum Abstraction.}} High-level quantum programming languages are usually designed to follow the \emph{QRAM model}~\cite{Knill1996}, where quantum computers are used as co-processors to classical computers. The classical computer generates descriptions of circuits to send to the quantum computer and then processes the measurement results.
Computation on a quantum value state consists of a series of \emph{quantum operations}, each of which acts on a subset of qubits in the value state. 
In the standard presentation, quantum computations are expressed as \emph{circuits}, as shown in \Cref{fig:circuit-example},
which takes a qubit array and constructs a circuit that prepares the Greenberger-Horne-Zeilinger (GHZ) state \cite{Greenberger1989}, which is an $n$-qubit entangled quantum state of the form
{
\begin{center}
$
    \ket{\text{GHZ}^n} = \frac{1}{\sqrt{2}}(\ket{0}^{\otimes n}+\ket{1}^{\otimes n}).
$
\end{center}
}
In these circuits, each horizontal wire represents a qubit and boxes on these wires indicate quantum operations, or \emph{gates}. 
Applying a gate to a state \emph{evolves} the state. The traditional semantics of doing so is expressed by multiplying the state vector by the gate's corresponding matrix representation; single-qubit gates are 2-by-2 matrices, and two-qubit gates are 4-by-4 matrices. A gate's matrix must be \emph{unitary}, ensuring that it preserves the unitarity invariant of quantum states' amplitudes. An entire circuit can be expressed as a matrix by composing its constituent gates.
The GHZ circuit applies a \emph{Hadamard} (\coqe{H}) to turn the first qubit into superposition. The very next \emph{controlled-not} (\coqe{CNOT}) gate application entangles the second qubit with the superposition qubit; thus, creates a Bell pair.
Each of the next series of \emph{controlled-not} (\coqe{CNOT}) gate applications transitively moves one qubit to join the entanglement cluster and finally creates the $\ket{\text{GHZ}^n}$ state above. Essentially, a $\ket{\text{GHZ}^n}$ state is a generalized version of Bell pairs.

A \emph{measurement} operation extracts classical information from a quantum state, typically when a computation completes. Measurement collapses the state to a basis states with a probability related to the state's amplitudes. For example, measuring $\frac{1}{\sqrt{2}}(\ket{0} + \ket{1})$ in the computational basis will collapse the state to $\ket{0}$ with probability $\frac{1}{2}$ and likewise for $\ket{1}$, returning classical value $0$ or $1$, respectively.
The circuit representations are typically expressed as the \coqe{meas} gates in \Cref{fig:background-circuit-examplea} and the $\mathpzc{M}$ gates in \Cref{fig:background-circuit-exampleb}.

The above examples indicate that most current quantum programming languages tend to focus on the circuit level physical details of quantum computation behaviors rather than providing high level abstractions \cite{VOQC}.
For abstracting away circuit details but keeping the essence of quantum behaviors,
one really needs to focus on the utilities of different small quantum algorithms, such as GHZ,
because quantum computations for solving a task are typically constructed based on a limited set of small algorithms.
For quantum network communications, the essence of GHZ provides a quantum channel to connect different parties.
If we view the $n$ different qubits in GHZ as $n$ different parties,
measuring any qubit resulting in a bit, either $0$ or $1$, makes other parties consensually produce the same bit.
Thus, the GHZ circuit can be thought abstractly as a channel to agree on all parties.
We will see two more small algorithm abstractions below.

\noindent\textbf{\textit{Quantum Network based on Teleportation.}}
Quantum teleportation \cite{PhysRevLett.70.1895,Rigolin_2005} is an algorithm to communicate a quantum message, a single qubit or multiple qubits, between two parties.
Almost all quantum network protocols are originated from the algorithm.
The circuit diagram in \Cref{fig:background-circuit-exampleb} describes a quantum teleportation circuit for communicating a qubit message between Alice and Bob.
In the circuit in \Cref{fig:background-circuit-examplea},
Alice holds a quantum qubit message, represented as the first qubit line, and a $\ket{0}$ qubit;
while Bob holds the third $\ket{0}$ qubit.
The application \coqe{bel1000} creates a quantum channel (Bell pair) between Alice's second qubit and Bob's qubit.

The \coqe{alice} application in \Cref{fig:background-circuit-examplea} does two tasks: 1) it pushes Alice's qubit message $\varphi$ to attach to the quantum channel between Alice and Bob; and 2) Alice then measures her two qubits, which destroys the entangled three qubits and results in two classical bits held by Alice and one classical bit held by Bob.
The \coqe{bob} application also has two tasks: 1) it uses the bold classical message channel to send Alice's two classical bits to Bob; and 2) Bob receives the two classical bits and uses a proper devices, the \coqe{X} and \coqe{Z} gates, to recover the message $\varphi$, stored in Bob's qubit.

Regardless the circuit details, the quantum teleportation procedure can be summarized as five steps:
1) we create a quantum channel between Alice and Bob;
2) we push Alice's quantum message to the quantum channel;
3) Alice measures and destroys her two qubits, so Alice has two classical remains while Bob has one;
4) Alice sends Bob two classical bits;
and 5) Bob recovers the quantum message based on the three classical bits.

The multiple qubit version of quantum teleportation \cite{Rigolin_2005} is proportional to the single qubit version
and has the same procedure.
Since every quantum network protocol utilizes the algorithm, we model these five steps as five operations in QAM.

\liyi{Le: not very clear. need to add some pictures and explain in details how swaps work.}
\noindent\textbf{\textit{Extended Quantum Network Communication By Quantum Routing.}}
Quantum teleportation describes the circuit details of communicating quantum messages through two parties.
For a long distance communication, the quantum routing concept is involved.
A typical and near term possible technique to send quantum information through long distance is \textit{entanglement swap},
which relies on intermediate quantum repeaters between the two parties. Suppose Alice and Bob stay in two separate places and there is a quantum repeater, $r_1$ between them, they can create a long-distance entanglement in two steps. First, Alice and Bob each create an entanglement state with $r_1$; then $r_1$ performs a Bell state measurement locally. After measurement, the two original entanglements are destroyed and therefore, a new entanglement pair is created between Alice and Bob. 

Through the entanglement swaps, quantum messages can be sent through long-distance, with a consumption of qubit resources and reduction of success rate of transmitting the messages, because each entanglement pair destruction and creation requires qubit resources and has a certain percentage of failure. 
Since almost all quantum network protocols utilize the quantum teleportation and quantum routing as the guidance to send quantum messages, in QAM, we model quantum teleportation as the step to receive a message, and quantum routing as the step to transmit a message through a long distance by consuming some qubit resources and reducing some success rate.
% \liyi{later.}

\liyi{Le: need better description. Please help find the CHEM paper and modify. }
\noindent\textbf{\textit{Chemical Abstract Machine.}} Chemical abstract machine \cite{BERRY1992217} models multi-party communication behaviors as chemical reactions, where processes live in molecule membrane structures, named \textit{cells} in QAM, and they can react by the reaction rules defining how molecule membranes can connect together and processes living inside a membrane can also communicate through the heating and cooling rules, 
defining the transformation between process active and inactive states, where active and inactive states represent processes being able to communicate or not.

% Writing a quantum algorithm now, with SQIR (but likewise with Quipper, Pyquil, Circ, etc.). Example: Shor’s
% Write quantum parts (QPE) 
% Classical parts via library functions in meta-language (Modular multiplier)
% Refer to particular quipper functions, e.g., for adding, subtraction, etc.
% https://www.mathstat.dal.ca/~selinger/quipper/doc/Quipper-Libraries-QFTAdd.html qft_add_in_place :: QDInt -> QDInt -> Circ (QDInt, QDInt), Add one QDInt onto a second, in place; i.e. (x,y) ↦ (x,x+y). Arguments are assumed to be of equal size. This implementation follows the implementation in Thomas G. Draper's paper "Addition on a Quantum Computer" which doesn't require the use of any ancilla qubits through a clever use of the quantum Fourier transform.
% Mention Q# too
% https://github.com/microsoft/QuantumLibraries/tree/main/Numerics/src
% Maybe Scaffold:
% Write oracles via “RKQC intrinsic” functions (see sec 4.1 of https://iopscience.iop.org/article/10.1088/2058-9565/ab8c2c/pdf). The intrinsics they talk about here are super simple (swap two registers or add two registers together)
% Quipper: Write in Haskell, build_circuit, is better
% Problems to solve: Efficient compilation, verification of that compilation
% Verification: Prior work with ReverC, but only classical gates


\section{The Quantum Abstract Machine:  Syntax and Semantics} \label{sec:qam}

\begin{figure}[t]
{\small
  \[\begin{array}{llcl} 
      \texttt{Variable} & x,y \\
      \texttt{Probability} & p &\in &\Rs\\
      \texttt{Message} & m &\in& \mathbb{M}\\
    \texttt{Channel} & c &\in& \Ls\\
    \texttt{Time Stamp} & t &\in& \mathbb{T}\\
    \texttt{Label} & a &\in& \Ls \times \mathbb{M}\\
      \texttt{Singleton Action} & A & ::= & \qsend{p}{c}{m}^{t?} \mid \qrev{c}{x} \\
      \texttt{Process} & P,Q & ::= & 0 \mid A.P \mid \parl{P}{Q} \mid \comp{P} \\
      \texttt{Relations} & R & \in & \Ls \times \Ls \times \Rs \\
      \texttt{Predicate Action} & \alpha & ::= & \qsend{p}{c}{m}^t \mid (c,c,\qsend{p}{c}{m}^t) \\
      \texttt{Objects} & \Os \\
      \texttt{Custom Cell} & \varphi,\psi & ::= & \pcell{P}{n}{c} \mid \qcell{\Os}{c}\\
      \texttt{Configuration} & C & ::= & \varphi^* \qcell{R}{\cn{rel}} \qcell{t}{\cn{gt}}
                             \qcell{\alpha}{\cn{comm}} \pcell{\cn{bool}}{t}{\cn{pred}} \psi^* \\
      \texttt{Rules} & \rules & ::= & C \longrightarrow C \\
    \end{array}
  \]
}
\caption{Quantum Abstract Machine Syntax Table}
  \label{fig:q-pi-syntax}
\end{figure}

This section describes QAM's syntax and semantics,
whose design is based on two observations of different quantum network protocols, through an example connectivity diagram in \Cref{fig:q-pi-example}.

A QAM system is a structure $(\Ms,\Ls,\Ts,C,\overline{\rules})$,
where $\Ms$ is a set of messages;
$\Ls$ is a set of channels containing at least $\cn{rel}$, $\cn{comm}$, $\cn{gt}$, and $\cn{pred}$;
$\Ts$ is a set of time stamps that forms a linear order ($<$) with $0$ being the minimum;
$C$ is the configuration containing different cells that represent the nodes with processes and environment for the system, at least containing cell $\cn{rel}$, $\cn{comm}$, $\cn{gt}$, and $\cn{pred}$, mentioned in \Cref{sec:qamsyntax};
and $\overline{\rules}$ is a set of rule for guiding how the configuration is transited, at least contain rules \rulelab{CT},\rulelab{GC}, \rulelab{MT}, \rulelab{NT}, \rulelab{PC}, and \rulelab{Com} in \Cref{fig:q-pi-semantics}. 




\subsection{Syntax} \label{sec:qamsyntax}

QAM describes multiparty communication behaviors,
whose syntax is similarity to the chemistry abstract machine \cite{BERRY1992217} and $\Pi$-calculus.
The message label that is used to communicate different parties has the form $c.m$, 
where $c$ refers to a communication channel, and $m$ is a message, possibly a quantum state.
Each single process action can be a message sending $\qsend{p}{c}{m}^{t?}$, meaning that a message $m$ is sent through the channel $c$ with the success probability rate $p$ initialized at the time stamp $t$, and a message receipt $\qrev{c}{x}$, referring to that a message is received through the channel $c$ and represented as variable $x$.
The time stamp $t$ in a message sending operation can be omitted as $\qsend{p}{c}{m}$, meaning that the time stamp is generated when the message is sent out from its starting place.

To describe the behaviors of a network system, every QAM system comes with a configuration $C$
containing two different kinds of cells --- process and content cells.
A configuration might contain one or more process cells $\pcell{P}{n}{c}$, representing the execution maltreated process of a local node with name $c$, such as the $\cn{Ann}$ and $r_1$ nodes in \Cref{fig:q-pi-example},
and $n$ is the available qubit resource in the node.
A multi-threaded process $P$ is the program actions that node $c$ can perform, which has similar syntax as $\Pi$-calculus.
Each process might contain multiple threads competing for executions, which are separated by a parallel operation $\cn{|}$.
Each single thread is a sequence of singleton actions that ends at the unit process $0$,
and $\comp{P}$ is a replication process that repeatedly executes the process $P$, 
which is used for resending messages, explained in \Cref{sec:qamsemantics}.

In QAM, a parallel process of a message sending and receipt, $\parl{\qact{\qsend{p}{c}{m}^t}{P}}{\qact{\qrev{c}{x}}{Q}}$,
causes the process to transit to $\parl{P}{Q[m/x]}$, which is similar to the $\Pi$-calculus semantics.
On the other hand, a parallel of two message sending, $\parl{\qact{\qsend{p}{c}{m}^t}{P}}{\qsend{p'}{c'}{m'}^{t'}{P'}}$,
refers to that the two operations are competing to transmit the message outside of the node, possibly to other destinations.

The content cells act as control units for storing environment information for executing a QAM system.
Users are able to define their own content cell by instantiating object types $\Os$ to different user-defined types.
In QAM, we require at least four kinds of content cells: 
1) $\qcell{R}{\cn{rel}}$ is a relation cell defining the probability value to send out one message from a node to the other;
2) $\qcell{t}{\cn{gt}}$ is a global clock cell storing the global time stamp;
3) $\qcell{A}{\cn{comm}}$ is a communication cell storing the message about to communicate;
and 4) $\pcell{\cn{bool}}{t}{\cn{pred}}$ is a predicate cell 
determining if a message communication can happen ($\texttt{true}$ in the cell) or not ($\texttt{false}$ in the cell) at time $t$.

Defining protocol systems might require the extension of semantic rule set to include specific behaviors.
We provide a meta level rule declaration operation ($\decl{\rules}{\Cs}$), allowing users to declare a new rule $\rules$ used in the system $\Cs$. 

{\footnotesize
\[
\Cella{\qsend{1}{c}{m}}{10}{\cn{Cat}}\Cella{0}{10}{r_1}
\Cella{\qrev{c}{x}.0}{10}{\cn{Dan}} 
\qcell{\{(\cn{Cat},r_1,0.5), (r_1,\cn{Dan},0.5)\}}{\cn{rel}}
\qcell{\emptyset}{\cn{comm}}
\pcell{\texttt{false}}{0}{\cn{pred}}
\qcell{0}{\cn{gt}}
\]
}

As an example of the QAM syntax, the above configuration defines the initial program state for sending a message $c.m$ from $\cn{Cat}$ to $\cn{Dan}$ via the router $r_1$, as part of the communication in \Cref{fig:q-pi-example}. Initially, the relation cell stores the connectivity between \cn{Cat}, \cn{Dan}, and $r_1$, with the success rates. 
The global time is initialized as $0$ in the \cn{gt} cell, and the predicate cell has a fixed value $\texttt{false}$.
The message ($\qsend{1}{c}{m}$) sent from \cn{Cat} has an initial probability value $1$.
Node $r_0$ acts as a intermediate router, so it only contains the unit process $0$, and \cn{Dan} is waiting on receiving a message ($\qrev{c}{x}.0$). 

\begin{figure}[t]
{\small
  \begin{mathpar}
\mprset{flushleft}
   \inferrule[GC]{}
       {\Cellb{\qsend{p}{c}{m}^t}{i}{c_1}\Cella{P}{j}{c_2} \qcell{\{(c_1,c_2,p')\}\cup R}{\cn{rel}}\qcell{t'}{\cn{gt}}
        \qcell{\emptyset}{\cn{comm}}\pcell{b}{\_}{\cn{pred}}
       \\\\\qquad\qquad \longrightarrow \Cellb{}{i-}{c_1}\Cella{\qsend{p*p'}{c}{m}^{t}\texttt{|}P}{j-}{c_2}
              \qcell{\{(c_1,c_2,p')\}\cup R}{\cn{rel}}\qcell{t'}{\cn{gt}}
               \qcell{(c_1,c_2,\qsend{p}{c}{m}^t)}{\cn{comm}}\pcell{b}{t'}{\cn{pred}}}

\ignore{
   \inferrule[GenQubit]{}
       {\Cella{P}{n,t'}{a}\qcell{t}{\cn{gt}}\longrightarrow \Cella{P}{n+,t}{a}\qcell{t}{\cn{gt}}}\;\;\texttt{when}\;t \texttt{|} \beta\wedge t' < t
}

   \inferrule[CT]{}
       {\Cellb{\qsend{p}{c}{m}}{i}{c_1}\qcell{t}{\cn{gt}} \longrightarrow \Cellb{\qsend{p}{c}{m}^t}{i}{c_1}\qcell{t}{\cn{gt}}}

   \inferrule[MT]{}
       {\comp{P} \longrightarrow \parl{P}{\comp{P}}}
      
   \inferrule[NT]{}
       {\comp{P} \longrightarrow 0}

  \inferrule[PC]{}
      { \Cellb{\qsend{p}{c}{m}^t\texttt{|} \qrev{c}{x}.P}{n}{c}\qcell{t'}{\cn{gt}}\qcell{\emptyset}{\cn{comm}}\pcell{b}{\_}{\cn{pred}}
           \longrightarrow
         \Cellb{P[m/x]}{n}{c}\qcell{t'}{\cn{gt}}\qcell{\qsend{p}{c}{m}^t}{\cn{comm}}\pcell{b}{t'}{\cn{pred}}}
                  
  \inferrule[Com]{}
      { \qcell{t'}{\cn{gt}}\qcell{\qsend{p}{c}{m}^t}{\cn{comm}}\pcell{\cn{true}}{t'}{\cn{pred}}
           \xrightarrow{p.c.m}  
         \qcell{t'+}{\cn{gt}}\qcell{\emptyset}{\cn{comm}}\pcell{\cn{false}}{t'}{\cn{pred}} } 

  \inferrule[FC]{}
      { \qcell{t}{\cn{gt}}\qcell{(c_1,c_2,A)}{\cn{comm}}\pcell{\cn{true}}{t}{\cn{pred}}
           \longrightarrow
         \qcell{t+}{\cn{gt}}\qcell{\emptyset}{\cn{comm}}\pcell{\cn{false}}{t}{\cn{pred}} } 

  \end{mathpar}
}
\caption{Quantum Pi Semantics. $\beta$, $\mu$, and $\nu$ are globally defined for the qubit generation period, the message threshold probability, and message sending finished threshold. $\Cella{P}{n}{a}$ refers to that the $t$ in $\Cella{P}{n,t}{a}$ is omitted in the rule.}
  \label{fig:q-pi-semantics}
\end{figure}

\ignore{
\begin{figure}[t]
{\small
{\hspace*{-2em}
\begin{tikzpicture}[align=center,node distance=1.5cm and -1cm, thick] 
\node (1) {S$\langle\{(a,r_1,0.5), (a,r_2,0.5)\}\cup$R$\rangle$}; 
\node (2) [below left= of 1] {$\Cella{0}{9}{a}$ $\Cella{\qsend{0.5}{c}{m}|0}{9}{r_1}$... $\langle\{(r_1,r_4)\}\cup$R$\rangle$}; 
\node (3) [below right= of 1] {\text{\ \ \ \ \ \ }$\Cella{0}{9}{a}$ $\Cella{\qsend{0.5}{c}{m}|0}{9}{r_2}$..., $\ccell{\{(r_2,r_3)\}\cup\text{R}}$}; 
\node (4) [below of=2] {$\Cella{0}{8}{r_1}$ $\Cella{\qsend{0.25}{c}{m}|0}{9}{r_4}$... $\langle\{(r_4,b)\}\cup$R$\rangle$};
\node (5) [below of=3] {\text{\ \ \ \ \ \ }$\Cella{0}{8}{r_2}$ $\Cella{\qsend{0.25}{c}{m}|0}{9}{r_3}$..., $\ccell{\{(r_3,r_4)\}\cup\text{R}}$};
\node (6) [below of=4] {$\Cella{0}{8}{r_4}$ $\Cella{\qsend{0.125}{c}{m}|\qrev{c}{x}.0}{9}{b}$... $\ccell{\text{R}}$};
\node (7) [below of=5] {\text{\ \ \ \ \ \ \ \ \;}$\Cella{0}{8}{r_3}$ $\Cella{\qsend{0.125}{c}{m}|0}{9}{r_4}$..., $\ccell{\{(r_4,b)\}\cup\text{R}}$};
\node (8) [below of=6] {$\Cella{0}{9}{b}$... $\ccell{\text{R}}$};
\node (9) [below of=7] {\text{\ \ \ \ \ \ \ \ }$\Cella{0}{8}{r_4}$ $\Cella{\qsend{0.0625}{c}{m}|\qrev{c}{x}.0}{9}{b}$..., $\ccell{\text{R}}$};
\node (10) [below of=9] {\text{\ \ \ \ \ \ }$\Cella{0}{n}{b}$... $\ccell{\text{R}}$};
\draw[->] (1) -- node[midway, above left] {} (2); 
\draw[->] (1) -- node[midway, above right] {} (3); 
\draw[->] (2) -- node[midway, right] {} (4); 
\draw[->] (4) -- node[midway, right] {} (6);
\draw[->] (6) -- node[midway, right] {$0.125.c.m$} (8); 
\draw[->] (3) -- node[midway, right] {} (5); 
\draw[->] (5) -- node[midway, right] {} (7); 
\draw[->] (7) -- node[midway, right] {} (9);
\draw[->] (9) -- node[midway, right] {$0.0625.c.m$}  (10); 
\end{tikzpicture} 
}
}
\caption{Quantum Machine Transitions for \Cref{fig:q-pi-example}}
  \label{fig:q-pi-example1}
\end{figure} 
}

\begin{figure}[t]
{\small
\begin{center}
\begin{tikzpicture}[node distance={1cm}, thick, main/.style = {draw, circle}] 
\node[main] (1) {\cn{Ann}}; 
\node[main] (2) [right of=1] {$r_1$};
\node[main] (3) [right of=2] {$r_4$};
\node[main] (4) [right of=3] {\cn{Bob}};
\node[main] (5) [above right=0.5cm and 1.5cm of 1] {$r_2$};
\node[main] (6) [right of=5] {$r_3$};
\node[main] (7) [below of=1] {\cn{Cat}};
\node[main] (8) [below of=3] {\cn{Dan}};
\draw[-] (1) --  (2);
\draw[-] (2) --  (3);
\draw[-] (3) --  (4);
\draw[-] (1) --  (5);
\draw[-] (5) --  (6);
\draw[-] (6) --  (3);
\draw[-] (7) --  (2);
\draw[-] (2) --  (8);
\end{tikzpicture}
\end{center}
}
\caption{Example Path Connectivity}
  \label{fig:q-pi-example}
\end{figure}

\subsection{Semantics} \label{sec:qamsemantics}

The QAM semantics is given in terms of the definition of a QAM system,
which is a structure $(\Ms,\Ls,\Ts,C,\overline{\rules})$,
where $\Ms$ is a set of messages;
$\Ls$ is a set of channels containing at least $\cn{rel}$, $\cn{comm}$, $\cn{gt}$, and $\cn{pred}$;
$\Ts$ is a set of time stamps that forms a linear order ($<$) with $0$ being the minimum;
$C$ is the configuration containing different cells that represent the nodes with processes and environment for the system, at least containing cell $\cn{rel}$, $\cn{comm}$, $\cn{gt}$, and $\cn{pred}$, mentioned in \Cref{sec:qamsyntax};
and $\overline{\rules}$ is a set of rule for guiding how the configuration is transited, at least contain rules \rulelab{CT},\rulelab{GC}, \rulelab{MT}, \rulelab{NT}, \rulelab{PC}, and \rulelab{Com} in \Cref{fig:q-pi-semantics}. 

QAM is flexible enough to define most quantum network/security protocols 
through the ability to extend rules by the rule declaration operation in \Cref{fig:q-pi-syntax}.
Every rule set must contain the six rules listed in \Cref{fig:q-pi-semantics},
because they represent the universally sound facts about multiparty quantum communication. 
We first discuss these six rules through the example configuration at the end of \Cref{sec:qamsyntax}. 
For showing the consecutive transitions of a system, there must be additional rules for
granting every step of transition by turning the \cn{pred} cell's content to be \cn{true}.
Here, we first assume our example system includes the simplest granting rule by assigning \cn{true} to the \cn{pred} cell if the global time stamps in the \cn{gt} and \cn{pred} cells matched, as:

{\small
  \begin{mathpar}
   \inferrule[Grant]{}
       {\qcell{t}{\cn{gt}}\pcell{\cn{false}}{t}{\cn{pred}} \longrightarrow \qcell{t}{\cn{gt}}\pcell{\cn{true}}{t}{\cn{pred}}}
\end{mathpar}
}

In QAM, there are mainly two groups of rules, finishing the tasks of transmitting a message from a starting node to its destination via many middle routers and delivering the message in the destination.
Each group can be divided into a consecutive three kinds of rule applications: 1) we finish the major functionality of the task and move necessary information to the \cn{comm} and \cn{pred} cells for granting; 2) one or more granting rules are applied on the system to check if the task is finished correctly; and 3) we clean up the task information from the \cn{comm} and \cn{pred} cells and update the global clock to finalize the task execution.

In transmitting a message from a starting node to its destination,
we have the rule applications \rulelab{GC} and \rulelab{FC} for step (1) and (3) above, with some possible granting rules, such as rule \rulelab{Grant} above.
Rule \rulelab{GC} defines the message transmission behavior that might be an intermediate router node, with the consumption of a qubit in each of the parties and a slot time change happening in the global clock.
The $...$ operation in cell $a$ refers to that there might 
be other paralleled processes associated in the process cell, but we disregard them.
In rule \rulelab{GC}, it also moves the necessary information, such as the sender and receiver channel names as well as the message, in the \cn{comm} cell with the update of the time stamp $t'$ in the \cn{pred} cell to be the current global time,
indicating that at the current point, the system conveys a message from a node to another.
After an intermediate granting step, rule \rulelab{FC} is applied on a configuration to empty the \cn{comm} cell, initialize the \cn{pred} cell by putting an \cn{false} value, and increment the global clock.
Additionally, rule \rulelab{CT} generates a time stamp for a message that is about to send out.

{\footnotesize
\[
\begin{array}{ll}
\longrightarrow\;\;
\Cella{\qsend{1}{c}{m}^0}{10}{\cn{Cat}}\Cella{0}{10}{r_1}
\Cella{\qrev{c}{x}.0}{10}{\cn{Dan}} 
\qcell{\{(\cn{Cat},r_1,0.5), (r_1,\cn{Dan},0.5)\}}{\cn{rel}}
\qcell{\emptyset}{\cn{comm}}
\pcell{\texttt{false}}{0}{\cn{pred}}
\qcell{0}{\cn{gt}}
&
(\rulelab{CT})
\\[0.2em]
\longrightarrow\;\;
\Cella{0}{9}{\cn{Cat}}\Cella{A}{9}{r_1}
\Cella{\qrev{c}{x}.0}{10}{\cn{Dan}} 
\qcell{R}{\cn{rel}}
\qcell{(\cn{Cat},r_1,A)}{\cn{comm}}
\pcell{\texttt{false}}{0}{\cn{pred}}
\qcell{0}{\cn{gt}}
&
(\rulelab{GC})
\\[0.2em]
\longrightarrow\;\;
\Cella{0}{9}{\cn{Cat}}\Cella{A}{9}{r_1}
\Cella{\qrev{c}{x}.0}{10}{\cn{Dan}} 
\qcell{R}{\cn{rel}}
\qcell{(\cn{Cat},r_1,A)}{\cn{comm}}
\pcell{\texttt{true}}{0}{\cn{pred}}
\qcell{0}{\cn{gt}}
&
(\rulelab{Grant})
\\[0.2em]
\longrightarrow\;\;
\Cella{0}{9}{\cn{Cat}}\Cella{A}{9}{r_1}
\Cella{\qrev{c}{x}.0}{10}{\cn{Dan}} 
\qcell{R}{\cn{rel}}
\qcell{\emptyset}{\cn{comm}}
\pcell{\texttt{false}}{0}{\cn{pred}}
\qcell{1}{\cn{gt}}
&
(\rulelab{FC})
\\[0.2em]
\longrightarrow\;\;
\Cella{0}{9}{\cn{Cat}}\Cella{0}{8}{r_1}
\Cella{\parl{A'}{\qrev{c}{x}.0}}{9}{\cn{Dan}} 
\qcell{R}{\cn{rel}}
\qcell{(r_1,\cn{Dan},A')}{\cn{comm}}
\pcell{\texttt{false}}{1}{\cn{pred}}
\qcell{1}{\cn{gt}}
&
(\rulelab{GC})
\end{array}
\]
}
{\footnotesize
\begin{center}
$R\triangleq\{(\cn{Cat},r_1,0.5), (r_1,\cn{Dan},0.5)\}
\qquad
A\triangleq\qsend{0.5}{c}{m}^0
\qquad
A'\triangleq\qsend{0.25}{c}{m}^0$
\end{center}
}

The above transitions evolves the configuration in \Cref{sec:qamsyntax}. We first apply rule \rulelab{CT} to attach the current time $0$ to the message to send as $\qsend{1}{c}{m}^0$; then, apply rule \rulelab{GC}. The message $\qsend{1}{c}{m}^0$ in cell \cn{Cat} is transmitted to the $r_1$ cell with the new probability $0.5$, meaning that only $50\%$ change this communication is guaranteed.
This can happen because there is a defined relation tuple $(\cn{Cat},r_1,0.5)$ in the relation cell $\cn{rel}$. During the process, the qubit resources in $\cn{Cat}$ and $r_1$ both reduce one, and the global clock is updated. 
It is worth noting that 
we associate the parallel process $\cn{|}$ with identity, associativity, and commutativity equational rules, 
such that $\parl{P}{0}=P$, $\parl{(\parl{P}{Q})}{Q'}=\parl{P}{(\parl{Q}{Q'})}$, and $\parl{P}{Q}=\parl{Q}{P}$.
In the above case, $\qsend{0.5}{c}{m}^0$ can be viewed as $\parl{\qsend{0.5}{c}{m}^0}{0}$.
We then apply rule \rulelab{Grant} to put a \cn{true} value in the \cn{pred} cell and the \rulelab{FC} rule application cleans up the cells, so the transition of the message from node $r_1$ to \cn{Dan} can happen through the last \rulelab{GC} rule application.

{\footnotesize
\[
\begin{array}{lll}
&\Cella{0}{9}{\cn{Cat}}\Cella{0}{8}{r_1}
\Cella{\parl{A'}{\qrev{c}{x}.0}}{9}{\cn{Dan}} 
\qcell{R}{\cn{rel}}
\qcell{\emptyset}{\cn{comm}}
\pcell{\texttt{false}}{1}{\cn{pred}}
\qcell{2}{\cn{gt}}
\\[0.2em]
\longrightarrow
&
\Cella{0}{9}{\cn{Cat}}\Cella{0}{8}{r_1}
\Cella{0}{9}{\cn{Dan}} 
\qcell{R}{\cn{rel}}
\qcell{A'}{\cn{comm}}
\pcell{\texttt{false}}{2}{\cn{pred}}
\qcell{2}{\cn{gt}}
&
(\rulelab{PC})
\\[0.2em]
\longrightarrow
&
\Cella{0}{9}{\cn{Cat}}\Cella{0}{8}{r_1}
\Cella{0}{9}{\cn{Dan}} 
\qcell{R}{\cn{rel}}
\qcell{A'}{\cn{comm}}
\pcell{\texttt{true}}{2}{\cn{pred}}
\qcell{2}{\cn{gt}}
&
(\rulelab{Grant})
\\[0.2em]
\xrightarrow{0.25.c,m}
&
\Cella{0}{9}{\cn{Cat}}\Cella{0}{8}{r_1}
\Cella{0}{9}{\cn{Dan}} 
\qcell{R}{\cn{rel}}
\qcell{\emptyset}{\cn{comm}}
\pcell{\texttt{false}}{2}{\cn{pred}}
\qcell{3}{\cn{gt}}
&
(\rulelab{Com})
\end{array}
\]
}
{\footnotesize
\begin{center}
$R\triangleq\{(\cn{Cat},r_1,0.5), (r_1,\cn{Dan},0.5)\}
\qquad
A'\triangleq\qsend{0.25}{c}{m}^0$
\end{center}
}

For delivering a message in the destination,
rules \rulelab{PC} and \rulelab{Com} define the steps except the intermediate granting step.
Rule \rulelab{PC} is applicable if the sending operation is already in its destination cell, and being paralleled with a same-channel message receipt process $\qrev{c}{x}.P$. After applying the rule, the message is processed as $P[m/x]$, and we also move the message $\qsend{p}{c}{m}^t$ to the communication cell $\cn{comm}$ with the current time stamp $t'$, so further user-defined rules can be applied on the message.
The above example first provides an \rulelab{PC} rule application showing the exactly procedure above, i.e, the message is moved to the \cn{comm} cell with the current global time $2$.
After the granting step,
a \rulelab{Com} rule application processes the delivered message $\qsend{p}{c}{m}^t$ by removing it from the $\cn{comm}$ cell if the predicate cell $\cn{pred}$ is $\texttt{true}$. The processing refers to making the message a transition label $p.c.m$ appearing in the transition,
so the message delivery is publicized. Other than applying rule \rulelab{Com}, all other rule applications are considered to be internal communications. 
In the above example, the last \rulelab{Com} rule application removes the message from the cell with a publicized transition label $0.25.c,m$. 

Rules \rulelab{MT} and \rulelab{NT} define the behaviors of the replication $\comp{P}$.
The former suggests that a process can be replicated as many as we want, while the latter one refers to that a replication process can non-deterministically terminates.
In quantum communication, message deliveries are always associated with a success rate,
which indicates that users might be interested in repeatedly sending a message until some success rate assurance is guaranteed.
We will see an example success rate guarantee mechanism by using the replication operations with additional user-defined rules in \Cref{sec:add-prop}.

\noindent\textbf{Extension of Rules.}
The rule set $\overline{\rules}$ is extendable by user-defined rules, 
such that if a user declares the statement $\decl{\rules_1}{...\; \decl{\rules_n}{C_0}}$
in the QAM system $(\Ms,\Ls,\Ts,C,\overline{\rules})$, $C=C_0$ and $\overline{\rules}=\{\rules_1,...\rules_n,\rulelab{CT},\rulelab{GC},\rulelab{MT},\rulelab{NT}, \rulelab{PC},\rulelab{Com}\}$.

Permitting user-defined rules allows the definitions of different protocol guarantees in QAM,
which are useful in evaluating different performance merits of different protocols,
while rules in \Cref{fig:q-pi-semantics} establish the semantic basis of quantum network communication.
As the \rulelab{Grant} rule example above, the guarantees are established by evaluating different predicates in the \cn{pred} cell,
which we name granting steps in handling the tasks of conveying and delivering messages.
Here, we discuss how to guarantee the on-time message delivery in the QPass/QCast protocols \cite{10.1145/3387514.3405853},
which is a property that people tried to guarantee in these these protocols, but was not established formally \cite{10.1145/3387514.3405853}.

A key observation in quantum network is that qubit messages have short liftime,
so the above quantum protocols developed complicated algorithms to guarantee the high delivery rates for quantum messages.
If a message is not delivered in time, its quantum state is decohered and its information loses.

{\footnotesize
\[
\begin{array}{l}
\decl{\pcell{\qsend{p}{c}{m}^t}{t'}{\cn{comm}}\qcell{b}{\cn{pred}}\qcell{f}{\cn{tp}} 
     \longrightarrow \pcell{\qsend{p}{c}{m}^t}{t'}{\cn{comm}}\qcell{f(t,t')}{\cn{pred}}\qcell{f}{\cn{tp}}}{}
\\[0.2em]
\qquad
\Cella{\qsend{1}{c}{m}^0}{10}{\cn{Cat}}\Cella{0}{10}{r_1}
\Cella{\qrev{c}{x}.0}{10}{\cn{Dan}} 
\qcell{R}{\cn{rel}}
\pcell{\emptyset}{0}{\cn{comm}}
\qcell{\texttt{false}}{\cn{pred}}
\qcell{0}{\cn{gt}}
\qcell{\lambda\;(t,t')\,.\,t'-t<5}{\cn{tp}}
\end{array}
\]
}
{\footnotesize
\begin{center}
$R\triangleq\{(\cn{Cat},r_1,0.5), (r_1,\cn{Dan},0.5)\}$
\end{center}
}

To guarantee the property, we add a new cell $\cn{tp}$ and declare a new rule above (let's call it rule \rulelab{TP}).
Every time when an action is in the \cn{comm} cell, which we know rule \rulelab{PC} is applied prior to the state, we apply the function $f$ in the new content cell \cn{tp} to the two time stamps $t$ and $t'$, and check if the message starting time $t$ and its delivery time $t'$ is within a threshold.
In the configuration setting, function $f$ is defined as a lambda abstraction $\lambda\;(t,t')\,.\,t'-t<5$ and it takes two time stamps and checks if $t'-t$ is less than the threshold number $5$. In addition, we can rewrite the \rulelab{Grant} rule below to only grant transitions for conveying messages; in such case, the \cn{comm} cell contains a tuple $(c_1,c_2,A)$.  

{\small
  \begin{mathpar}
   \inferrule[Grant]{}
       {\qcell{t}{\cn{gt}}\qcell{(c_1,c_2,A)}{\cn{comm}}\pcell{\cn{false}}{t}{\cn{pred}} \longrightarrow \qcell{t}{\cn{gt}}\qcell{(c_1,c_2,A)}{\cn{comm}}\pcell{\cn{true}}{t}{\cn{pred}}}
\end{mathpar}
}

After we declare rule \rulelab{TP} and the new configuration, we essentially create a new QAM system $(\Ms,\Ls,\Ts,C',\overline{\rules'})$,
where $C'$ is the new declared configuration and the rule set $\overline{\rules'}$ contains the six rules in \Cref{fig:q-pi-semantics}, rule \rulelab{Grant} above, as well as rule \rulelab{TP}.
In the execution example above, after applying rule \rulelab{PC}, the \cn{comm} cell becomes $\pcell{\qsend{0.25}{c}{m}^0}{2}{\cn{comm}}$ with the starting time stamp $0$ and the delivery time stamp $2$.
By applying rule \rulelab{TP}, the expression $2-0 < 5$ results in \texttt{true}, which grants the message transition for the following \rulelab{Com} rule application. 
However, if we set the threshold to be $2$, such as having the function definition $\lambda\;(t,t')\,.\,t'-t<2$ in the \cn{tp} cell of the above configuration, the expression $2-0 < 2$ results in \texttt{false}, which means that the following \rulelab{Com} rule is not applicable.
The system might be stuck because there is no rule to empty the \cn{comm} cell, which can be classified as a failure due to the violation of the delivery time guarantee. 












\section{Behavioral Refinement and Equivalence} \label{sec:refinement}

Here, we define QAM trace refinement relations, where two QAM systems are equivalent if the same sequences of actions can be performed
from their respective initial states.
Essentially, a QAM system $(\Ms,\Ls,\Ts,C,\overline{\rules})$
can be viewed as a tuple of four fixed sets $\Cs=(\Ms,\Ls,\Ts,\overline{\rules})$
and a configuration $C$ representing the initial program state.
Each rule in $\overline{\rules}$ is a label transition relation $C_1 \xrightarrow{\alpha} C_2$,
where $C_1$ and $C_2$ are configurations representing the states
before and after the rule application, and $\alpha$ is either empty or $p.c.m$.

Based on the transition step definition, we define QAM traces as an inductive set below. Given a QAM system tuple $\Cs=(\Ms,\Ls,\Ts,\overline{\rules})$, a ground term configuration $C$ is \textit{stuck}, if and only if no rules in $\overline{\rules}$ can be applied on $C$ for a transition step. We also refer to $C \longrightarrow^* C'$ as configuration $C$ transitioning to $C'$, based on the rules in $\Cs$, via several internal steps (steps having no labels).

\begin{definition}\label{def:traces}\rm[QAM Traces]
Given a QAM system tuple $\Cs=(\Ms,\Ls,\Ts,\overline{\rules})$, traces of a configuration $C$ on the tuple is defined as an inductive set $\Tts(\Cs,C)$ of label sequences as:

\begin{itemize}
\item If $C \longrightarrow^* C'$ and $C'$ is stuck, $\Tts(\Cs,C)=\emptyset$.
\item If $C \longrightarrow^* C' \xrightarrow{p.c.m} C''$ and $C'$ is stuck, $\Tts(\Cs,C)=(c.m)::\Tts(\Cs,C'')$, i.e., $(c.m)::\Tts(\Cs,C'')$ is a new set that for all sequence $\xi \in \Tts(\Cs,C'')$, $(c.m)::\xi \in (c.m)::\Tts(\Cs,C'')$ and $::$ is a sequence concatenation operation.
\end{itemize}
\end{definition}

With the trace definition above, we can then define a trace refinement relation between two different QAM systems with tuples $\Cs=(\Ms,\Ls,\Ts,\overline{\rules})$ and $\Cs=(\Ms',\Ls',\Ts',\overline{\rules'})$, as well as initial configurations $C$ and $C'$.

\begin{definition}\label{def:traceeq}\rm[QAM Trace Refinement]
Given two QAM system tuples $\Cs=(\Ms,\Ls,\Ts,\overline{\rules})$ and $\Cs=(\Ms',\Ls',\Ts',\overline{\rules'})$, and initial configurations $C$ and $C'$,  we say that $(\Cs,C)$ trace refines $(\Cs',C')$, written as $(\Cs,C) \sqsubseteq (\Cs',C')$, iff $\Tts(\Cs,C)\subseteq \Tts(\Cs',C')$.

\end{definition}

Sometimes, in dealing with quantum protocol evaluation, we want to know if every step of transition in a QAM system has higher success rate than the other system. In such case, we first define probability traces in QAM below.

\begin{definition}\label{def:ptraces}\rm[QAM Probability Traces]
Given a QAM system tuple $\Cs=(\Ms,\Ls,\Ts,\overline{\rules})$, traces of a configuration $C$ on the tuple is defined as an inductive set $\Pts(\Cs,C)$ of label sequences as:

\begin{itemize}
\item If $C \longrightarrow^* C'$ and $C'$ is stuck, $\Pts(\Cs,C)=\emptyset$.
\item If $C \longrightarrow^* C' \xrightarrow{p.c.m} C''$ and $C'$ is stuck, $\Pts(\Cs,C)=(c.m)^p::\Pts(\Cs,C'')$, i.e., $(c.m)^p::\Pts(\Cs,C'')$ is a new set that for all sequence $\xi \in \Pts(\Cs,C'')$, $(c.m)^p::\xi \in (c.m)^p::\Pts(\Cs,C'')$ and $::$ is a sequence concatenation operation.
\end{itemize}
\end{definition}

We can then define the probability subset relation between two trace sets and the probability trace refinement below.

\begin{definition}\label{def:ptracesub}\rm[QAM Probability Trace Subset]
Given two probability trace sets $S$ and $S'$, $S \subseteq^p S'$ iff, for all $\xi \in S$, there is $\xi' \in S'$, such that for every $i$-th position in $\xi$ and $\xi'$, $\xi[i]=(c.m)^p$ and $\xi'[i]=(c.m)^{p'}$ and $p \le p'$.

\end{definition}

\begin{definition}\label{def:ptracerefine}\rm[QAM Probability Trace Refinment]
Given two QAM system tuples $\Cs=(\Ms,\Ls,\Ts,\overline{\rules})$ and $\Cs=(\Ms',\Ls',\Ts',\overline{\rules'})$, and initial configurations $C$ and $C'$,  we say that $(\Cs,C)$ probabilistically trace refines $(\Cs',C')$, written as $(\Cs,C) \sqsubseteq^p (\Cs',C')$, iff $\Pts(\Cs,C)\subseteq^p \Pts(\Cs',C')$.

\end{definition}






\section{Case Study: QPass and QCast Protocol}\label{sec:case-study}

In the end of \Cref{sec:qamsemantics}, an example rule extension in QAM is mentioned to guarantee every message delivered is within a time limit. In this section, we see how we can utilize the rule extension in QAM to capture the behaviors of real-world quantum network protocols, such as QPass and QCast. In addition, an additional rule set is defined to show how we can utlize QAM to define additional properties that many quantum network protocols were intended but failed to do. The two properties are to guarantee that every message must be delivered in a threshold probability, and different nodes can periodically generate new qubits.

\ignore{
In this section, we show how to use the quantum abstract machine to model two quantum network protocols: QPass and QCast. These two protocols are developed by Shi and Qian \textit{[cite]}. Since a large-scale quantum network has many devices, the goal of those protocols is to solve the entanglement routing problem and build long distance entanglement for multiple source-destination pairs concurrently. These protocols have four phases: at the first phase, all nodes of the network are informed about the source-destination node pairs. Then these s-d pairs are inputted to the routing algorithm at phase two. The algorithm will output paths between source and destination. (Phase two is also the phase that our abstract machine tries to simulate.) At the third phase, all nodes exchanges their link states to their neighbors through classical channel. The entanglement swap is performed at the fourth phase so the long distance entanglement is eventuality created for quantum teleportation. QPass and QCast protocols are similar in general and the differences between them will be discussed in the following sections. }


\subsection{Defining Qpass and QCast Protocols}

\begin{figure}[t]
{\footnotesize
  \begin{mathpar}

\inferrule[QPassConfig]{}{
\overline{\varphi}
\qcell{R}{\cn{rel}}
\qcell{0}{\cn{gt}}
\qcell{\emptyset}{\cn{comm}}
\pcell{\cn{false}}{0}{\cn{pred}}
\qcell{f}{\cn{tp}}}

\mprset{flushleft}
  \inferrule[SP]{}
      {\qcell{R}{\cn{rel}}\qcell{t}{\cn{gt}}\qcell{(c_1,c_2,A(c))}{\cn{comm}}\pcell{b}{t}{\cn{pred}}
        \longrightarrow 
            \qcell{R}{\cn{rel}}\qcell{t}{\cn{gt}}\qcell{(c_1,c_2,A(c))}{\cn{comm}}\pcell{\exists p'\,.\,(c_1,c_2,p')\in \cn{sp}(R,c_1,c)}{t}{\cn{pred}}}

  \inferrule[MP]{}
      {\pcell{\_}{n}{c_1}\pcell{\_}{m}{c_2}
        \qcell{R}{\cn{rel}}\qcell{t}{\cn{gt}}\qcell{(c_1,c_2,A(c))}{\cn{comm}}\pcell{b}{t}{\cn{pred}}
    \\\qquad\longrightarrow 
      \pcell{\_}{n}{c_1}\pcell{\_}{m}{c_2}
  \qcell{\cn{up}(R,c_1,n,c_2,m)}{\cn{rel}}\qcell{t}{\cn{gt}}\qcell{(c_1,c_2,A(c))}{\cn{comm}}
      \pcell{\exists p'\,.\,(c_1,c_2,p')\in \cn{mp}(R,c_1,c)}{t}{\cn{pred}}}
  \end{mathpar}
}
{\footnotesize
\begin{center}
$A(c)\triangleq\qsend{p}{c}{m}^{t'}
\qquad
f\triangleq\lambda\;(t,t')\,.\,t'-t<5
$
\end{center}
}
\caption{The additional rules and configuration for QPass and QCast, which have the same initial configuration, and \rulelab{SP} is for QPass, while \rulelab{MP} is for QCast.}
  \label{fig:qpass-rule}
\end{figure}

We first discuss the QPass protocol definition in QAM, and then the QCast one.

\noindent\textbf{The QPass Protocol.}
The main observations of the QPass protocol are two: 1) dynamically computing the probability value of each relation pair in the \cn{rel} cell is hard; and 2) it is more efficient to rely on the statically commutable shortest path as a fixed way of conveying messages between two nodes. The key feature for modeling QPass is to test if a message conveying path is the shortest path. We first model the initial configuration of the QPass protocol as the \rulelab{QPassConfig} configuration in \Cref{fig:qpass-rule}, which is the same configuration at the end of \Cref{sec:qamsemantics}. In defining QPass, we also assume that the process cell name and the message channel name are the same. For example in \Cref{fig:q-pi-example}, if \cn{Cat} sends a message $\qsend{p}{c}{m}^{t'}$ to \cn{Dan}, $c$ must be \cn{Dan}.

The rule set definition for the QPass protocol includes the \rulelab{TP} rule. Now, we add a new rule \rulelab{SP},
so it contains the six rules in \Cref{fig:q-pi-semantics}, rule \rulelab{TP} above, and rule \rulelab{SP}.
Rule \rulelab{SP} grants a message conveying step. In each step, when a message $\qsend{p}{c}{m}^{t'}$ is conveyed from $c_1$ to $c_2$,
we define a predicate $\exists p'\,.\,(c_1,c_2,p')\in \cn{sp}(R,c_1,c)$ in cell \cn{pred} to check if pair $(c_1,c_2)$ is a part of the shortest path from $c_1$ to $c$. The function $\cn{sp}(R,c_1,c)$ finds the shortest path from $c_1$ to $c$ based on the relation set $R$.
If the predicate is satisfied, the message conveying step is granted, we then apply rule \cn{FC} to clean up everything and prepare for the next transition tasks.

{\footnotesize
\begin{center}
\[
\begin{array}{ll}
\Cella{0}{1}{\cn{Cat}}
\Cella{0}{0}{r_1}
\Cella{0}{1}{\cn{Dan}}
\Cella{A}{2}{\cn{Ann}}
\Cella{0}{2}{r_2}
\Cella{0}{2}{r_3}
\Cella{0}{2}{r_4}
\Cella{\qrev{\cn{Bob}}{x}.0}{2}{\cn{Bob}}
\qcell{R}{\cn{rel}}
\qcell{\emptyset}{\cn{comm}}
\pcell{b}{2}{\cn{pred}}
\qcell{3}{\cn{gt}}
\\[0.2em]
\longrightarrow
\;\;
\Cella{0}{1}{\cn{Cat}}
\Cella{0}{0}{r_1}
\Cella{0}{1}{\cn{Dan}}
\Cella{0}{1}{\cn{Ann}}
\Cella{A}{1}{r_2}
\Cella{0}{2}{r_3}
\Cella{0}{2}{r_4}
\Cella{\qrev{\cn{Bob}}{x}.0}{2}{\cn{Bob}}
\qcell{R}{\cn{rel}}
\qcell{(\cn{Ann},r_2,A)}{\cn{comm}}
\pcell{b}{3}{\cn{pred}}
\qcell{3}{\cn{gt}}
&
(\rulelab{GC})
\\[0.2em]
\longrightarrow
\;\;
\Cella{0}{1}{\cn{Cat}}
\Cella{0}{0}{r_1}
\Cella{0}{1}{\cn{Dan}}
\Cella{0}{1}{\cn{Ann}}
\Cella{A}{1}{r_2}
\Cella{0}{2}{r_3}
\Cella{0}{2}{r_4}
\Cella{\qrev{\cn{Bob}}{x}.0}{2}{\cn{Bob}}
\qcell{R}{\cn{rel}}
\qcell{(\cn{Ann},r_2,A)}{\cn{comm}}
\pcell{\cn{false}}{3}{\cn{pred}}
\qcell{3}{\cn{gt}}
&
(\rulelab{SP})
\end{array}
\]
\end{center}
}
{\footnotesize
\begin{center}
$A\triangleq\qsend{1}{\cn{Bob}}{m}^3
$
\end{center}
}

The above process cells are a part of the complete configuration that reflects the connectivity in \Cref{fig:q-pi-example}, by omitting cell \cn{tp}.
We assume that the system has a limited qubit resources, so every cell has initially $2$ qubits,
and a message from \cn{Cat} is delivered to the \cn{Dan} cell. In addition, we also assume that a \rulelab{CT} rule application is applied to attach the time stamp $3$ to the message $\qsend{1}{\cn{Bob}}{m}$. 
At this point, cell $r_1$ has $0$ qubit resource, and a message at the \cn{Ann} cell is waiting to convey to either $r_1$ or $r_2$,
but we cannot apply rule \rulelab{GC} to convey the message to $r_1$, since it has no enough resource; therefore, we apply rule \rulelab{GC} to convey to message to cell $r_2$.
Since $r_2$ is not part of the shortest path from $\cn{Ann}$ to $\cn{Bob}$, the application of rule \cn{SP} disqualifies the transition by marking the \cn{pred} cell to be $\cn{false}$.
After the \rulelab{SP} rule application, the configuration is stuck since there is no other possible rule that can be applied to it; thus, the system enters an error state.

\noindent\textbf{The QCast Protocol.}
From the above example, we can see that the QPass protocol can be unreasonable sometimes.
Since all message sending paths are statically computed as the shortest paths,
it is possible that other possible paths such as the message conveying from \cn{Ann} to $r_2$ above are omitted in the execution.
QCast modifies this drawback by making estimations of different probability values for pairs in the \cn{rel} cell
and select a path with the highest probability estimation value, instead of selecting the shortest path in QPass.
Additionally, we update predicted probabilities dynamically to reflect the probability reductions causing by qubit resource
consumption in different nodes.

In modeling the QCast protocol, we utilize the same initial configuration as QPass in \Cref{fig:qpass-rule}.
Instead of having rule \rulelab{SP}, we design the \rulelab{MP} rule for QCast.
There are two key differences. First, we now check if the pair $(c_1,c_2)$ is in the maximum probability path from $c_1$ to $c$, computed by the function $\cn{mp}(R,c_1,c)$. Instead of checking if the pair is in a shortest path from $c_1$ to $c$ in QPass, we now find the path that can deliver a message in a maximum chance by firstly computing the probability estimations for all different paths from $c_1$ to $c$ and then selecting the one having the maximum value. Second, we proportionally reduce the probability estimations that are associated with the $c_1$ and $c_2$ nodes. The qubit resources in nodes $c_1$ and $c_2$ are reduced, so paths through these two nodes are have a less likelihood to successfully convey messages, compared to the case when qubit resources are enough.
For simplicity, for every pair $(c_l,c_r,p)$, if $c_l$ and $c_r$ are one of $c_1$ (or $c_2$), whose qubit resource is $n$ (or $m$), we multiply $p$ with $\frac{n}{n+1}$ (or$ \frac{m}{m+1}$). 

{\footnotesize
\begin{center}
\[
\begin{array}{ll}
\Cella{0}{1}{\cn{Cat}}
\Cella{0}{0}{r_1}
\Cella{0}{1}{\cn{Dan}}
\Cella{A}{2}{\cn{Ann}}
\Cella{0}{2}{r_2}
\Cella{0}{2}{r_3}
\Cella{0}{2}{r_4}
\Cella{\qrev{\cn{Bob}}{x}.0}{2}{\cn{Bob}}
\qcell{R'}{\cn{rel}}
\qcell{\emptyset}{\cn{comm}}
\pcell{b}{2}{\cn{pred}}
\qcell{3}{\cn{gt}}
\\[0.2em]
\longrightarrow
\;\;
\Cella{0}{1}{\cn{Cat}}
\Cella{0}{0}{r_1}
\Cella{0}{1}{\cn{Dan}}
\Cella{0}{1}{\cn{Ann}}
\Cella{A}{1}{r_2}
\Cella{0}{2}{r_3}
\Cella{0}{2}{r_4}
\Cella{\qrev{\cn{Bob}}{x}.0}{2}{\cn{Bob}}
\qcell{R'}{\cn{rel}}
\qcell{(\cn{Ann},r_2,A)}{\cn{comm}}
\pcell{b}{3}{\cn{pred}}
\qcell{3}{\cn{gt}}
&
(\rulelab{GC})
\\[0.2em]
\longrightarrow
\;\;
\Cella{0}{1}{\cn{Cat}}
\Cella{0}{0}{r_1}
\Cella{0}{1}{\cn{Dan}}
\Cella{0}{1}{\cn{Ann}}
\Cella{A}{1}{r_2}
\Cella{0}{2}{r_3}
\Cella{0}{2}{r_4}
\Cella{\qrev{\cn{Bob}}{x}.0}{2}{\cn{Bob}}
\qcell{R''}{\cn{rel}}
\qcell{(\cn{Ann},r_2,A)}{\cn{comm}}
\pcell{\cn{true}}{3}{\cn{pred}}
\qcell{3}{\cn{gt}}
&
(\rulelab{MP})
\end{array}
\]
\end{center}
}
{\footnotesize
\begin{center}
$A\triangleq\qsend{1}{\cn{Bob}}{m}^3$
\end{center}
}

The main purpose of the dynamic probability estimation update is to take into account the qubit resources as a factor in conveying messages. The above example is a rewrite of the one in the QPass protocol. The \cn{rel} cell content $R'$ is updated before the execution, as the probability estimations for some pairs are not $0.5$ anymore; especially, the connection between $\cn{Ann}$ and $r_1$ has $0$ probability as $(\cn{Ann},r_1,0)$, because cell $r_1$ has $0$ qubit resource. Thus, the \rulelab{MP} rule application at the end grants the transition by marking cell \cn{pred} to be \cn{true}, because the path from \cn{Ann} to \cn{Bob} via $r_2$ has a higher chance than the one via $r_1$, whose chance value is $0$. Additionally, we update the connectivity table $R'$ to $R''$, with the probability reductions on pairs related to nodes $\cn{Ann}$ and $r_2$.

Notice that the $\cn{up}$ function in rule \rulelab{MP} can be parameterized. If we parameterize the \cn{up} function to not update any relations in \cn{rel} cell, i.e., $\cn{up}(R,c_1,n,c_2,m)=R$, starting from the same configuration, we can show that the QPass system probabilistically trace refines the QCast system, which is stated as follows:

\begin{theorem}\label{def:traceeq}\rm[QAM Trace Refinement]
Given an initial configuration $C$ as \rulelab{QPassConfig} in \Cref{fig:qpass-rule}, with processes $\overline{\varphi}$, relation $R$, and \cn{tp} cell threshold function $f$, QPass $\sqsubseteq^p$ QCast.

\end{theorem}

\begin{figure}[t]
{\footnotesize
  \begin{mathpar}

\inferrule[HighConfig]{}{
\overline{\varphi}
\qcell{R}{\cn{rel}}
\qcell{0}{\cn{gt}}
\qcell{\emptyset}{\cn{comm}}
\pcell{\cn{false}}{0}{\cn{pred}}
\qcell{f}{\cn{tp}}
\pcell{g}{\beta}{\cn{at}}
}

\inferrule[QIConfig]{}{
\overline{\varphi}
\qcell{R}{\cn{rel}}
\qcell{0}{\cn{gt}}
\qcell{\emptyset}{\cn{comm}}
\pcell{\cn{false}}{0}{\cn{pred}}
\qcell{f}{\cn{tp}}
\pcell{\cn{false}}{\nu}{\cn{qi}}
}

\mprset{flushleft}
  \inferrule[HP]{}
      {\qcell{R}{\cn{rel}}\qcell{t}{\cn{gt}}\qcell{\qsend{p}{c}{m}^{t'}}{\cn{comm}}\pcell{b}{t}{\cn{pred}}\qcell{f}{\cn{tp}}\pcell{g}{\beta}{\cn{at}}
\\\qquad
        \longrightarrow 
            \qcell{R}{\cn{rel}}\qcell{t}{\cn{gt}}\qcell{\qsend{p}{c}{m}^{t'}}{\cn{comm}}\pcell{f(t,t')\wedge g(c.m)<\beta}{t}{\cn{pred}}
        \pcell{g[c.m\mapsto g(c.m)\oplus p]}{\beta}{\cn{at}}
      }

  \inferrule[QI1]{}
      {\pcell{P}{n}{c}\qcell{t}{\cn{gt}}\pcell{\cn{false}}{\nu}{\cn{qi}}
        \longrightarrow 
        \pcell{P}{n}{c}\qcell{t}{\cn{gt}}\pcell{\nu\cn{|}t}{\nu}{\cn{qi}}
      }

  \inferrule[QI2]{}
      {\pcell{P}{n+}{c}\pcell{\cn{true}}{\nu}{\cn{qi}}
        \longrightarrow 
        \pcell{P}{n+}{c}\pcell{\cn{false}}{\nu}{\cn{qi}}
      }

  \inferrule[TI]{}
      {\qcell{t}{\cn{gt}}
        \longrightarrow 
        \qcell{t+}{\cn{gt}} \\\cn{[owise]}
      }

  \end{mathpar}
}
{\footnotesize
\begin{center}
$A(c,t')\triangleq\qsend{p}{c}{m}^{t'}
\qquad
f\triangleq\lambda\;(t,t')\,.\,t'-t<5
\qquad
\cn{type}(g)\triangleq c.m \rightharpoonup p
$
\end{center}
}
\caption{The additional rules and configuration for high probability of message deliveries and qubit resource generation.}
  \label{fig:mes-rule}
\end{figure}

\subsection{Additional Quantum Network Properties}
\label{sec:add-prop}

There are many different properties that we want from a quantum network protocols.
Here, we introduce two of them that were discussed in the previous works but not captured by these works. 
The first property is to guarantee a high probability threshold of every message delivery,
while the second one is to introduce qubit creation. 

\noindent\textbf{Guarantee High Probability of Message Deliveries.}
Quantum message sending has a certain chance of failure, which depends on the transmission distance.
To guarantee the success rate, one simple solution is to repeatedly send the message until a threshold is reached.
We introduce a new initial configuration \rulelab{HighConfig} to model such behavior,
with a new cell \cn{at} having a content map $g$, storing the current success rates for different messages $c.m$, and a flag $\beta$, the success rate threshold every message needs to satisfy in $g$.
We then introduce the \rulelab{HP} rule based on the \rulelab{TP} rule in \Cref{sec:qamsemantics}
for granting a message delivery.
A message $c.m$ is valid to deliver if it satisfies the time threshold $f(t,t')$
and its current success rate in $g$ is less than the threshold $\beta$.
Once the threshold is reached, we do not need additional repetition of sending $c.m$, so that we only need to grant message delivery that has not yet satisfied the threshold.
Additionally, every success message delivery increments ($g[c.m\mapsto g(c.m)\oplus p]$) the current success rate for that message in the \cn{at} cell.

{\footnotesize
\[
\begin{array}{ll}
\Cella{\comp{(\qsend{1}{c}{m}})}{10}{\cn{Cat}}\Cella{0}{10}{r_1}
\Cella{\comp{(\qrev{c}{x}.0)}}{10}{\cn{Dan}} 
\qcell{R}{\cn{rel}}
\qcell{\emptyset}{\cn{comm}}
\qcell{0}{\cn{gt}}
\qcell{f}{\cn{tp}}
\pcell{g}{0.36}{\cn{at}}
\\[0.2em]
\longrightarrow\;\;
\Cella{\parl{\qsend{1}{c}{m}}{\comp{(\qsend{1}{c}{m}})}}{10}{\cn{Cat}}
\Cella{0}{10}{r_1}
\Cella{\comp{(\qrev{c}{x}.0)}}{10}{\cn{Dan}} 
\qcell{R}{\cn{rel}}
\qcell{\emptyset}{\cn{comm}}
\qcell{0}{\cn{gt}}
\qcell{f}{\cn{tp}}
\pcell{g}{0.36}{\cn{at}}
&
(\rulelab{MT})
\\[0.2em]
\longrightarrow\;\;
\Cella{\parl{\qsend{1}{c}{m}^0}{\comp{(\qsend{1}{c}{m}})}}{10}{\cn{Cat}}
\Cella{0}{10}{r_1}
\Cella{\comp{(\qrev{c}{x}.0)}}{10}{\cn{Dan}} 
\qcell{R}{\cn{rel}}
\qcell{\emptyset}{\cn{comm}}
\qcell{0}{\cn{gt}}
\qcell{f}{\cn{tp}}
\pcell{g}{0.36}{\cn{at}}
&
(\rulelab{CT})
\\[0.2em]
\longrightarrow\;\;
\Cella{\comp{(\qsend{1}{c}{m}})}{9}{\cn{Cat}}
\Cella{\qsend{0.5}{c}{m}^0}{9}{r_1}
\Cella{\comp{(\qrev{c}{x}.0)}}{10}{\cn{Dan}} 
\qcell{R}{\cn{rel}}
\qcell{A}{\cn{comm}}
\qcell{0}{\cn{gt}}
\qcell{f}{\cn{tp}}
\pcell{g}{0.36}{\cn{at}}
&(\rulelab{GC})
\\[0.2em]
\longrightarrow\;...\;\longrightarrow
\\[0.2em]
\Cella{\comp{(\qsend{1}{c}{m}})}{9}{\cn{Cat}}
\Cella{0}{8}{r_1}
\Cella{\parl{\qsend{0.25}{c}{m}^0}{\comp{(\qrev{c}{x}.0)}}}{9}{\cn{Dan}} 
\qcell{R}{\cn{rel}}
\qcell{\emptyset}{\cn{comm}}
\qcell{2}{\cn{gt}}
\qcell{f}{\cn{tp}}
\pcell{g}{0.36}{\cn{at}}
\\[0.2em]
\longrightarrow\;\;
\Cella{\comp{(\qsend{1}{c}{m}})}{9}{\cn{Cat}}
\Cella{0}{8}{r_1}
\Cella{\parl{\qsend{0.25}{c}{m}^0}{\parl{\qrev{c}{x}.0}{\comp{(\qrev{c}{x}.0)}}}}{9}{\cn{Dan}} 
\qcell{R}{\cn{rel}}
\qcell{\emptyset}{\cn{comm}}
\qcell{2}{\cn{gt}}
\qcell{f}{\cn{tp}}
\pcell{g}{0.36}{\cn{at}}
&
(\rulelab{MT})
\\[0.2em]
\longrightarrow\;\;
\Cella{\comp{(\qsend{1}{c}{m}})}{9}{\cn{Cat}}
\Cella{0}{8}{r_1}
\Cella{\comp{(\qrev{c}{x}.0)}}{9}{\cn{Dan}} 
\qcell{R}{\cn{rel}}
\qcell{A'}{\cn{comm}}
\qcell{2}{\cn{gt}}
\qcell{f}{\cn{tp}}
\pcell{g}{0.36}{\cn{at}}
&
(\rulelab{PC})
\end{array}
\]
}
{\footnotesize
\begin{center}
$R\triangleq\{(\cn{Cat},r_1,0.5), (r_1,\cn{Dan},0.5)\}
\qquad
A\triangleq (\cn{Cat},r_1,\qsend{0.5}{c}{m}^0)
\qquad
A'\triangleq \qsend{0.25}{c}{m}^0
\qquad
g=\lambda x\,.\,0
$
\end{center}
}

We create message repetition by the $\comp{P}$ process, 
as the example configuration transitions above, which omit the \cn{pred} cell.
Here, both the sender and receiver cells have repetition processes, 
$\comp{(\qsend{1}{c}{m}^0)}$ and $\comp{(\qrev{c}{x}.0)}$.
When it is necessary, the \cn{MT} rule application generates additional sending/receiving process to send/receive a message.
After a \cn{MT} rule application to generate a single message sending $\qsend{1}{c}{m}$, 
a \cn{CT} rule application is applied to attach time stamp $0$ to the message.
Then, the message is transmitted to the \cn{Dan} cell
and we use rule \cn{MT} to generate a message receipt $\qrev{c}{x}.0$;
thus, a \cn{PC} rule application is applied to move the message to the \cn{comm} cell, waiting for its granting.
The configuration below finalizes the above configuration with the \cn{pred} cell.

{\footnotesize
\[
\begin{array}{ll}
\Cella{\comp{(\qsend{1}{c}{m}})}{9}{\cn{Cat}}
\Cella{0}{8}{r_1}
\Cella{\comp{(\qrev{c}{x}.0)}}{9}{\cn{Dan}} 
\qcell{R}{\cn{rel}}
\qcell{A(0)}{\cn{comm}}
\pcell{\emptyset}{2}{\cn{pred}}
\qcell{2}{\cn{gt}}
\qcell{f}{\cn{tp}}
\pcell{g}{0.36}{\cn{at}}
\\[0.2em]
\longrightarrow\;\;
\Cella{\comp{(\qsend{1}{c}{m}})}{9}{\cn{Cat}}
\Cella{0}{8}{r_1}
\Cella{\comp{(\qrev{c}{x}.0)}}{9}{\cn{Dan}} 
\qcell{R}{\cn{rel}}
\qcell{\emptyset}{\cn{comm}}
\pcell{f(0,2)\wedge 0 < 0.36}{0}{\cn{pred}}
\qcell{2}{\cn{gt}}
\qcell{f}{\cn{tp}}
\pcell{g'}{0.36}{\cn{at}}
&
(\rulelab{HP})
\\[0.2em]
\xrightarrow{0.25.c.m}  \;\;
\Cella{\comp{(\qsend{1}{c}{m}})}{9}{\cn{Cat}}
\Cella{0}{8}{r_1}
\Cella{\comp{(\qrev{c}{x}.0)}}{9}{\cn{Dan}} 
\qcell{R}{\cn{rel}}
\qcell{\emptyset}{\cn{comm}}
\pcell{\cn{false}}{0}{\cn{pred}}
\qcell{3}{\cn{gt}}
\qcell{f}{\cn{tp}}
\pcell{g'}{0.36}{\cn{at}}
&
(\rulelab{Com})
\\[0.2em]
\longrightarrow\;...\;\longrightarrow
\\[0.2em]
\Cella{\comp{(\qsend{1}{c}{m}})}{8}{\cn{Cat}}
\Cella{0}{6}{r_1}
\Cella{\comp{(\qrev{c}{x}.0)}}{8}{\cn{Dan}} 
\qcell{R}{\cn{rel}}
\qcell{A(3)}{\cn{comm}}
\pcell{\emptyset}{2}{\cn{pred}}
\qcell{5}{\cn{gt}}
\qcell{f}{\cn{tp}}
\pcell{g'}{0.36}{\cn{at}}
\\[0.2em]
\longrightarrow\;\;
\Cella{\comp{(\qsend{1}{c}{m}})}{9}{\cn{Cat}}
\Cella{0}{8}{r_1}
\Cella{\comp{(\qrev{c}{x}.0)}}{9}{\cn{Dan}} 
\qcell{R}{\cn{rel}}
\qcell{\emptyset}{\cn{comm}}
\pcell{f(0,2)\wedge 0.25 < 0.36}{0}{\cn{pred}}
\qcell{5}{\cn{gt}}
\qcell{f}{\cn{tp}}
\pcell{g_1}{0.36}{\cn{at}}
&
(\rulelab{HP})
\\[0.2em]
\xrightarrow{0.25.c.m}  \;\;
\Cella{\comp{(\qsend{1}{c}{m}})}{8}{\cn{Cat}}
\Cella{0}{6}{r_1}
\Cella{\comp{(\qrev{c}{x}.0)}}{8}{\cn{Dan}} 
\qcell{R}{\cn{rel}}
\qcell{\emptyset}{\cn{comm}}
\pcell{\cn{false}}{0}{\cn{pred}}
\qcell{5}{\cn{gt}}
\qcell{f}{\cn{tp}}
\pcell{g_1}{0.36}{\cn{at}}
&
(\rulelab{Com})
\\[0.2em]
\longrightarrow  \;\;
\Cella{0}{8}{\cn{Cat}}
\Cella{0}{6}{r_1}
\Cella{\comp{(\qrev{c}{x}.0)}}{8}{\cn{Dan}} 
\qcell{R}{\cn{rel}}
\qcell{\emptyset}{\cn{comm}}
\pcell{\cn{false}}{0}{\cn{pred}}
\qcell{5}{\cn{gt}}
\qcell{f}{\cn{tp}}
\pcell{g_1}{0.36}{\cn{at}}
&
(\rulelab{NT})
\\[0.2em]
\longrightarrow  \;\;
\Cella{0}{8}{\cn{Cat}}
\Cella{0}{6}{r_1}
\Cella{0}{8}{\cn{Dan}} 
\qcell{R}{\cn{rel}}
\qcell{\emptyset}{\cn{comm}}
\pcell{\cn{false}}{0}{\cn{pred}}
\qcell{5}{\cn{gt}}
\qcell{f}{\cn{tp}}
\pcell{g_1}{0.36}{\cn{at}}
&
(\rulelab{NT})
\end{array}
\]
}
{\footnotesize
\begin{center}
$
\begin{array}{l}
R\triangleq\{(\cn{Cat},r_1,0.5), (r_1,\cn{Dan},0.5)\}
\qquad
A(t)\triangleq \qsend{0.25}{c}{m}^t
\\[0.2em]
g=\lambda x\,.\,0
\qquad
g'=g[c.m \mapsto 0.25]
\qquad
g_1=g[c.m \mapsto 0.39]
\end{array}
$
\end{center}
}

In the above transition, we apply rule \rulelab{HP} to grant the message sending operation $\qsend{0.25}{c}{m}^0$, since $f(0,2)$ is equivalent to $2-0<5$ and $g(c.m)=0$ is less than the threshold $0.36$ associated with the \cn{at} cell.
In the \cn{at} cell, we updates $g$ for $c.m$ to be $0.25$.
We assume that if messages have the same channel $c$ and message name $m$, they refer to the same message,
so repetitions of a message $c.m$ refer to the same one in the \cn{at} cell.
The procedure can be repeated to send more copies of message $c.m$ and then we can use \rulelab{HP} and \rulelab{Com} rule applications to grant the message delivery. In the above example, after two repeating $c.m$ message deliveries, the current success rate of $c.m$ is $0.39$ which is beyond the $0.36$ threshold. Any additional repetition causes the system to be stuck, so we apply two consecutive \rulelab{NT} rule applications to turn away the two repetition process operations. 

\noindent\textbf{Modeling Qubit Generation.}
An useful case for quantum network protocols is to generate new qubit resources for supporting message transmissions.
For example, in the QPass protocol example above, if we can generate qubits for node $r_1$, we can then transmit the message via node $r_1$ without the problem of qubit resource lacking.
It is always that the generation of a qubit takes multiple time slots compare to transmitting/sending messages.

In modeling the qubit generation mechanism, we include a new cell \cn{qi} in configuration \rulelab{QIConfig} in \Cref{fig:mes-rule},
which determines if the qubit resource in a process cell can increment.
The additional flag $\nu$ is an integer defining the time period for generating a qubit.
Rule \rulelab{QI1} increments a qubit in cell $c$ if $\nu$ divides the current global time $t$ ($\nu \cn{|} t$),
while rule \rulelab{QI2} cleans up the \cn{qi} cell for next possible qubit increment computation.
By including the new rule \rulelab{QI1}, it is possible for a QAM system to reach deadlock,
because a system can be stuck due to lack of a qubit in a process cell, but in the very next time, the qubit resource in the cell can increment by applying rule \rulelab{QI1}; however, there is no rule in the system for the global clock cell \cn{gt} to increment by itself.
To overcome this, we include rule \rulelab{TI} here to allow self-increment in the \cn{gt} cell, 
with the flag $\cn{owise}$ meaning that the rule can be applied only if there is no other possible rules to apply.







\section{Related Work}
\label{Related Work}

\noindent\textbf{Traditional Process Algebra.}
CSP \cite{CSPm,FDR2}, $\Pi$-calculus \cite{DBLP:conf/concur/Sangiori93}, Chemical Abstract Machine \cite{BERRY1992217},
the K framework \cite{rosu-serbanuta-2010-jlap}, modulo structural operational semantics \cite{MOSSES2004195}.

\noindent\textbf{Quantum Single-threaded Languages.}
SQIR \cite{VOQC}, Silq \cite{sliqlanguage}, QWhile in QHL \cite{10.1007/s00165-018-0465-3}, QWIRE \cite{Rand2018ReQWIRERA}, Quipper \cite{10.1145/2491956.2462177}, Q\# \cite{qsharp}.

\noindent\textbf{Quantum Multi-threaded Process Algebra.}
CQP \cite{9165801}. qCCS \cite{10.1145/1507244.1507249}, Using CCS to define a quantum process algebra \cite{10.1145/977091.977108},
bi-simulation for quantum processes \cite{10.1145/2400676.2400680}, CQP with ground bi-simulation \cite{10.1007/978-3-030-45237-7_2},
qCCS for cryptographic protocols \cite{https://doi.org/10.48550/arxiv.1507.05278},
a quantum circuit building framework based on Markov decision processes \cite{https://doi.org/10.48550/arxiv.2207.03403}.

\noindent\textbf{Quantum Network Protocols.}
Quantum teleportation is the first quantum network protocol \cite{PhysRevLett.70.1895},
quantum swaps \cite{fundamentallimits,aam9288,10.1145/3386367.3431293}, 
a network protocol increasing the reliability of QPass/QCast \cite{Pirker_2019},
QPass/QCast protocol \cite{10.1145/3387514.3405853},
a better data transmission protocol \cite{PhysRevResearch.4.043064},
a better protocol than QPass/QCast by including a graph analysis to find the best path \cite{https://doi.org/10.48550/arxiv.1907.11630},
physical level protocol to improve the success rates \cite{10.1145/3341302.3342070}.
A quantum network stack and protocols for reliable entanglement-based networks\cite{Pirker_2019}.
Fidelity-Guarantee Entanglement Routing in Quantum Networks\cite{10.48550/arxiv.2111.07764}.
Optimal Routing for Quantum Networks\cite{8068178}.  




\section{Conclusions and Directions for Future Work} \label{sec:conclusions}

We are proposing a quantum $\Pi$-calculus that can describe the quantum network protocols using long-distance entanglement. 


% \Cref{fig:q-pi-syntax} provides the syntax of the language. Every channel is a list of qubits,
% written as $\parl{\parl{q_0}{...}}{q_n}$. We also have a predicate $S$ determining the adjacency of two channel qubits, i.e., two qubits are adjacent with each other if they are close to each other in distance.
% $\qchan{c}{P}$ and $\qchana{c}{T}$ are new operations referring to that a channel is built on top of a list of qubits. Process $T$ describes the behaviors of routers which holds a finite set of qubits waiting for constructing channels.
% \Cref{fig:q-pi-semantics} provides the reduction rules.
% Rule \rulelab{SCon1} to \rulelab{SCon3} are congruence rules such that the channel building operations commute with other operations.
% Rule \rulelab{Com} sends a quantum message from the left process to the right via a channel held by process $T$, while rule \rulelab{Chan} builds a long distance channel via $n$ qubits. For these two rules to happen, we need to check if all the qubits are adjacent with each other and the lifetime, checked by predicate $\texttt{time}$, of each qubit is not expired. Rule \rulelab{Com} is a probabilistic rule, such that the function \texttt{rate} produces a probability that the arrow will exist. The probability depends on the channel length.
\bibliography{reference}

\end{document}



% {\small
%   \begin{mathpar}

%     \inferrule[Com1]{}{R_1, R_2\equiv R_2, R_1}

%     \inferrule[Com2]{}{\langle\parl{R_1}{R_2}\rangle\equiv\langle\parl{R_2}{R_1}\rangle}
         
%   \inferrule[Gen]{q \not\in R \\ Size(R)< max\_size}{R \longrightarrow (R \cup [q]) }
         
%     \inferrule[Entangle]{}{\qact{q_1}{R_1}, \qact{q_2}{R_2} \rightharpoonup \langle\parl{R_1}{R_2}\rangle}
    
%     \inferrule[Decoherence]{}{\langle\parl{R_1}{R_2}\rangle \rightharpoondown R_1, R_2 }

%   \inferrule[ChanGen1]{\qact{q_1}{R_1}, \qact{q_2_1}{\qact{q_2_2}{R_2}} \rightharpoonup \langle\parl{R_1}{\qact{q_2_2}{R_2}}\rangle \qquad\qquad \qact{q_2_2}{R_2}, \qact{q_3}{R_3} \rightharpoonup \langle\parl{R_2}{R_3}\rangle }
%          {\qact{q_1}{R_1}, \qact{q_2_1}{\qact{q_2_2}{R_2}}, \qact{q_3}{R_3} \longrightarrow \alchan{(\langle\parl{R_1}{R_3}\rangle)}{\chansol{R_2}}  }
         
%   \inferrule[ChanGen2]{\qact{q_3}{R_3}, \qact{q_4}{R_4} \rightharpoonup \langle\parl{R_3}{R_4}\rangle }
%      {\alchan{(\langle\parl{R_1}{\qact{q_3}{R_3}}\rangle)}{\chansol{R_2}}, \qact{q_4}{R_4} \longrightarrow \alchan{(\langle\parl{R_1}{R_4}\rangle)}{\chansol{R_2, R_3}}  }


%   \inferrule[Com]{}
%       { \parl{\qact{\qsend{c}{m}}{P}}{\qact{\qrev{c}{x}}{P}}
%       \xrightarrow{\texttt{prob}(c),c.m} \parl{P}{Q} }

%   \end{mathpar}
% }

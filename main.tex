%%%%%%%%%%%%%%%%%%%%%%%%%%%%%%%%%%%%%%%%%%%%%%%%%%%%%%%%%%%%%%%
%
% Welcome to Overleaf --- just edit your LaTeX on the left,
% and we'll compile it for you on the right. If you open the
% 'Share' menu, you can invite other users to edit at the same
% time. See www.overleaf.com/learn for more info. Enjoy!
%
%%%%%%%%%%%%%%%%%%%%%%%%%%%%%%%%%%%%%%%%%%%%%%%%%%%%%%%%%%%%%%%
\documentclass{article}
%\usepackage{semantic}
%\usepackage[margin=0.6in]{geometry}
%% Some recommended packages.
\usepackage{booktabs}   %% For formal tables:
                        %% http://ctan.org/pkg/booktabs
\usepackage{subcaption} %% For complex figures with subfigures/subcaptions
                        %% http://ctan.org/pkg/subcaption
\usepackage{bussproofs}
\usepackage[cal=boondoxo]{mathalfa}
\DeclareMathAlphabet{\mathpzc}{OT1}{pzc}{m}{it}
\usepackage{amsmath}
\newtheorem{theorem}{Theorem}[section]
\newtheorem{observation}[theorem]{Observation}
\usepackage{color}
\usepackage{xspace}
\input{macros}
\usepackage{bm}
\usepackage{amssymb}
\input{preamble}
\usepackage{pifont}% http://ctan.org/pkg/pifont

\newcommand{\cmark}{\text{\ding{51}}}
\newcommand{\xmark}{\text{\ding{55}}}

% \usepackage{forest}
% Using the geometry package with a small
% page size to create the article graphic

\title{Quantum Abstract Machine}  

\begin{document}
\maketitle

We are proposing a quantum $\Pi$-calculus that can describe the quantum network protocols using long-distance entanglement. 

\newcommand{\qsend}[3]{#1\texttt{.}#2\texttt{!<}#3\texttt{>}}
\newcommand{\qrev}[2]{#1\texttt{?(}#2\texttt{)}}
\newcommand{\qact}[2]{#1.#2}

\newcommand{\qchan}[3]{#1\texttt{[}#2\texttt{]}.#3}
\newcommand{\qchana}[3]{#1\texttt{[}#2\texttt{]}\,@\,#3}

\newcommand{\parp}[2]{#1\,\texttt{@}\,#2}

\newcommand{\parl}[2]{#1\,\texttt{|}\,#2}

\newcommand{\parll}[2]{#1\,\texttt{||}\,#2}

% \newcommand{\alchan}[2]{#1\,\triangleleft\,#2}

% \newcommand{\chansol}[1]{\texttt{\{|}#1\texttt{|\}}}

\newcommand{\mcomm}[2]{\texttt{[}#1\texttt{]}#2}

\newcommand{\Ss}{\mathbb{S}}
\newcommand{\Rs}{\mathbb{R}}
\newcommand{\Ps}{\mathbb{P}}
\newcommand{\Path}[1]{<#1> }
\newcommand{\Cfg}[3]{\langle #1\texttt{,}#2\texttt{,}#3 \rangle}
\newcommand{\Seta}[2]{#1\texttt{@}#2}
\newcommand{\PSet}[1]{\texttt{\{}#1\texttt{\}}}

\newcommand{\qcell}[1]{\langle #1 \rangle}
\newcommand{\Cella}[3]{\langle #1 \rangle^{#2}_{#3}}
\newcommand{\Cellb}[3]{\langle #1 \, \;...\rangle^{#2}_{#3}}

\begin{figure}[h]
{\small
  \[\begin{array}{llcl} 
      \texttt{Variables} & x,y \\
      \texttt{Probability} & p\\
      \texttt{Messages} & m\\
    \texttt{Channels} & c\\
      \texttt{Config} & \mathcal{L} & = & \Ss \qcell{\Rs}\\
      \texttt{Sets} & \Ss & ::= & (\langle A (\texttt{|} P )^?\rangle^{n}_{c})^* \\
      \texttt{Relations} & \Rs & ::= & [c \times c \times p] \\
      \texttt{Actions} & A & ::= & \emptyset \mid \qsend{p}{c}{m} \mid \qrev{c}{x} \\
    %   \text{Process} & P & ::= & \parp{A}{P} \\
      \text{Process} & P,Q & ::= & 0 \mid A.P \\
    %   \text{Adjacent Predicate} & S & \triangleq & A \times A \to \texttt{Bool} \\
    \end{array}
  \]
}
\caption{Quantum Pi Syntax}
  \label{fig:q-pi-syntax}
\end{figure}

\begin{figure}[h]
{\small
  \begin{mathpar}
   
   \inferrule[GenChan]{}
       {\Cellb{\qsend{p}{c}{m}}{i}{a}\Cellb{}{j}{b} \qcell{\{(a,b,p')\}\cup R})
        \longrightarrow (\Cellb{}{i-}{a}\Cellb{\qsend{(p\cdot p')}{c}{m}}{j-}{b},\{(a,b)\}\cup R)}


   \inferrule[GenQubit]{}
       {\Cellb{}{n}{a}\longrightarrow \Cellb{}{n+}{a}}
      
  \inferrule[Communication]{}
      { \qsend{p}{c}{m}\texttt{|} \qrev{c}{x}.P \xrightarrow{p.c.m}P[m/x]}
    


  \end{mathpar}
}
\caption{Quantum Pi Semantics}
  \label{fig:q-pi-semantics}
\end{figure}

\begin{figure}[h]
{\small
\centering
\begin{tikzpicture}[node distance={1cm}, thick, main/.style = {draw, circle}] 
\node[main] (1) {a}; 
\node[main] (2) [right of=1] {$r_1$};
\node[main] (3) [right of=2] {$r_4$};
\node[main] (4) [right of=3] {b};
\node[main] (5) [above right=0.5cm and 1.5cm of=1] {$r_2$};
\node[main] (6) [right of=5] {$r_3$};
\draw[-] (1) --  (2);
\draw[-] (2) --  (3);
\draw[-] (3) --  (4);
\draw[-] (1) --  (5);
\draw[-] (5) --  (6);
\draw[-] (6) --  (3);
\end{tikzpicture} 



  \begin{mathpar}
   \mathcal{L} =  \texttt{S} \qcell{\texttt{R}} \\
   = \Cella{\qsend{1}{c}{m}|0}{10}{a}\Cella{0}{10}{r_1}\Cella{0}{10}{r_2}\Cella{0}{10}{r_3}\Cella{0}{10}{r_4}\Cella{\qrev{c}{m}.0}{10}{b} \\
   \qcell{(a,r_1,0.5), (a,r_2,0.5), (r_1, r_4, 0.5), (r_2, r_3, 0.5), (r_3,r_4, 0.5), (b,r_4,0.5)} \\
  \end{mathpar}


\begin{tikzpicture}[node distance={1.5cm}, thick, main/.style = {}] 
\node[main] (1) {S$\langle$\{(a,r_1,0.5), (a,r_2,0.5)\}$\cup$R$\rangle$}; 
\node[main] (2) [below left=2cm and 15cm of=1] {$\Cella{0}{9}{a}$ $\Cella{\qsend{0.5}{c}{m}|0}{9}{r_1}$... $\langle\{(r_1,r_4)\}\cup$R$\rangle$}; 
\node[main] (3) [below right=2cm and 15cm of=1] {\text{\ \ \ \ \ \ }$\Cella{0}{9}{a}$ $\Cella{\qsend{0.5}{c}{m}|0}{9}{r_2}$..., $\qcell{\{(r_2,r_3)\}\cup\text{R}}$}; 
\node[main] (4) [below of=2] {$\Cella{0}{8}{r_1}$ $\Cella{\qsend{0.25}{c}{m}|0}{9}{r_4}$... $\langle\{(r_4,b)\}\cup$R$\rangle$};
\node[main] (5) [below of=3] {\text{\ \ \ \ \ \ }$\Cella{0}{8}{r_2}$ $\Cella{\qsend{0.25}{c}{m}|0}{9}{r_3}$..., $\qcell{\{(r_3,r_4)\}\cup\text{R}}$};
\node[main] (6) [below of=4] {$\Cella{0}{8}{r_4}$ $\Cella{\qsend{0.125}{c}{m}|\qrev{c}{m}.0}{9}{b}$... $\qcell{\text{R}}$};
\node[main] (7) [below of=5] {\text{\ \ \ \ \ \ \ \ \;}$\Cella{0}{8}{r_3}$ $\Cella{\qsend{0.125}{c}{m}|0}{9}{r_4}$..., $\qcell{\{(r_4,b)\}\cup\text{R}}$};
\node[main] (8) [below of=6] {$\Cella{0}{9}{b}$... $\qcell{\text{R}}$};
\node[main] (9) [below of=7] {\text{\ \ \ \ \ \ \ \ }$\Cella{0}{8}{r_4}$ $\Cella{\qsend{0.0625}{c}{m}|\qrev{c}{m}.0}{9}{b}$..., $\qcell{\text{R}}$};
\node[main] (10) [below of=9] {\text{\ \ \ \ \ \ }$\Cella{0}{n}{b}$... $\qcell{\text{R}}$};
\draw[->] (1) -- node[midway, above left] {} (2); 
\draw[->] (1) -- node[midway, above right] {} (3); 
\draw[->] (2) -- node[midway, right] {} (4); 
\draw[->] (4) -- node[midway, right] {} (6);
\draw[->] (6) -- node[midway, right] {$0.125.c.m$} (8); 
\draw[->] (3) -- node[midway, right] {} (5); 
\draw[->] (5) -- node[midway, right] {} (7); 
\draw[->] (7) -- node[midway, right] {} (9);
\draw[->] (9) -- node[midway, right] {$0.0625.c.m$}  (10); 

\end{tikzpicture} 


% \begin{forest}
% for tree={circle,draw, l sep=20pt}
% [(S,\{(a,r_1), (a,r_2)\}\cup R) 
%     [(\Cella{0}{n}{a}\Cella{\qsend{c}{m}|0}{n}{r_1}\texttt{...}, \{(r_1,r_4)\}\cup R), edge label={node[midway,left]{(a,r_1)}} 
%       [(\Cella{0}{n}{r_1}\Cella{\qsend{c}{m}|0}{n}{r_4}\texttt{...}, \{(r_1,r_4)\}\cup R),edge label={node[midway,left] {(r_1,r_4)}} ] 
%       [1] 
%       [3]
%     ]
%     [2
%       [3] 
%       [2] 
%       [5]
%   ] 
% ]
% \end{forest}
   
   
   
   
    % (\Cella{\qsend{c}{m}.0}{n}{a}\Cella{0}{n}{r_1}\Cella{\qrev{c}{m}.0}{n}{b},\{(a,r_1,true), (b,r_1,true)\}\cup R) \\
    % \xrightarrow{(a,r_1,true), (b_r_1,true)} \\ (\Cella{0}{n-}{a}\Cella{0}{n--}{r_1}\Cella{\qsend{c}{m}\texttt{|}\qrev{c}{m}.0}{n-}{b},\{(a,r_1,false),(b,r_1,false)\}\cup R) \\
    % \xrightarrow{} \\
    % (\Cella{0}{n}{a}\Cella{0}{n}{r_1}\Cella{0}{n}{b},\{(a,r_1,true),(b,r_1,true)\}\cup R)
        


}
\caption{Quantum Pi Example}
  \label{fig:q-pi-example}
\end{figure}

% \Cref{fig:q-pi-syntax} provides the syntax of the language. Every channel is a list of qubits,
% written as $\parl{\parl{q_0}{...}}{q_n}$. We also have a predicate $S$ determining the adjacency of two channel qubits, i.e., two qubits are adjacent with each other if they are close to each other in distance.
% $\qchan{c}{P}$ and $\qchana{c}{T}$ are new operations referring to that a channel is built on top of a list of qubits. Process $T$ describes the behaviors of routers which holds a finite set of qubits waiting for constructing channels.
% \Cref{fig:q-pi-semantics} provides the reduction rules.
% Rule \rulelab{SCon1} to \rulelab{SCon3} are congruence rules such that the channel building operations commute with other operations.
% Rule \rulelab{Com} sends a quantum message from the left process to the right via a channel held by process $T$, while rule \rulelab{Chan} builds a long distance channel via $n$ qubits. For these two rules to happen, we need to check if all the qubits are adjacent with each other and the lifetime, checked by predicate $\texttt{time}$, of each qubit is not expired. Rule \rulelab{Com} is a probabilistic rule, such that the function \texttt{rate} produces a probability that the arrow will exist. The probability depends on the channel length.

\end{document}



% {\small
%   \begin{mathpar}

%     \inferrule[Com1]{}{R_1, R_2\equiv R_2, R_1}

%     \inferrule[Com2]{}{\langle\parl{R_1}{R_2}\rangle\equiv\langle\parl{R_2}{R_1}\rangle}
         
%   \inferrule[Gen]{q \not\in R \\ Size(R)< max\_size}{R \longrightarrow (R \cup [q]) }
         
%     \inferrule[Entangle]{}{\qact{q_1}{R_1}, \qact{q_2}{R_2} \rightharpoonup \langle\parl{R_1}{R_2}\rangle}
    
%     \inferrule[Decoherence]{}{\langle\parl{R_1}{R_2}\rangle \rightharpoondown R_1, R_2 }

%   \inferrule[ChanGen1]{\qact{q_1}{R_1}, \qact{q_2_1}{\qact{q_2_2}{R_2}} \rightharpoonup \langle\parl{R_1}{\qact{q_2_2}{R_2}}\rangle \qquad\qquad \qact{q_2_2}{R_2}, \qact{q_3}{R_3} \rightharpoonup \langle\parl{R_2}{R_3}\rangle }
%          {\qact{q_1}{R_1}, \qact{q_2_1}{\qact{q_2_2}{R_2}}, \qact{q_3}{R_3} \longrightarrow \alchan{(\langle\parl{R_1}{R_3}\rangle)}{\chansol{R_2}}  }
         
%   \inferrule[ChanGen2]{\qact{q_3}{R_3}, \qact{q_4}{R_4} \rightharpoonup \langle\parl{R_3}{R_4}\rangle }
%      {\alchan{(\langle\parl{R_1}{\qact{q_3}{R_3}}\rangle)}{\chansol{R_2}}, \qact{q_4}{R_4} \longrightarrow \alchan{(\langle\parl{R_1}{R_4}\rangle)}{\chansol{R_2, R_3}}  }


%   \inferrule[Com]{}
%       { \parl{\qact{\qsend{c}{m}}{P}}{\qact{\qrev{c}{x}}{P}}
%       \xrightarrow{\texttt{prob}(c),c.m} \parl{P}{Q} }

%   \end{mathpar}
% }
\section{The Quantum Abstract Machine:  Syntax and Semantics} \label{sec:qam}

This section describes QAM's syntax and semantics through the QPass example \cite{10.1145/3387514.3405853}.

\begin{figure}[t]
{\small
\begin{center}
\begin{tikzpicture}[node distance={1cm}, thick, main/.style = {draw, circle}] 
\node[main] (1) {\cn{Ann}}; 
\node[main] (2) [right of=1] {$r_1$};
\node[main] (3) [right of=2] {$r_4$};
\node[main] (4) [right of=3] {\cn{Bob}};
\node[main] (5) [above right=0.5cm and 1.5cm of 1] {$r_2$};
\node[main] (6) [right of=5] {$r_3$};
\node[main] (7) [below of=1] {\cn{Cat}};
\node[main] (8) [below of=3] {\cn{Dan}};
\draw[-] (1) --  (2);
\draw[-] (2) --  (3);
\draw[-] (3) --  (4);
\draw[-] (1) --  (5);
\draw[-] (5) --  (6);
\draw[-] (6) --  (3);
\draw[-] (7) --  (2);
\draw[-] (2) --  (8);
\end{tikzpicture}
\end{center}
}
\caption{QPass Example Path Connectivity}
  \label{fig:q-pi-example}
\end{figure}

\subsection{Capturing the Quantum Network Behaviors} \label{sec:qamexample}

As we mentioned in \Cref{sec:background}, all quantum network protocols are based on the physical procedures of quantum teleportation and quantum swap operations and their physical phenomena and mathematical semantics have been well studied.
On the other hand, building an evaluation framework for quantum network protocols does not require us to implement the complete physical procedure of individual operations; 
instead, we intend to capture the operational behaviors that affect the evaluation factors,
including the successful rates and probabilities of sending messages and the required resources.
In fact, both quantum teleportation and swap can be summarized as a procedure to communicate a qubit between two parties with the cost of two qubits.

Here, we investigate the modeling of message transmission in QAM by the QPass protocol \cite{10.1145/3387514.3405853}.
\Cref{fig:q-pi-example} shows an example connectivity diagram such that party $\cn{Ann}$ and $\cn{Cat}$ send messages to $\cn{Bob}$ and $\cn{Dan}$, via the middle routers $r_1$ to $r_4$. There are two paths for \cn{Ann} to send to \cn{Bob} ($r_2 - r_4$ and $r_1 - r_4$), while there is only one path to send message from \cn{Cat} to \cn{Dan} ($r_1$). 

In QAM, sending a message $a$ is viewed as a sequence of transmissions of ownership of $a$ from a node to is adjacent node.
For example, assume that we want to send message $a$ from \cn{Cat} to \cn{Dan}, so it locates at $a$ at first.
Then, $a$ is transmitted from \cn{Cat} to $r_1$ and from $r_1$ to \cn{Dan}, each of which costs $a$'s sender node two qubits.
There is also a probability reduction in guaranteeing that $a$ to be delivered to \cn{Dan}:
the more ownership transmissions, the lower probability that $a$ is delivered.
For example, the successful rate of the transmission from \cn{Cat} to $r_1$ and from $r_1$ to \cn{Dan} might be user-defined probability values $p$ and $p'$, so the successful rate to send a message from \cn{Cat} to \cn{Dan} is $p * p'$.
For simplicity, in the example in \Cref{fig:q-pi-example}, we assume that each station transmission reduces the successful rate by $50\%$,
i.e., if a message is transmitted from $a$ to $r_1$, it has $50\%$ chance that the message is lost.

In QAM, sending a message can be viewed as a sequence of transitions as $\to^{*}\xrightarrow{a}$,
where we might first execute several internal communication ($\to^{*}$, referring to $\tau$ transitions in classical process algebra),
with a final external communication $\xrightarrow{a}$, meaning that two parties are communicated through a message $a$.
When running a QAM program, there might be different ways where it can transit, which forms different sequences of transmissions that represent different possible ways of communication. It is also possible that some sequences only contain the internal communication part without the final external communication, which represents failure tries that there are some bottlenecks preventing the communication moving forward, because we want to allow different parties send messages via the same network at the same time period; however, this might cause message transition bottlenecks due to limited resources in different nodes.

Since quantum qubit information has lifetime, it might cause a message $a$ to lost completely if we simply queue a message at a bottleneck location until enough resources are available. Most quantum network protocols have the mechanism to bypass the message sending to another route. For example, assume that \cn{Ann} and \cn{Cat}'s messages are transmitted to $r_1$ at the same time, and $r_1$ only has two qubit resource and decide to send \cn{Cat}'s message first. Thus, \cn{Ann}'s message is queued and it is classified as a failure try that does not have an ending external communication. To successfully send both messages, we need \cn{Ann}'s message to be bypassed through the $r_2 - r_4$ route, so that the bottleneck in node $r_1$ is resolved and both messages can be delivered on time.

In the next two sections, we introduce the modeling of quantum network protocols via QAM formally.

\begin{figure}[t]
{\small
  \[\begin{array}{llcl} 
      \texttt{Variables} & x,y \\
      \texttt{Probability} & p\\
      \texttt{Messages} & m\\
    \texttt{Channels} & c\\
    \texttt{Global Time} & t\\
    \texttt{Actions} & a & ::= & c \times m\\
      \texttt{Singleton Process} & A & ::= & \emptyset  \mid \qsenda{c}{m} \mid \qsend{p}{c}{m}^t \mid \qrev{c}{x} \\
      \texttt{Process} & P,Q & ::= & 0 \mid A.P \mid \parl{P}{Q} \mid \comp{P} \\
      \texttt{Process Cell} & \varphi & ::= & \pcell{P}{n,t}{c} \\
      \texttt{Relations} & R & ::= & [c \times c \times p] \\
      \texttt{Relation Functions} & \Fs & ::= & t \to R\\
    \texttt{Action Probabilities} & \rho & ::=& a \to p\\
      \texttt{Configuration} & C & ::= & \varphi^* \qcell{R}_{\textsf{rel}}\qcell{a}_{\textsf{comm}}\qcell{\Fs}_{\textsf{fun}}\qcell{t}_{\textsf{gt}}\qcell{\rho}_{\textsf{at}}
    \end{array}
  \]
}
\caption{Quantum Pi Syntax}
  \label{fig:q-pi-syntax}
\end{figure}


\subsection{Syntax} \label{sec:qamsyntax}

As shown in \Cref{fig:q-pi-syntax}, the execution of a QAM program relies on the concept of 
a configuration that describes conceptual computer components for QAM program interactions.
A configuration can possibly contain many cell structures $\qcell{\Theta}_{\eta}$, where $\Theta$ is the content of $\eta$ is the name.
For example, $\qcell{\Ls}_{p}$ is a program execution cell with name $p$, and program execution configuration as its content $\Ls$,
Cell $\qcell{\Fs}_{f}$ holds a function $\Fs$ representing the dynamically updated network node connectivity with respect to time, and $\qcell{\tau}_{t}$ is a global time cell representing periods of time for updating the connectivity graph.

Program executions represent multiple communications of quantum messages through different nodes in a network.
We assume that a single message communication always happens presently within a global time period so that during a message communication, the global clock remains the same.
A program execution configuration acts as the content $\Ls$ of the program execution cell $\qcell{\Ls}_{p}$ with the syntax $\Ss \qcell{\Rs}_r$, where cell $\qcell{\Rs}_r$ stores the node connectivity at a time, $\Ss$ is a set of cells representing different nodes.
The content $\Rs$ in a relation cell contains relations defining the connectivity of computer locations.
Each relation triple defines an adjacent pair of two locations, i.e., 
a computer can send a message to another location only if the two locations are adjacent.
There is also a probability value defining the probability of the successful rate of the communication between the two locations.

In a node cell, $c$ represents the node name and $n$ represents the number of qubit resources in the node.
$A (\texttt{|} P )^?$ is a Pi-calculus lite process language representing the communication program content in the node.
$A$ represents a message action that is about to send and receive from the node. $P$ represents a possible local process for the node to execute after it receives a message.
An action is either a message send ($\qsend{p}{c}{m}$) or receipt ($\qrev{c}{x}$) operation,
with $c$ being a channel name and $m$ is a message. $p$ is the likelihood of the send operation.
A local process consists with a chain of actions ($A.P$) that ends at a terminated process ($0$).
If a node cell only contains a singleton process,
it means that the cell is not ready for communication and the actions in the process will be transmitted to other cells.

{\small
\[
\begin{array}{l}
\texttt{S} \qcell{\texttt{R}}_r \equiv 
\Cella{\qsend{1}{c}{m}|0}{10}{a}\Cella{0}{10}{r_1}\Cella{0}{10}{r_2}\Cella{0}{10}{r_3}\Cella{0}{10}{r_4}\Cella{\qrev{c}{m}.0}{10}{b} 
\\\qquad\qquad
\qcell{(a,r_1,0.5), (a,r_2,0.5), (r_1, r_4, 0.5), (r_2, r_3, 0.5), (r_3,r_4, 0.5), (b,r_4,0.5)} 
\end{array}
\]
}

As an example of the QAM syntax, the above program execution configuration defines the program to send a message from location $a$ to $b$.
The first line corresponds to node cells $\texttt{S}$, containing the target and destination nodes $a$ and $b$ as well as intermediate routers $r_1$ to $r_4$; while the second line corresponds to relation cell $\qcell{\texttt{R}}_r$ capturing the connectivity in \Cref{fig:q-pi-example}.

\begin{figure}[t]
{\small
  \begin{mathpar}
   
   \inferrule[GenChanT]{}
       {\Cellb{\qsenda{c}{m}}{i}{a}\Cella{P}{j}{b} \qcell{\{(a,b,p)\}\cup R}_{\textsf{rel}}\qcell{t}_{\textsf{gt}}
        \longrightarrow \Cellb{}{i-}{a}\Cella{\qsend{p}{c}{m}^t\texttt{|}P}{j-}{b}\qcell{\{(a,b,p)\}\cup R}_{\textsf{rel}} \qcell{t}_{\textsf{gt}} }

   \inferrule[GenChan]{}
       {\Cellb{\qsend{p}{c}{m}^t}{i}{a}\Cella{P}{j}{b} \qcell{\{(a,b,p)\}\cup R}_{\textsf{rel}}
        \longrightarrow \Cellb{}{i-}{a}\Cella{\qsend{p}{c}{m}^t\texttt{|}P}{j-}{b}\qcell{\{(a,b,p)\}\cup R}_{\textsf{rel}} }


   \inferrule[GenQubit]{}
       {\Cella{P}{n,t'}{a}\qcell{t}_{\textsf{gt}}\longrightarrow \Cella{P}{n+,t}{a}\qcell{t}_{\textsf{gt}}}\;\;\texttt{when}\;t \texttt{|} \beta\wedge t' < t

   \inferrule[MoreTries]{}
       {\comp{P} \longrightarrow \parl{P}{\comp{P}}}
      
   \inferrule[NoTries]{}
       {\comp{P} \longrightarrow 0}

  \inferrule[Communication]{}
      { \qcell{\qsend{p}{c}{m}^t\texttt{|} \qrev{c}{x}.P}^n_c\qcell{\emptyset}_{\textsf{comm}}
           \longrightarrow  
         \qcell{P[m/x]}^n_c\qcell{\qsend{p}{c}{m}^t}_{\textsf{comm}}}
    
  \inferrule[topup]{}
      { \qcell{\qsend{p}{c}{m}^{t'}}_{\textsf{comm}}\qcell{\rho}_{\textsf{at}}\qcell{t}_{\textsf{gt}} \xrightarrow{c.m} 
        \qcell{\emptyset}_{\textsf{comm}}\qcell{\rho[c.m\mapsto \rho(c.m)\odot p]}_{\textsf{at}}\qcell{t+}_{\textsf{gt}}}\;\;\texttt{when}\;\rho(c.m)\odot p<\mu\wedge t - t' < \nu

  \inferrule[RelationUp]{}
      { \qcell{R}_{\textsf{r}}\qcell{\Fs}_{\textsf{fun}}\qcell{t}_{\textsf{gt}} \longrightarrow
            \qcell{\Fs(t)}_{\textsf{r}}\qcell{\Fs}_{\textsf{fun}}\qcell{t}_{\textsf{gt}}}

  \end{mathpar}
}
\caption{Quantum Pi Semantics. $\beta$, $\mu$, and $\nu$ are globally defined for the qubit generation period, the message threshold probability, and message sending finished threshold. $\Cella{P}{n}{a}$ refers to that the $t$ in $\Cella{P}{n,t}{a}$ is omitted in the rule.}
  \label{fig:q-pi-semantics}
\end{figure}


\begin{figure}[t]
{\small
{\hspace*{-2em}
\begin{tikzpicture}[align=center,node distance=1.5cm and -1cm, thick] 
\node (1) {S$\langle\{(a,r_1,0.5), (a,r_2,0.5)\}\cup$R$\rangle$}; 
\node (2) [below left= of 1] {$\Cella{0}{9}{a}$ $\Cella{\qsend{0.5}{c}{m}|0}{9}{r_1}$... $\langle\{(r_1,r_4)\}\cup$R$\rangle$}; 
\node (3) [below right= of 1] {\text{\ \ \ \ \ \ }$\Cella{0}{9}{a}$ $\Cella{\qsend{0.5}{c}{m}|0}{9}{r_2}$..., $\qcell{\{(r_2,r_3)\}\cup\text{R}}$}; 
\node (4) [below of=2] {$\Cella{0}{8}{r_1}$ $\Cella{\qsend{0.25}{c}{m}|0}{9}{r_4}$... $\langle\{(r_4,b)\}\cup$R$\rangle$};
\node (5) [below of=3] {\text{\ \ \ \ \ \ }$\Cella{0}{8}{r_2}$ $\Cella{\qsend{0.25}{c}{m}|0}{9}{r_3}$..., $\qcell{\{(r_3,r_4)\}\cup\text{R}}$};
\node (6) [below of=4] {$\Cella{0}{8}{r_4}$ $\Cella{\qsend{0.125}{c}{m}|\qrev{c}{m}.0}{9}{b}$... $\qcell{\text{R}}$};
\node (7) [below of=5] {\text{\ \ \ \ \ \ \ \ \;}$\Cella{0}{8}{r_3}$ $\Cella{\qsend{0.125}{c}{m}|0}{9}{r_4}$..., $\qcell{\{(r_4,b)\}\cup\text{R}}$};
\node (8) [below of=6] {$\Cella{0}{9}{b}$... $\qcell{\text{R}}$};
\node (9) [below of=7] {\text{\ \ \ \ \ \ \ \ }$\Cella{0}{8}{r_4}$ $\Cella{\qsend{0.0625}{c}{m}|\qrev{c}{m}.0}{9}{b}$..., $\qcell{\text{R}}$};
\node (10) [below of=9] {\text{\ \ \ \ \ \ }$\Cella{0}{n}{b}$... $\qcell{\text{R}}$};
\draw[->] (1) -- node[midway, above left] {} (2); 
\draw[->] (1) -- node[midway, above right] {} (3); 
\draw[->] (2) -- node[midway, right] {} (4); 
\draw[->] (4) -- node[midway, right] {} (6);
\draw[->] (6) -- node[midway, right] {$0.125.c.m$} (8); 
\draw[->] (3) -- node[midway, right] {} (5); 
\draw[->] (5) -- node[midway, right] {} (7); 
\draw[->] (7) -- node[midway, right] {} (9);
\draw[->] (9) -- node[midway, right] {$0.0625.c.m$}  (10); 
\end{tikzpicture} 
}
}
\caption{Quantum Machine Transitions for \Cref{fig:q-pi-example}}
  \label{fig:q-pi-example1}
\end{figure} 

\subsection{Semantics} \label{sec:qamsemantics}

The QAM semantics is given as a process algebra style labeled transition system listed in \Cref{fig:q-pi-semantics}.
Each transition label has either an empty label or a communication action on label $p.c.m$, where $p$ is the probability the action $c.m$ happens and $c$ is the channel name and $m$ is the actual message potentially being quantum.
Rule \rulelab{RelationUp} describes the global transitions where program execution connectivity graph is updated once upon a time,
while rule \rulelab{TimeUp} updates the global clock of the configuration.
In QAM, the semantic rules only need to mention the cells that are involved in a transition.
For example, in rule \rulelab{TimeUp}, we do not mention the cells $\qcell{\Ls}_{p}$ and $\qcell{\Fs}_{f}$ because they have no changes during the transition.

Rules \rulelab{GenChan}, \rulelab{GenQubit}, and \rulelab{Communication} describe the transitions happened in a program configuration.
Rule \rulelab{GenChan} represents the procedure that a message is transmitted from one location to another via a quantum channel. During the procedure, the probability for the message being successfully sent is reduced and the two locations also consume two qubits.
Here, an action $\qsend{p}{c}{m}$ is transmitted from one location, defined by the cell $\qcell{}_a$, to another location $\qcell{}_b$, if there is a triple relation $(a,b,p')$ in the relation cell. 
The $p'$ value represents the reduced probability factor to communicate the message between the two locations, i.e.,
the probability value $p$ in action $\qsend{p}{c}{m}$ is accumulated to become $p\cdot p'$.
Each transmission requires the consumption of two qubits: each cell consumes one qubits.
This is why $a$ and $b$'s result cells ($\Cellb{}{i-}{a}$ and $\Cellb{}{j-}{b}$) have the qubit value $i$ and $j$ being reduced, respectively.
Here, the $...$ in cell $\Cellb{}{i-}{a}$ represents that we only care about the left hand side action in a process cell regarding the rest of the content.
In the example transition in \Cref{fig:q-pi-example1} that starts from the configuration $\texttt{S} \qcell{\texttt{R}}_r$ above,
the transitions from the top to the left and right are both applied a \rulelab{GenChan} rule.
On the left hand side, the action $c!<m>$ is transmitted from cell $a$ to cell $r_1$, and the probability value is reduced to $0.5$.
Rule \rulelab{GenQubit} generates a new qubit in a location to help transmit messages from one location to another.

Rule \rulelab{Communication} defines the behavior of consuming a message by a local process in a node cell.
In the parallel composition $\qsend{p}{c}{m}\texttt{|} \qrev{c}{x}.P$, action $\qsend{p}{c}{m}$ is waiting to be consumed on channel $c$, 
$\qrev{c}{x}$ is a receipt binding of channel $c$, and $P$ is the following process to consume the action.
After the transition, the waiting action is removed, and the remaining process $P$ is executed by substituting variable $x$ with the massage $m$. The transition is a non-empty one and the parallel composition communicates on the action $p.c.m$.

An example machine is given in \Cref{fig:q-pi-example} and \Cref{fig:q-pi-example1}.
The protocol of the definition follows the QPASS model \cite{10.1145/3387514.3405853}. 
The example contains six distinct computer locations and we want to send messages from location $a$ to $b$.
Locations $r_1$, $r_2$, $r_3$, and $r_4$ are intermediate routers. 
\Cref{fig:q-pi-example} represents the connectivity of different computers. 
\Cref{fig:q-pi-example1} provides the transition computation tree of two possible ways of sending a message from $a$ to $b$.
At the end of each path in the computation tree in \Cref{fig:q-pi-example1}, the left hand path communicates an action $0.125.c.m$, while the right hand path communicates an action $0.0625.c.m$; thus, the left hand path is selected according to the QPass model because it has a higher successful rate to send out the message.




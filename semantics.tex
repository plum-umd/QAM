\section{The Quantum Abstract Machine:  Syntax and Semantics} \label{sec:qam}

\begin{figure}[t]
{\small
  \[\begin{array}{llcl} 
      \texttt{Variable} & x,y \\
      \texttt{Probability} & p &\in &\Rs\\
      \texttt{Message} & m &\in& \mathbb{M}\\
    \texttt{Channel} & c &\in& \Ls\\
    \texttt{Time Stamp} & t &\in& \mathbb{T}\\
    \texttt{Label} & \alpha &::=& c.m \mid p.c.m \mid (p.c.m)^t \mid \emptyset\\
      \texttt{Configuration} & C & \\
      \texttt{Rule} & \rules & ::= & C \xrightarrow{\alpha} C 
    \end{array}
  \]
}
\caption{Quantum Abstract Machine Syntax Table}
  \label{fig:q-pi-syntax}
\end{figure}

This section describes QAM's syntax and semantics, whose design is the abstraction of different quantum network protocols based on the observations given in \Cref{sec:introduction}.

We define QAM system to represent the transitions for any quantum network protocols, which allows programmers to define initial configurations, representing the initial programs and process environment, as well as user-definable transition rules, guiding how configurations are transitioned. A QAM system is a structure $(\Ms,\Ls,\Ts,\overline{\rules})$,
where $\Ms$ is a set of messages;
$\Ls$ is a set of channels;
$\Ts$ is a set of time stamps that forms a linear order ($<$) with $0$ being the minimum;
and $\overline{\rules}$ is a finite set of rules for guiding how configurations are transitioned. 
Any rule has the form $C_1 \xrightarrow{\alpha} C_2$, referring to that we transition configuration $C_1$ to $C_2$ via a label $\alpha$, being either emptily internal ($\emptyset$), a pair $c.m$, a triple $p.c.m$, or a quadruple $(p.c.m)^t$, where $c$ is the channel for communication, $m$ is a possibly quantum message, $p$ is the success rate of the message delivery, and $t$ is the initial time stamp of the message. In QAM, rules are not allowed to associated with side conditions, and we provide special ways of such definitions, explained in \Cref{sec:qamsyntax1}.

The execution in QAM is with respect to a QAM system $(\Ms,\Ls,\Ts,\overline{\rules})$ and an initial configuration $C$ that defines the input program and initial process environment.
If we collect the free metavariables ($\cn{FV}(-)$) appearing in a configuration and rule, 
the initial configuration $C$ is a \textit{ground term} without any metavariables ($\cn{FV}(C)=\emptyset$);
while $\cn{FV}(\alpha) \cup \cn{FV}(C_2)\subseteq \cn{FV}(C_1)$ for every rule $C_1 \xrightarrow{\alpha} C_2$, which is the well-formedness assumption for a QAM system.
We can define the transitions in a QAM system as follows:

\begin{definition}\label{def:labeledsystem}\rm[One QAM Transition Step]
Given a QAM system $\Cs=(\Ms,\Ls,\Ts,\overline{\rules})$ and a ground term initial configuration $C$, a transition step defining for a rule $C_1 \xrightarrow{\alpha} C_2 \in \overline{\rules}$ on $C$ is given as:
\begin{itemize}
\item We find a substitution $\sigma$ mapping every metavariable in $C_1$ to a term, such that $\sigma(C_1)=C$, where we substitute metavariables $x$ in $C_1$ with the mapped terms as $\sigma(x)$.
\item The result label and configuration by applying the rule is $\sigma(\alpha)$ and $\sigma(C_2)$.
\end{itemize}
\end{definition}

\Cref{def:labeledsystem} provides an abstraction of transitions in QAM, where the details configurations and rules are parameterized as some abstract objects. Below, we provide step by step instantiation of configurations and rules to capture different quantum network protocol behaviors. 

\subsection{Intra-Destination Process Communication} \label{sec:qamsyntax}


\begin{figure}[t]
{\small
$\textcolor{blue}{\text{Syntax:}}\\$
  \[\begin{array}{llcl} 
    \texttt{Singleton Action} & A & ::= & \qsend{p}{c}{m}^{t?} \mid \qrev{c}{x}\\
      \texttt{Process} & P,Q & ::= & 0 \mid A.P \mid \parl{P}{Q} \mid \parp{P}{Q}\mid \comp{P} \\
      \textcolor{red}{\texttt{Configuration First Pattern}} & \textcolor{red}{C} & \textcolor{red}{::=} & \textcolor{red}{\overline{P}}
    \end{array}
  \]

$\textcolor{blue}{\text{Semantics:}}\\$
  \begin{mathpar}
\mprset{flushleft}

   \inferrule[Heating]{}
       {\parl{P}{Q} \longrightarrow \pard{P}{Q}}

   \inferrule[Cooling]{}
       {\pard{P}{Q} \longrightarrow \parl{P}{Q}}

   \inferrule[ID]{}
       {\pard{0}{P} \longrightarrow P}

  \inferrule[CL]{}
      {\parp{P}{Q} \longrightarrow P}

  \inferrule[CR]{}
      {\parp{P}{Q} \longrightarrow Q}

  \inferrule[Com]{}
      { \pard{\qsend{p}{c}{m}^t.Q}{ \qrev{c}{x}.P}
           \xrightarrow{(p.c.m)^t} \pard{Q}{P[m/x]}}

   \inferrule[MT]{}
       {\comp{P} \longrightarrow \pard{P}{\comp{P}}}
      
   \inferrule[NT]{}
       {\comp{P} \longrightarrow 0}
  \end{mathpar}
}
\caption{Single-party process communication syntax and semantics. The $?$ mark in $\qsend{p}{c}{m}^{t?}$ means that the time stamp $t$ can be omitted. $\emptyset$ labels are omitted. }
  \label{fig:q-pi-semantics1}
\end{figure}

We first investigate the quantum communication between parties in a single location.
Essentially, quantum teleportation can be split into two parts: 1) using quantum swap mechanism to build the channel for communicating Alice and Bob; and 2) Bob reproduces the quantum message by a piece of the message and two classical bits.
Here, we model (1) along with the modeling for quantum routing because they are essentially the same, while we model the intra-destination level communication mainly focuses on Bob's receipt of the quantum message, which can be modeled based on the CHEM mechanism in \Cref{fig:q-pi-semantics1}.

The syntax of the intra-destination level communication is enlightened by $\Pi$-calculus, as we instantiate QAM configurations as a multiset of processes $\overline{P}$, referring to the red part in \Cref{fig:q-pi-semantics1}.
Each single process action is either a message sending $\qsend{p}{c}{m}^{t}$, meaning that a message $m$ is sent through the channel $c$ with the success probability rate $p$ initialized at the time stamp $t$, or a message receipt $\qrev{c}{x}$, referring to that a message is received through the channel $c$ and represented as variable $x$ in the following computation.
The time stamp $t$ in a message sending operation can be omitted as $\qsend{p}{c}{m}$, meaning that an initial time stamp will be generated when the message is sent out from its starting place.

A process can be an empty process $0$, a process ($A.P$) executing a singleton action with continuation, a parallel process ($\parl{P}{Q}$) of two parties $P$ and $Q$, a choice operation($\parp{P}{Q}$), and a replication process $\comp{P}$ referring to that $P$ can repeatedly happen zero or multiple times. As an instance, process $Q$ in $\parl{\qact{\qsend{0.5}{\cn{Ann}}{m}^0}{P}}{\qact{\qrev{\cn{Ann}}{x}}{Q}}$ represents that \cn{Ann} receives a message $m$, initializing at time $0$, from the process $P$ in a $50\%$ change; after $m$'s receipt, the message is represented as variable $x$ in the continuation of process $Q$.

The intra-destination level semantics in \Cref{fig:q-pi-semantics1} is inherited from CHEM.
One thing worth noting is that we omit $\emptyset$ labels in all the internal labeled rules.
Through the \rulelab{Heating} rule, a parallel process $\parl{P}{Q}$ is dissolved to a multiset ($\pard{P}{Q}$) containing two elements $P$ and $Q$, ready for semantic evaluation; while the \rulelab{Cooling} rule merges two elements in a multiset ($\pard{P}{Q}$) as a parallel process. Rules \rulelab{Heating}, \rulelab{Cooling}, and \rulelab{ID} captures the implicit associativity, commutativity, and identity equational properties for QAM processes, so that any two processes in the process multiset might communicate, without syntactic barriers. For example, in a multiset, rule \rulelab{Com} defines the behavior of a sender process, on the left, communicating an action $\qsend{p}{c}{m}^t.Q$ with a receiver process $\qrev{c}{x}.P$, on the right, and resulting in the two multiset elements $\pard{Q}{P[m/x]}$, indicating that process $P$ consumes the message $m$ as variable $x$ in the following continuation. The above three rules allow us not to care about the positions of the sender and receiver in the multiset.

Rules \rulelab{CL} and \rulelab{CR} define conditional choice operations. For example, if we have:
$\parl{\qact{\qsend{0.5}{\cn{Ann}}{m}^0}{P}}{\parp{\qact{\qrev{\cn{Ann}}{x}}{Q}}{\qact{\qrev{\cn{Ann}}{x}}{Q'}}}$, 
we can nondeterministically choose the receiver process as $Q$ or $Q'$:
$\parl{\qact{\qsend{0.5}{\cn{Ann}}{m}^0}{P}}{\qact{\qrev{\cn{Ann}}{x}}{Q}}$
or
$\parl{\qact{\qsend{0.5}{\cn{Ann}}{m}^0}{P}}{\qact{\qrev{\cn{Ann}}{x}}{Q'}}$.
Rules \rulelab{MT} and \rulelab{NT} define the nondeterministic behavior of a replication operation $\comp{P}$,
i.e., to replicate zero or more processes $P$. In defining quantum network protocols, 
such operations represent the concept of repeatedly sending a same message to ensure success rate for message delivery.
We will explain the utility shortly in \Cref{sec:add-prop}.

\subsection{Inter-Destination Communication: Frames} \label{sec:qamsyntax1}

\begin{figure}[t]

{\small
$\textcolor{blue}{\text{Additional Syntax:}}\\$
  \[\begin{array}{llcl} 
      \texttt{Communication Location Flag} & \beta & ::= & \emptyset \mid c \mid c \rightarrow c\\[0.2em]
      \texttt{Process Cell} & \varphi & ::= & \pcell{\overline{P}}{n}{c}\\[0.2em]
      \textcolor{red}{\texttt{Configuration Second Pattern}} & \textcolor{red}{C} & \textcolor{red}{::=} & 
\textcolor{red}{\overline{\varphi}\pcell{\alpha}{\beta}{\cn{comm}} \qcell{\cn{bool}}{\cn{pred}}}
    \end{array}
  \]

$\textcolor{blue}{\text{Semantics:}}\\$
  \begin{mathpar}
\mprset{flushleft}
   \inferrule[GC]{}
       {\pcell{\pard{\qsend{p}{c}{m}^t.P}{...}}{\cn{S}\;i}{c_1}\pcell{ ...}{\cn{S}\;j}{c_2}
        \qcell{\emptyset}{\cn{comm}}
        \longrightarrow \pcell{\pard{P}{...}}{i}{c_1}\pcell{\pard{\qsend{p}{c}{m}^t}{...}}{j}{c_2}
               \pcell{(p.c.m)^t}{c_1\rightarrow c_2}{\cn{comm}}}

  \inferrule[PC]{\overline{P}\xrightarrow{\alpha^t} \overline{Q} }
      { \pcell{\overline{P}}{n}{c}\qcell{\emptyset}{\cn{comm}}
           \longrightarrow
         \pcell{\overline{Q}}{n}{c}\pcell{\alpha^t}{c}{\cn{comm}}}
               
   \inferrule[Grant]{}
       {\qcell{\cn{F}}{\cn{pred}} \longrightarrow \qcell{\cn{T}}{\cn{pred}}}
  
  \inferrule[FC]{}
      { \pcell{\alpha}{c_1\rightarrow c_2}{\cn{comm}}\qcell{\cn{T}}{\cn{pred}}
           \longrightarrow \qcell{\emptyset}{\cn{comm}}\qcell{\cn{F}}{\cn{pred}} } 
 
  \inferrule[AC]{}
      { \pcell{\alpha^t}{c}{\cn{comm}}\qcell{\cn{T}}{\cn{pred}}
           \xrightarrow{\alpha}  \qcell{\emptyset}{\cn{comm}}\qcell{\cn{F}}{\cn{pred}} } 

  \end{mathpar}
}
\caption{Inter-destination communication syntax and semantics. $\beta$ can be omitted when being $\emptyset$. $\alpha^t$ is an action such that $\alpha$ has the form $p.c.m$. $\qsend{p}{c}{m}^t\equiv \qsend{p}{c}{m}^t.0$. $\cn{T}$/$\cn{F}$: true or false values.}
  \label{fig:q-pi-semantics2}
\end{figure}

\begin{figure}[t]
{\footnotesize
\[
\begin{array}{lll}
&
\pcellb{\qsend{1}{c}{m}^0.0}{10}{\cn{Cat}}
\pcellb{\qrev{c}{x}.0}{10}{\cn{Dan}} 
\qcellb{\emptyset}{\cn{comm}}
\qcellb{\texttt{F}}{\cn{pred}}
&
\\[0.3em]
\longrightarrow
&
\pcellb{0}{9}{\cn{Cat}}
\pcellb{\pard{\qsend{1}{c}{m}^0.0}{\qrev{c}{x}.0}}{9}{\cn{Dan}} 
\pcellb{(1.c.m)^0}{\cn{Cat}\rightarrow\cn{Dan}}{\cn{comm}}
\qcellb{\texttt{F}}{\cn{pred}}
&
(\rulelab{GC})
\\[0.3em]
\longrightarrow
&
\pcellb{0}{9}{\cn{Cat}}
\pcellb{\pard{\qsend{1}{c}{m}^0.0}{\qrev{c}{x}.0}}{9}{\cn{Dan}} 
\pcellb{(1.c.m)^0}{\cn{Cat}\rightarrow\cn{Dan}}{\cn{comm}}
\qcellb{\texttt{T}}{\cn{pred}}
&
(\rulelab{Grant})
\\[0.3em]
\longrightarrow
&
\pcellb{0}{9}{\cn{Cat}}
\pcellb{\pard{\qsend{1}{c}{m}^0.0}{\qrev{c}{x}.0}}{9}{\cn{Dan}} 
\qcellb{\emptyset}{\cn{comm}}
\qcellb{\texttt{F}}{\cn{pred}}
&
(\rulelab{FC})
\\[0.3em]
\longrightarrow
&
\pcellb{0}{9}{\cn{Cat}}
\pcellb{0}{9}{\cn{Dan}} 
\pcellb{(1.c.m)^0}{\cn{Dan}}{\cn{comm}}
\qcellb{\texttt{F}}{\cn{pred}}
&
(\rulelab{Com}+\rulelab{PC})
\\[0.3em]
\longrightarrow
&
\pcellb{0}{9}{\cn{Cat}}
\pcellb{0}{9}{\cn{Dan}} 
\pcellb{(1.c.m)^0}{\cn{Dan}}{\cn{comm}}
\qcellb{\texttt{T}}{\cn{pred}}
&
(\rulelab{Grant})
\\[0.3em]
\xrightarrow{1.c.m}&
\pcellb{0}{9}{\cn{Cat}}
\pcellb{0}{9}{\cn{Dan}} 
\qcellb{\emptyset}{\cn{comm}}
\qcellb{\texttt{F}}{\cn{pred}}
&
(\rulelab{AC})
\end{array}
\]
}
\caption{Example transitions based on the model in \Cref{fig:q-pi-semantics2}.}
  \label{fig:q-pi-example}
\end{figure}

To model inter-destination communication, i.e., message transformations across different locations,
the QAM system is inspired by the abstract machine concept that models communication as sequencing steps of
interactions with different abstract environment components.
Here, we instantiate the QAM configuration to a multiset of cells in \Cref{fig:q-pi-semantics2},
as we utilize a membrane (cell) in chemical abstract machine to model a distinct component in a network protocol.
Processes in different destination lives in a process cell $\pcell{\overline{P}}{n}{c}$, where $\overline{P}$ is a process multiset for communication, $n$ is a linear order indicating the qubit resource in the location, and $c$ is the location name.
The other cells (\cn{comm} and \cn{pred}) store environment information, which represent the QAM ways to define side conditions, since rule syntax does not allow side condition definitions, explained shortly below. 
\cn{comm} cell ($\pcell{\alpha}{\beta}{\cn{comm}}$) contains a flag $\beta$ defining the location the label $\alpha$ happens: $c$ refer to that the transition happens at $c$ and $c_1\rightarrow c_2$ means that the transition happens at conveying messages from $c_1$ to $c_2$.

The configuration instantiation to a cell multiset in \Cref{fig:q-pi-semantics2} provides a frame for semantic rule definitions
and allow rule definitions to mention only the necessary parts regardless the other interactions.
For example, rule \rulelab{Grant} mentions only cell \cn{pred} and it means that we can always turn cell \cn{pred}'s value \cn{T} regardless the other cells. Additionally, the configuration extension from the one in \Cref{fig:q-pi-semantics1} to \Cref{fig:q-pi-semantics2} does not invalidate the rules in the former, as they are still valid for intra-process communications.
In this sense, the configuration setup, based on the extension of previous configuration, permits gradual semantic development for quantum network protocols with only minimal changes in the rule definitions.

To definite protocol communication nondeterminism in QAM, every communication transition is divided into three basic steps: \textit{communicating}, \textit{granting}, and \textit{cleaning}. 
A communicating step performs the actual transition and moves the result label to the \cn{comm} cell waiting for granting.
In \Cref{fig:q-pi-semantics2}, rules \rulelab{GC} and \rulelab{PC} performs a communicating step.
The former conveys a message sending action from locations $c_1$ to $c_2$ and moves the label $(p.c.m)^t$ to the \cn{comm} cell with the $c_1\rightarrow c_2$ flag referring to that the action conveys between the two locations.
The latter performs an intra-process communication in the $c$ cell with the label $\alpha^t$ ($\alpha$ has form $p.c.m$) and moves the label to the \cn{comm} cell with the $c$ flag.
A granting step validates a communicating step through possibly additional predicate judgment applying on the \cn{comm} cell labels, with the result storing in the \cn{pred} cell. For the simplest case, we define rule \rulelab{Grant} to grant every communicating step.
A cleaning step initializes the \cn{comm} and \cn{pred} cells for next communication transitions to perform.
Rules \rulelab{FC} and \rulelab{AC} are cleaning rules corresponding to rules \rulelab{GC} and \rulelab{PC}, respectively.
The former purges the two cells after a \rulelab{GC} rule is performed,
while the latter purges the two cells after a message is delivered by applying rule \rulelab{PC} and it also moves the delivered message label $\alpha$ to the transition arrow as $\xrightarrow{\alpha}$, making the communication external for further analysis, discussed in \Cref{sec:refinement}.  

Evaluating an initial configuration
{
$\pcell{\qsend{1}{c}{m}^0.0}{10}{\cn{Cat}}
\pcell{\qrev{c}{x}.0}{10}{\cn{Dan}} 
\qcell{\emptyset}{\cn{comm}}
\qcell{\texttt{F}}{\cn{pred}}$
}
results in the transitions steps in \Cref{fig:q-pi-example}.
The initial configuration instantiates the parameters in the configuration pattern in \Cref{fig:q-pi-semantics2}.
Specifically, we initialize the process cells $\overline{\varphi}$ as two distinct elements: \cn{Cat} and \cn{Dan},
with \cn{Cat} containing a sending process and \cn{Dan} containing a receiver process; 
\cn{comm} is initialized as $\emptyset$ and \cn{pred} has value \cn{F}.
In the beginning \cn{Cat} and \cn{Dan} have $10$ pieces of qubit resources \footnote{A piece of qubit resource is user definable and it might not be exactly as one qubit.}.
We first apply rule \rulelab{GC} on the initial configuration
that conveys the sending operation from \cn{Cat} to \cn{Dan},
with the following two rule applications (\rulelab{Grant} and \rulelab{FC}) to grant and purge the \rulelab{GC} step.
During the \rulelab{GC} application, \cn{Cat} and \cn{Dan} lose a piece of resource that is enough to send out the quantum message $m$. 
The next step applies rule \rulelab{PC} that delivers a message the label $(1.c.m)^0$;
such step requires the application of an extra rule \rulelab{Com} to communicates the sender and receiver in the \cn{Dan} cell.
Then, the two following applications (\rulelab{Grant} and \rulelab{AC}) to grant and purge this step.

The final configuration results in $\pcell{0}{9}{\cn{Cat}}
\pcell{0}{9}{\cn{Dan}} 
\qcell{\emptyset}{\cn{comm}}
\qcell{\texttt{F}}{\cn{pred}}$
having process cells being all unit processes $0$ as well as cells \cn{comm} and \cn{pred} are purged.
Such configuration is a satisfied terminated state. However, configuration can transition to a stuck state, i.e., an error state that is not a satisfied final state.
We define a satisfied terminated and stuck configurations below.

\begin{definition}\label{def:stuck}\rm[Terminated and Stuck Configurations]
A configuration $C$ is a terminated state if no rule is possible to apply to $C$ and:
\begin{itemize}
\item For all process cell in $C$, its content is a unit process $0$.
\item $\alpha$ and $\beta$ in $\pcell{\alpha}{\beta}{\cn{comm}}$ are $\emptyset$ and $b$ in $\qcell{b}{\cn{pred}}$ is \cn{F}.
\end{itemize}
$C$ is stuck if no rule is possible to apply to it and the above conditions are not satisfied.
\end{definition}


\ignore{
As an example of the QAM syntax, the above configuration defines the initial program state for sending a message $c.m$ from $\cn{Cat}$ to $\cn{Dan}$ via the router $r_1$, as part of the communication in \Cref{fig:q-pi-example}. Initially, the relation cell stores the connectivity between \cn{Cat}, \cn{Dan}, and $r_1$, with the success rates. 
The global time is initialized as $0$ in the \cn{gt} cell, and the predicate cell has a fixed value $\texttt{false}$.
The message ($\qsend{1}{c}{m}$) sent from \cn{Cat} has an initial probability value $1$.
Node $r_0$ acts as a intermediate router, so it only contains the unit process $0$, and \cn{Dan} is waiting on receiving a message ($\qrev{c}{x}.0$). 

\begin{figure}[t]
{\small
  \begin{mathpar}
\mprset{flushleft}
   \inferrule[GC]{}
       {\Cellb{\qsend{p}{c}{m}^t}{i}{c_1}\Cella{P}{j}{c_2} \qcell{\{(c_1,c_2,p')\}\cup R}{\cn{rel}}\qcell{t'}{\cn{gt}}
        \qcell{\emptyset}{\cn{comm}}\pcell{b}{\_}{\cn{pred}}
       \\\\\qquad\qquad \longrightarrow \Cellb{}{i-}{c_1}\Cella{\qsend{p*p'}{c}{m}^{t}\texttt{|}P}{j-}{c_2}
              \qcell{\{(c_1,c_2,p')\}\cup R}{\cn{rel}}\qcell{t'}{\cn{gt}}
               \qcell{(c_1,c_2,\qsend{p}{c}{m}^t)}{\cn{comm}}\pcell{b}{t'}{\cn{pred}}}

\ignore{
   \inferrule[GenQubit]{}
       {\Cella{P}{n,t'}{a}\qcell{t}{\cn{gt}}\longrightarrow \Cella{P}{n+,t}{a}\qcell{t}{\cn{gt}}}\;\;\texttt{when}\;t \texttt{|} \beta\wedge t' < t
}

   \inferrule[CT]{}
       {\Cellb{\qsend{p}{c}{m}}{i}{c_1}\qcell{t}{\cn{gt}} \longrightarrow \Cellb{\qsend{p}{c}{m}^t}{i}{c_1}\qcell{t}{\cn{gt}}}

   \inferrule[MT]{}
       {\comp{P} \longrightarrow \parl{P}{\comp{P}}}
      
   \inferrule[NT]{}
       {\comp{P} \longrightarrow 0}

  \inferrule[PC]{}
      { \Cellb{\qsend{p}{c}{m}^t\texttt{|} \qrev{c}{x}.P}{n}{c}\qcell{t'}{\cn{gt}}\qcell{\emptyset}{\cn{comm}}\pcell{b}{\_}{\cn{pred}}
           \longrightarrow
         \Cellb{P[m/x]}{n}{c}\qcell{t'}{\cn{gt}}\qcell{\qsend{p}{c}{m}^t}{\cn{comm}}\pcell{b}{t'}{\cn{pred}}}
                  

  \inferrule[Com]{}
      { \qcell{t'}{\cn{gt}}\qcell{\qsend{p}{c}{m}^t}{\cn{comm}}\pcell{\cn{true}}{t'}{\cn{pred}}
           \xrightarrow{p.c.m}  
         \qcell{t'+}{\cn{gt}}\qcell{\emptyset}{\cn{comm}}\pcell{\cn{false}}{t'}{\cn{pred}} } 

  \inferrule[FC]{}
      { \qcell{t}{\cn{gt}}\qcell{(c_1,c_2,A)}{\cn{comm}}\pcell{\cn{true}}{t}{\cn{pred}}
           \longrightarrow
         \qcell{t+}{\cn{gt}}\qcell{\emptyset}{\cn{comm}}\pcell{\cn{false}}{t}{\cn{pred}} } 

  \end{mathpar}
}
\caption{Quantum Pi Semantics. $\beta$, $\mu$, and $\nu$ are globally defined for the qubit generation period, the message threshold probability, and message sending finished threshold. $\Cella{P}{n}{a}$ refers to that the $t$ in $\Cella{P}{n,t}{a}$ is omitted in the rule.}
  \label{fig:q-pi-semantics}
\end{figure}
}

\ignore{
\begin{figure}[t]
{\small
{\hspace*{-2em}
\begin{tikzpicture}[align=center,node distance=1.5cm and -1cm, thick] 
\node (1) {S$\langle\{(a,r_1,0.5), (a,r_2,0.5)\}\cup$R$\rangle$}; 
\node (2) [below left= of 1] {$\Cella{0}{9}{a}$ $\Cella{\qsend{0.5}{c}{m}|0}{9}{r_1}$... $\langle\{(r_1,r_4)\}\cup$R$\rangle$}; 
\node (3) [below right= of 1] {\text{\ \ \ \ \ \ }$\Cella{0}{9}{a}$ $\Cella{\qsend{0.5}{c}{m}|0}{9}{r_2}$..., $\ccell{\{(r_2,r_3)\}\cup\text{R}}$}; 
\node (4) [below of=2] {$\Cella{0}{8}{r_1}$ $\Cella{\qsend{0.25}{c}{m}|0}{9}{r_4}$... $\langle\{(r_4,b)\}\cup$R$\rangle$};
\node (5) [below of=3] {\text{\ \ \ \ \ \ }$\Cella{0}{8}{r_2}$ $\Cella{\qsend{0.25}{c}{m}|0}{9}{r_3}$..., $\ccell{\{(r_3,r_4)\}\cup\text{R}}$};
\node (6) [below of=4] {$\Cella{0}{8}{r_4}$ $\Cella{\qsend{0.125}{c}{m}|\qrev{c}{x}.0}{9}{b}$... $\ccell{\text{R}}$};
\node (7) [below of=5] {\text{\ \ \ \ \ \ \ \ \;}$\Cella{0}{8}{r_3}$ $\Cella{\qsend{0.125}{c}{m}|0}{9}{r_4}$..., $\ccell{\{(r_4,b)\}\cup\text{R}}$};
\node (8) [below of=6] {$\Cella{0}{9}{b}$... $\ccell{\text{R}}$};
\node (9) [below of=7] {\text{\ \ \ \ \ \ \ \ }$\Cella{0}{8}{r_4}$ $\Cella{\qsend{0.0625}{c}{m}|\qrev{c}{x}.0}{9}{b}$..., $\ccell{\text{R}}$};
\node (10) [below of=9] {\text{\ \ \ \ \ \ }$\Cella{0}{n}{b}$... $\ccell{\text{R}}$};
\draw[->] (1) -- node[midway, above left] {} (2); 
\draw[->] (1) -- node[midway, above right] {} (3); 
\draw[->] (2) -- node[midway, right] {} (4); 
\draw[->] (4) -- node[midway, right] {} (6);
\draw[->] (6) -- node[midway, right] {$0.125.c.m$} (8); 
\draw[->] (3) -- node[midway, right] {} (5); 
\draw[->] (5) -- node[midway, right] {} (7); 
\draw[->] (7) -- node[midway, right] {} (9);
\draw[->] (9) -- node[midway, right] {$0.0625.c.m$}  (10); 
\end{tikzpicture} 
}
}
\caption{Quantum Machine Transitions for \Cref{fig:q-pi-example}}
  \label{fig:q-pi-example1}
\end{figure} 
}


\subsection{Adding Connectivity and Time Stamps} \label{sec:qamsemantics1}

\begin{figure}[t]
{\small
$\textcolor{blue}{\text{Extended Syntax:}}\\$
  \[\begin{array}{llcl} 
      \texttt{Time Predicate Function} & f & \in & \Ts \times \Ts \to \cn{bool} \\[0.2em]
      \texttt{Relation} & R & ::= & \overline{(c,c,p)}  \\[0.2em]
      \textcolor{red}{\texttt{Configuration Third Pattern}} & \textcolor{red}{C} & \textcolor{red}{::=} & 
\textcolor{red}{\overline{\varphi}\pcell{\alpha}{\beta}{\cn{comm}}
  \qcell{\cn{bool}}{\cn{pred}}\qcell{R}{\cn{rel}}\qcell{t}{\cn{gt}} }
    \end{array}
  \]

$\textcolor{blue}{\text{Semantics:}}\\$
  \begin{mathpar}
\mprset{flushleft}
   \inferrule[GC1]{}
       {\pcell{\pard{\qsend{p}{c}{m}^t.P}{...}}{\cn{S}\;i}{c_1}\pcell{ ...}{\cn{S}\;j}{c_2}
        \qcell{\emptyset}{\cn{comm}}
        \qcell{\pard{(c_1,c_2,p')}{...}}{\cn{rel}}
        \\\\ \qquad \longrightarrow \pcell{\pard{P}{...}}{i}{c_1}\pcell{\pard{\qsend{p'\cn{*}p'}{c}{m}^t}{...}}{j}{c_2}
               \pcell{(p \cn{*} p'.c.m)^t}{c_1\rightarrow c_2}{\cn{comm}}
        \qcell{\pard{(c_1,c_2,p')}{...}}{\cn{rel}}}

   \inferrule[CT]{}
       {\pcell{\pard{\qsend{p}{c}{m}.P}{...}}{i}{c_1}\qcell{t}{\cn{gt}} \longrightarrow 
             \pcell{\pard{\qsend{p}{c}{m}^t.P}{...}}{i}{c_1}\qcell{t}{\cn{gt}}}

\textcolor{blue}{
  \inferrule[PC]{\overline{P}\xrightarrow{\alpha^t} \overline{Q} }
      { \pcell{\overline{P}}{n}{c}\qcell{\emptyset}{\cn{comm}}
           \longrightarrow
         \pcell{\overline{Q}}{n}{c}\pcell{\alpha^t}{c}{\cn{comm}}}
}
               
   \inferrule[Grant1]{}
       {\qcell{\cn{F}}{\cn{pred}}\pcell{\alpha}{c_1\rightarrow c_2}{\cn{comm}}
             \longrightarrow \qcell{\cn{T}}{\cn{pred}}\pcell{\alpha}{c_1\rightarrow c_2}{\cn{comm}}}
  
   \inferrule[Grant2]{}
       {\pcell{\alpha^t}{c}{\cn{comm}}\qcell{\cn{F}}{\cn{pred}}\qcell{t'}{\cn{gt}} 
     \longrightarrow \pcell{\alpha^t}{c}{\cn{comm}}\qcell{f(t,t')}{\cn{pred}}\qcell{t'}{\cn{gt}}}

  \inferrule[FC1]{}
      { \pcell{\alpha}{c_1\rightarrow c_2}{\cn{comm}}\qcell{\cn{T}}{\cn{pred}}\qcell{t}{\cn{gt}}
           \longrightarrow \qcell{\emptyset}{\cn{comm}}\qcell{\cn{F}}{\cn{pred}}\qcell{t\cn{+}}{\cn{gt}} } 
 
  \inferrule[AC1]{}
      { \pcell{\alpha^t}{c}{\cn{comm}}\qcell{\cn{T}}{\cn{pred}}\qcell{t}{\cn{gt}}
           \xrightarrow{\alpha}  \qcell{\emptyset}{\cn{comm}}\qcell{\cn{F}}{\cn{pred}}\qcell{t\cn{+}}{\cn{gt}} } 

  \end{mathpar}
}
\caption{Extended inter-destination communication syntax and semantics. $\beta$ can be omitted when being $\emptyset$. $\alpha^t$ is an action such that $\alpha$ has the form $p.c.m$. $\qsend{p}{c}{m}^t\equiv \qsend{p}{c}{m}^t.0$.}
  \label{fig:q-pi-semantics3}
\end{figure}

\begin{figure}[t]
{\small
\begin{center}
\begin{tikzpicture}[node distance={1cm}, thick, main/.style = {draw, circle}] 
\node[main] (1) {\cn{Ann}}; 
\node[main] (2) [right of=1] {$r_1$};
\node[main] (3) [right of=2] {$r_4$};
\node[main] (4) [right of=3] {\cn{Bob}};
\node[main] (5) [above right=0.5cm and 1.5cm of 1] {$r_2$};
\node[main] (6) [right of=5] {$r_3$};
\node[main] (7) [below of=1] {\cn{Cat}};
\node[main] (8) [below of=3] {\cn{Dan}};
\draw[-] (1) --  (2);
\draw[-] (2) --  (3);
\draw[-] (3) --  (4);
\draw[-] (1) --  (5);
\draw[-] (5) --  (6);
\draw[-] (6) --  (3);
\draw[-] (7) --  (2);
\draw[-] (2) --  (8);
\end{tikzpicture}
\end{center}
}
\caption{Example Path Connectivity \liyi{Le: this looks bad. I think a better diagram is to have only \cn{Cat} and \cn{Dan} through a directed graph, with success rate marking. }}
  \label{fig:examplepath}
\end{figure}

As in \Cref{sec:qamsyntax1}, QAM permits a quantum protocol semantics through the manipulation of configurations and rules describing the communicating, granting, and cleaning steps for framing quantum network communications.
Here, we extend the configuration and rules in \Cref{fig:q-pi-semantics2} to define two important properties appearing in almost all quantum network protocols: connectivity and time.
The former refers to that quantum messages are always transmitted based on a connectivity graph, i.e., two cells can communicate if they have an edge in the graph and each edge has a certain success rate of message transmission. One example connectivity is in \Cref{fig:examplepath}.
The latter means that quantum messages are time-sensitive and are required to deliver in a short lifetime; otherwise, the messages are decohered and lost.

To define the two properties, we extend the configuration with the \cn{gt} and \cn{rel} cells in \Cref{fig:q-pi-semantics3},
which respectively contain a global clock time $t$ and a multiset of relation triples $R$ whose element is a triple of source and target cell locations and the success rate to transmit a message from the source to the target.
A user defined validity check function $f$ is used to determine if a message spends too much time in its transmission.
\Cref{fig:q-pi-semantics3} shows the semantic rule definitions for the two properties. Rule \rulelab{PC} is marked blue because it keeps the same as the case in \Cref{fig:q-pi-semantics2}, while the other rules require some modifications, explained below.
 
\noindent\textbf{Defining connectivity.}
Definition the property requires the update of rule \rulelab{GC} to \rulelab{GC1} in \Cref{fig:q-pi-semantics3}
by including the extra \cn{rel} cell.
In each message ($\qsend{p}{c}{m}^t$) transmission, we pattern match the source $c_1$ and target $c_1$ locations in the \cn{rel} cell as $(c_1,c_2,p')$. After the message is transmitted, we reduce its success rate as $p\cn{*}p'$, referring to that the message transmission results in a success rate reduction.

\noindent\textbf{Defining time.}
Definition this property is essentially two procedures.
First, a global time clock is introduced in the cell \cn{gt}.
In the upgraded rules (\rulelab{FC1} and \rulelab{AC1}) for cleaning steps, we include the \cn{gt} cell and increments the clock ($t\cn{+}$). We also include a new rule \rulelab{CT} to generate time stamp label for a message, so that we can mark the sending time of the message correctly. Second, we need to define a new granting rule by applying the validity check function $f$ to verify if a message's delivery is past due. To do so, we split the granting rule (\rulelab{Grant} in \Cref{fig:q-pi-semantics2}) into two rules.
For the case of message transmission (flag $c_1 \rightarrow c_2$ in \cn{comm}), it keeps the same;
otherwise, we apply function $f$ on a pair of time stamps, indicating the starting time of the message ($t$) and the final delivery time ($t'$), respectively. Function $f$ is user-defined. One example definition could be a lambda abstraction $\lambda\,(t,t')\,.\,t'-t<m$, where $m$ is a threshold time period for a message guaranteeing to deliver.

\begin{figure}[t]
{\footnotesize
\[\hspace*{-1em}
\begin{array}{lll}
&
\pcellb{\qsend{1}{c}{m}.0}{10}{\cn{Cat}}
\pcellb{0}{10}{r_1}
\pcellb{\qrev{c}{x}.0}{10}{\cn{Dan}} 
\qcellb{\emptyset}{\cn{comm}}
\qcellb{\texttt{F}}{\cn{pred}}
\qcellb{R}{\cn{rel}}
\qcellb{0}{\cn{gt}}
&
\\[0.3em]
\longrightarrow
&
\pcellb{\qsend{1}{c}{m}^0.0}{10}{\cn{Cat}}
\pcellb{0}{10}{r_1}
\pcellb{\qrev{c}{x}.0}{10}{\cn{Dan}} 
\qcellb{\emptyset}{\cn{comm}}
\qcellb{\texttt{F}}{\cn{pred}}
\qcellb{R}{\cn{rel}}
\qcellb{0}{\cn{gt}}
&
(\rulelab{CT})
\\[0.3em]
\longrightarrow
&
\pcellb{0}{9}{\cn{Cat}}
\pcellb{\qsend{0.5}{c}{m}^0.0}{9}{r_1}
\pcellb{\qrev{c}{x}.0}{10}{\cn{Dan}} 
\pcellb{(1.c.m)^0}{\cn{Cat}\rightarrow r_1}{\cn{comm}}
\qcellb{\texttt{F}}{\cn{pred}}
\qcellb{R}{\cn{rel}}
\qcellb{0}{\cn{gt}}
&
(\rulelab{GC1})
\\[0.3em]
\longrightarrow
&
\pcellb{0}{9}{\cn{Cat}}
\pcellb{\qsend{0.5}{c}{m}^0.0}{9}{r_1}
\pcellb{\qrev{c}{x}.0}{10}{\cn{Dan}} 
\pcellb{(1.c.m)^0}{\cn{Cat}\rightarrow r_1}{\cn{comm}}
\qcellb{\texttt{T}}{\cn{pred}}
\qcellb{R}{\cn{rel}}
\qcellb{0}{\cn{gt}}
&
(\rulelab{Grant1})
\\[0.3em]
\longrightarrow
&
\pcellb{0}{9}{\cn{Cat}}
\pcellb{\qsend{0.5}{c}{m}^0.0}{9}{r_1}
\pcellb{\qrev{c}{x}.0}{10}{\cn{Dan}} 
\qcellb{\emptyset}{\cn{comm}}
\qcellb{\texttt{F}}{\cn{pred}}
\qcellb{R}{\cn{rel}}
\qcellb{1}{\cn{gt}}
&
(\rulelab{FC1})
\\[0.3em]
\longrightarrow ...
\\[0.3em]
\longrightarrow
&
\pcellb{0}{9}{\cn{Cat}}
\pcellb{0}{8}{r_1}
\pcellb{\pard{\qsend{0.25}{c}{m}^0.0}{\qrev{c}{x}.0}}{9}{\cn{Dan}} 
\qcellb{\emptyset}{\cn{comm}}
\qcellb{\texttt{F}}{\cn{pred}}
\qcellb{R}{\cn{rel}}
\qcellb{2}{\cn{gt}}
&
\\[0.3em]
\longrightarrow
&
\pcellb{0}{9}{\cn{Cat}}
\pcellb{0}{8}{r_1}
\pcellb{0}{9}{\cn{Dan}} 
\pcellb{(0.25.c.m)^0}{\cn{Dan}}{\cn{comm}}
\qcellb{\texttt{F}}{\cn{pred}}
\qcellb{R}{\cn{rel}}
\qcellb{2}{\cn{gt}}
&
(\rulelab{Com}+\rulelab{PC})
\\[0.3em]
\longrightarrow
&
\pcellb{0}{9}{\cn{Cat}}
\pcellb{0}{8}{r_1}
\pcellb{0}{9}{\cn{Dan}} 
\pcellb{(0.25.c.m)^0}{\cn{Dan}}{\cn{comm}}
\qcellb{f(0,2)}{\cn{pred}}
\qcellb{R}{\cn{rel}}
\qcellb{2}{\cn{gt}}
&
(\rulelab{Grant2})
\\[0.3em]
\xrightarrow{0.25.c.m}&
\pcellb{0}{9}{\cn{Cat}}
\pcellb{0}{8}{r_1}
\pcellb{0}{9}{\cn{Dan}} 
\qcellb{\emptyset}{\cn{comm}}
\qcellb{\texttt{F}}{\cn{pred}}
\qcellb{R}{\cn{rel}}
\qcellb{3}{\cn{gt}}
&
(\rulelab{AC1})
\end{array}
\]
}
{\footnotesize
\begin{center}
$R\triangleq\{(\cn{Cat},r_1,0.5), (r_1,\cn{Dan},0.5)\}$
\qquad
$f=\lambda\,(t,t')\,.\,t'-t<5$
\end{center}
}
\caption{Example transitions based on the model in \Cref{fig:q-pi-semantics3}.}
  \label{fig:q-pi-example1}
\end{figure}


\Cref{fig:q-pi-example1} shows an example evaluation of the initial configuration on the top,
which is an instantiation of metavariables in configuration pattern in \Cref{fig:q-pi-semantics3}.
We first apply rule \rulelab{CT} to generate time stamp $0$ for the sending process action $\qsend{0.25}{c}{m}$.
We then apply rule \rulelab{GC1} and the two consecutive rules to transmit the action to cell $r_1$.
We can perform rule \rulelab{GC1} because the triple $(\cn{Cat},r_1,0.5)$ is in $R$.
During the steps, a piece of qubit resouce is consumed in both \cn{Cat} and $r_1$ cells, the action's success rate is reduced to $0.5$, and the global clock is updated to $1$.
After another round of message transmission steps, the action is conveyed to cell \cn{Dan}.
We then use rule \rulelab{PC}, with rule \rulelab{Com}, to deliver the action.
with the rule applications \rulelab{Grant2} and \rulelab{AC1} to grant and clean the delivery.
The delivery is granted, because the validity check results in $f(0,2)=2-0<5=\cn{T}$.

On the other hand, if users set the validity check function to be $\lambda\,(t,t')\,.\,t'-t<2$,
the validity check here results in $f(0,2)=2-0<2=\cn{F}$, which means that the evaluation is stuck at the \rulelab{Grant2} step.








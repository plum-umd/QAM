\section{The Quantum Abstract Machine:  Syntax and Semantics} \label{sec:qam}

Essentially, quantum network protocols describe communication procedures 
among different quantum computers via the connections through classical network devices. 
In quantum teleportation (\Cref{sec:background}), there are two main procedures: 
1) a quantum message is transformed from Alice to Bob, 
and 2) a quantum entangled channel is built between Alice and Bob and causes Alice and Bob a qubit each to transform the message.
In a practical quantum network protocol, such as QCast and QPass \cite{10.1145/3387514.3405853}, 
there is another procedure, operated by classical computers, to determine how an optimized quantum channel is established to send a message from Alice to Bob.
If Alice and Bob is close enough to establish a direct entangled channel, the situation is the same as quantum teleportation;
otherwise, there is a complicated entangled qubit group that connects Alice and Bob as well as all the intermediate router stations,
so that the message can be sent from Alice to Bob. During the process, $n$ intermediate router stations cause $2n$ additional channels to establish the channel.

\begin{figure}[t]
{\small
\begin{center}
\begin{tikzpicture}[node distance={1cm}, thick, main/.style = {draw, circle}] 
\node[main] (1) {$a$}; 
\node[main] (2) [right of=1] {$r_1$};
\node[main] (3) [right of=2] {$r_4$};
\node[main] (4) [right of=3] {$b$};
\node[main] (5) [above right=0.5cm and 1.5cm of 1] {$r_2$};
\node[main] (6) [right of=5] {$r_3$};
\draw[-] (1) --  (2);
\draw[-] (2) --  (3);
\draw[-] (3) --  (4);
\draw[-] (1) --  (5);
\draw[-] (5) --  (6);
\draw[-] (6) --  (3);
\end{tikzpicture}
\end{center}
}
\caption{QPass Example Path Connectivity}
  \label{fig:q-pi-example}
\end{figure}

We model these quantum network protocols via an abstract machine framework that represents the above procedures.
We use a cell structure $\qcell{}$ to represent a conceptual device involving in the protocol communication.
For example, $\qcell{P}^{n}_{c}$ is a device named $c$ with $n$ qubit recourse and its program content is $P$.
In the section, we introduce the quantum abstract machine (QAM) syntax and semantics, following by the trace refinement definition for QAM processes. We use quantum teleportation, QCast and QPass as running examples.
QPass is a hybrid classical quantum network protocol. It contains a static classically generated graph describing the connectivity of different parties involved in a communication.
In \Cref{fig:q-pi-example}, a message is sent from node $a$ to $b$, and nodes $r_1$ to $r_4$ are intermediate routers.
A message is sent via either the path $a$, $r_1$, $r_4$, and $b$, or $a$, $r_2$, $r_3$, $r_4$ and $b$.

\begin{figure}[t]
{\small
  \[\begin{array}{llcl} 
      \texttt{Variables} & x,y \\
      \texttt{Probability} & p\\
      \texttt{Messages} & m\\
    \texttt{Channels} & c\\
    \texttt{Global Time} & t\\
    \texttt{Actions} & a & ::= & c \times m\\
      \texttt{Singleton Process} & A & ::= & \emptyset \mid \qsend{p}{c}{m} \mid \qrev{c}{x} \\
      \texttt{Process} & P,Q & ::= & 0 \mid A.P \mid \parl{P}{Q} \mid \comp{P} \\
      \texttt{Process Cell} & \varphi & ::= & \pcell{P}{n}{c} \\
      \texttt{Relations} & R & ::= & [c \times c \times p] \\
      \texttt{Relation Functions} & \Fs & ::= & t \to R\\
    \texttt{Action Probabilities} & \rho & ::=& a \to p\\
      \texttt{Program Configuration} & \Phi & ::= & \varphi^* \qcell{R}_{\textsf{r}}\qcell{a}_{\textsf{comm}}\\
      \texttt{Configuration} & C & ::= & \Phi\qcell{\Fs}_{\textsf{fun}}\qcell{t}_{\textsf{gt}}\qcell{\rho}_{\textsf{at}}\qcell{p}_{\textsf{s}}
    \end{array}
  \]
}
\caption{Quantum Pi Syntax}
  \label{fig:q-pi-syntax}
\end{figure}


\subsection{Syntax} \label{sec:qamsyntax}

As shown in \Cref{fig:q-pi-syntax}, the execution of a QAM program relies on the concept of 
a configuration that describes conceptual computer components for QAM program interactions.
A configuration can possibly contain many cell structures $\qcell{\Theta}_{\eta}$, where $\Theta$ is the content of $\eta$ is the name.
For example, $\qcell{\Ls}_{p}$ is a program execution cell with name $p$, and program execution configuration as its content $\Ls$,
Cell $\qcell{\Fs}_{f}$ holds a function $\Fs$ representing the dynamically updated network node connectivity with respect to time, and $\qcell{\tau}_{t}$ is a global time cell representing periods of time for updating the connectivity graph.

Program executions represent multiple communications of quantum messages through different nodes in a network.
We assume that a single message communication always happens presently within a global time period so that during a message communication, the global clock remains the same.
A program execution configuration acts as the content $\Ls$ of the program execution cell $\qcell{\Ls}_{p}$ with the syntax $\Ss \qcell{\Rs}_r$, where cell $\qcell{\Rs}_r$ stores the node connectivity at a time, $\Ss$ is a set of cells representing different nodes.
The content $\Rs$ in a relation cell contains relations defining the connectivity of computer locations.
Each relation triple defines an adjacent pair of two locations, i.e., 
a computer can send a message to another location only if the two locations are adjacent.
There is also a probability value defining the probability of the successful rate of the communication between the two locations.

In a node cell, $c$ represents the node name and $n$ represents the number of qubit resources in the node.
$A (\texttt{|} P )^?$ is a Pi-calculus lite process language representing the communication program content in the node.
$A$ represents a message action that is about to send and receive from the node. $P$ represents a possible local process for the node to execute after it receives a message.
An action is either a message send ($\qsend{p}{c}{m}$) or receipt ($\qrev{c}{x}$) operation,
with $c$ being a channel name and $m$ is a message. $p$ is the likelihood of the send operation.
A local process consists with a chain of actions ($A.P$) that ends at a terminated process ($0$).
If a node cell only contains a singleton process,
it means that the cell is not ready for communication and the actions in the process will be transmitted to other cells.

{\small
\[
\begin{array}{l}
\texttt{S} \qcell{\texttt{R}}_r \equiv 
\Cella{\qsend{1}{c}{m}|0}{10}{a}\Cella{0}{10}{r_1}\Cella{0}{10}{r_2}\Cella{0}{10}{r_3}\Cella{0}{10}{r_4}\Cella{\qrev{c}{m}.0}{10}{b} 
\\\qquad\qquad
\qcell{(a,r_1,0.5), (a,r_2,0.5), (r_1, r_4, 0.5), (r_2, r_3, 0.5), (r_3,r_4, 0.5), (b,r_4,0.5)} 
\end{array}
\]
}

As an example of the QAM syntax, the above program execution configuration defines the program to send a message from location $a$ to $b$.
The first line corresponds to node cells $\texttt{S}$, containing the target and destination nodes $a$ and $b$ as well as intermediate routers $r_1$ to $r_4$; while the second line corresponds to relation cell $\qcell{\texttt{R}}_r$ capturing the connectivity in \Cref{fig:q-pi-example}.

\begin{figure}[t]
{\small
  \begin{mathpar}
   
   \inferrule[GenChan]{}
       {\Cellb{\qsend{p}{c}{m}}{i}{a}\Cella{P}{j}{b} \qcell{\{(a,b,p')\}\cup R})
        \longrightarrow (\Cellb{}{i-}{a}\Cella{\qsend{(p\cdot p')}{c}{m}\texttt{|}P}{j-}{b},\{(a,b)\}\cup R)}


   \inferrule[GenQubit]{}
       {\Cella{P}{n}{a}\longrightarrow \Cella{P}{n+}{a}}

   \inferrule[MoreTries]{}
       {\comp{P} \longrightarrow \parl{P}{\comp{P}}}
      
   \inferrule[NoTries]{}
       {\comp{P} \longrightarrow 0}

  \inferrule[Communication]{}
      { \qcell{\qsend{p}{c}{m}\texttt{|} \qrev{c}{x}.P}^n_c\qcell{\emptyset}_{\textsf{comm}}
           \longrightarrow  
         \qcell{P[m/x]}^n_c\qcell{(p.c.m)}_{\textsf{comm}}}
    
  \inferrule[topup]{}
      { \qcell{(p.c.m)}_{\textsf{comm}}\qcell{\rho}_{\textsf{at}}\qcell{p'}_{\textsf{s}} \xrightarrow{c.m} 
        \qcell{\emptyset}_{\textsf{comm}}\qcell{\rho[c.m\mapsto \rho(c.m)\odot p]}_{\textsf{at}}\qcell{p'}_{\textsf{s}}}\;\;\texttt{when}\;\rho(c.m)\odot p<p'

  \inferrule[RelationUp]{}
      { \qcell{R}_{\textsf{r}}\qcell{\Fs}_{\textsf{fun}}\qcell{t}_{\textsf{gt}} \longrightarrow
            \qcell{\Fs(t)}_{\textsf{r}}\qcell{\Fs}_{\textsf{fun}}\qcell{t}_{\textsf{gt}}}

  \inferrule[TimeUp]{}
      {\qcell{t}_{\textsf{gt}} \longrightarrow \qcell{t+}_{\textsf{gt}}}

  \end{mathpar}
}
\caption{Quantum Pi Semantics}
  \label{fig:q-pi-semantics}
\end{figure}


\begin{figure}[t]
{\small
{\hspace*{-2em}
\begin{tikzpicture}[align=center,node distance=1.5cm and -1cm, thick] 
\node (1) {S$\langle\{(a,r_1,0.5), (a,r_2,0.5)\}\cup$R$\rangle$}; 
\node (2) [below left= of 1] {$\Cella{0}{9}{a}$ $\Cella{\qsend{0.5}{c}{m}|0}{9}{r_1}$... $\langle\{(r_1,r_4)\}\cup$R$\rangle$}; 
\node (3) [below right= of 1] {\text{\ \ \ \ \ \ }$\Cella{0}{9}{a}$ $\Cella{\qsend{0.5}{c}{m}|0}{9}{r_2}$..., $\qcell{\{(r_2,r_3)\}\cup\text{R}}$}; 
\node (4) [below of=2] {$\Cella{0}{8}{r_1}$ $\Cella{\qsend{0.25}{c}{m}|0}{9}{r_4}$... $\langle\{(r_4,b)\}\cup$R$\rangle$};
\node (5) [below of=3] {\text{\ \ \ \ \ \ }$\Cella{0}{8}{r_2}$ $\Cella{\qsend{0.25}{c}{m}|0}{9}{r_3}$..., $\qcell{\{(r_3,r_4)\}\cup\text{R}}$};
\node (6) [below of=4] {$\Cella{0}{8}{r_4}$ $\Cella{\qsend{0.125}{c}{m}|\qrev{c}{m}.0}{9}{b}$... $\qcell{\text{R}}$};
\node (7) [below of=5] {\text{\ \ \ \ \ \ \ \ \;}$\Cella{0}{8}{r_3}$ $\Cella{\qsend{0.125}{c}{m}|0}{9}{r_4}$..., $\qcell{\{(r_4,b)\}\cup\text{R}}$};
\node (8) [below of=6] {$\Cella{0}{9}{b}$... $\qcell{\text{R}}$};
\node (9) [below of=7] {\text{\ \ \ \ \ \ \ \ }$\Cella{0}{8}{r_4}$ $\Cella{\qsend{0.0625}{c}{m}|\qrev{c}{m}.0}{9}{b}$..., $\qcell{\text{R}}$};
\node (10) [below of=9] {\text{\ \ \ \ \ \ }$\Cella{0}{n}{b}$... $\qcell{\text{R}}$};
\draw[->] (1) -- node[midway, above left] {} (2); 
\draw[->] (1) -- node[midway, above right] {} (3); 
\draw[->] (2) -- node[midway, right] {} (4); 
\draw[->] (4) -- node[midway, right] {} (6);
\draw[->] (6) -- node[midway, right] {$0.125.c.m$} (8); 
\draw[->] (3) -- node[midway, right] {} (5); 
\draw[->] (5) -- node[midway, right] {} (7); 
\draw[->] (7) -- node[midway, right] {} (9);
\draw[->] (9) -- node[midway, right] {$0.0625.c.m$}  (10); 
\end{tikzpicture} 
}
}
\caption{Quantum Machine Transitions for \Cref{fig:q-pi-example}}
  \label{fig:q-pi-example1}
\end{figure} 

\subsection{Semantics} \label{sec:qamsemantics}

The QAM semantics is given as a process algebra style labeled transition system listed in \Cref{fig:q-pi-semantics}.
Each transition label has either an empty label or a communication action on label $p.c.m$, where $p$ is the probability the action $c.m$ happens and $c$ is the channel name and $m$ is the actual message potentially being quantum.
Rule \rulelab{RelationUp} describes the global transitions where program execution connectivity graph is updated once upon a time,
while rule \rulelab{TimeUp} updates the global clock of the configuration.
In QAM, the semantic rules only need to mention the cells that are involved in a transition.
For example, in rule \rulelab{TimeUp}, we do not mention the cells $\qcell{\Ls}_{p}$ and $\qcell{\Fs}_{f}$ because they have no changes during the transition.

Rules \rulelab{GenChan}, \rulelab{GenQubit}, and \rulelab{Communication} describe the transitions happened in a program configuration.
Rule \rulelab{GenChan} represents the procedure that a message is transmitted from one location to another via a quantum channel. During the procedure, the probability for the message being successfully sent is reduced and the two locations also consume two qubits.
Here, an action $\qsend{p}{c}{m}$ is transmitted from one location, defined by the cell $\qcell{}_a$, to another location $\qcell{}_b$, if there is a triple relation $(a,b,p')$ in the relation cell. 
The $p'$ value represents the reduced probability factor to communicate the message between the two locations, i.e.,
the probability value $p$ in action $\qsend{p}{c}{m}$ is accumulated to become $p\cdot p'$.
Each transmission requires the consumption of two qubits: each cell consumes one qubits.
This is why $a$ and $b$'s result cells ($\Cellb{}{i-}{a}$ and $\Cellb{}{j-}{b}$) have the qubit value $i$ and $j$ being reduced, respectively.
Here, the $...$ in cell $\Cellb{}{i-}{a}$ represents that we only care about the left hand side action in a process cell regarding the rest of the content.
In the example transition in \Cref{fig:q-pi-example1} that starts from the configuration $\texttt{S} \qcell{\texttt{R}}_r$ above,
the transitions from the top to the left and right are both applied a \rulelab{GenChan} rule.
On the left hand side, the action $c!<m>$ is transmitted from cell $a$ to cell $r_1$, and the probability value is reduced to $0.5$.
Rule \rulelab{GenQubit} generates a new qubit in a location to help transmit messages from one location to another.

Rule \rulelab{Communication} defines the behavior of consuming a message by a local process in a node cell.
In the parallel composition $\qsend{p}{c}{m}\texttt{|} \qrev{c}{x}.P$, action $\qsend{p}{c}{m}$ is waiting to be consumed on channel $c$, 
$\qrev{c}{x}$ is a receipt binding of channel $c$, and $P$ is the following process to consume the action.
After the transition, the waiting action is removed, and the remaining process $P$ is executed by substituting variable $x$ with the massage $m$. The transition is a non-empty one and the parallel composition communicates on the action $p.c.m$.

An example machine is given in \Cref{fig:q-pi-example} and \Cref{fig:q-pi-example1}.
The protocol of the definition follows the QPASS model \cite{10.1145/3387514.3405853}. 
The example contains six distinct computer locations and we want to send messages from location $a$ to $b$.
Locations $r_1$, $r_2$, $r_3$, and $r_4$ are intermediate routers. 
\Cref{fig:q-pi-example} represents the connectivity of different computers. 
\Cref{fig:q-pi-example1} provides the transition computation tree of two possible ways of sending a message from $a$ to $b$.
At the end of each path in the computation tree in \Cref{fig:q-pi-example1}, the left hand path communicates an action $0.125.c.m$, while the right hand path communicates an action $0.0625.c.m$; thus, the left hand path is selected according to the QPass model because it has a higher successful rate to send out the message.




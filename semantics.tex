\section{The Quantum Abstract Machine:  Syntax and Semantics} \label{sec:qam}

This section describes QAM's syntax and semantics through the QPass example \cite{10.1145/3387514.3405853}.

\begin{figure}[t]
{\small
\begin{center}
\begin{tikzpicture}[node distance={1cm}, thick, main/.style = {draw, circle}] 
\node[main] (1) {\cn{Ann}}; 
\node[main] (2) [right of=1] {$r_1$};
\node[main] (3) [right of=2] {$r_4$};
\node[main] (4) [right of=3] {\cn{Bob}};
\node[main] (5) [above right=0.5cm and 1.5cm of 1] {$r_2$};
\node[main] (6) [right of=5] {$r_3$};
\node[main] (7) [below of=1] {\cn{Cat}};
\node[main] (8) [below of=3] {\cn{Dan}};
\draw[-] (1) --  (2);
\draw[-] (2) --  (3);
\draw[-] (3) --  (4);
\draw[-] (1) --  (5);
\draw[-] (5) --  (6);
\draw[-] (6) --  (3);
\draw[-] (7) --  (2);
\draw[-] (2) --  (8);
\end{tikzpicture}
\end{center}
}
\caption{QPass Example Path Connectivity}
  \label{fig:q-pi-example}
\end{figure}

\subsection{Capturing the Quantum Network Behaviors} \label{sec:qamexample}

As we mentioned in \Cref{sec:background}, all quantum network protocols are based on the physical procedures of quantum teleportation and quantum swap operations and their physical phenomena and mathematical semantics have been well studied.
On the other hand, building an evaluation framework for quantum network protocols does not require us to implement the complete physical procedure of individual operations; 
instead, we intend to capture the operational behaviors that affect the evaluation factors,
including the success rates and probabilities of sending messages and the required resources.
In fact, both quantum teleportation and swap can be summarized as a procedure to communicate a qubit between two parties with the cost of two qubits.

Here, we investigate the modeling of message transmission in QAM by the QPass protocol \cite{10.1145/3387514.3405853}.
\Cref{fig:q-pi-example} shows an example connectivity diagram such that party $\cn{Ann}$ and $\cn{Cat}$ send messages to $\cn{Bob}$ and $\cn{Dan}$, via the middle routers $r_1$ to $r_4$. There are two paths for \cn{Ann} to send to \cn{Bob} ($r_2 - r_4$ and $r_1 - r_4$), while there is only one path to send message from \cn{Cat} to \cn{Dan} ($r_1$). 

In QAM, sending a message $a$ is viewed as a sequence of transmissions of $a$'s ownership from a node to its adjacent node.
For example, assume that we want to send message $a$ from \cn{Cat} to \cn{Dan}.
We transmit $a$ from \cn{Cat} to $r_1$ and from $r_1$ to \cn{Dan}, 
each of which costs the two participated nodes one qubit each.
There is also a probability reduction in guaranteeing that $a$ to be delivered to \cn{Dan}:
the more ownership transmissions, the lower probability that $a$ is delivered.
For example, the success rate of the transmission from \cn{Cat} to $r_1$ and from $r_1$ to \cn{Dan} might be user defined probability values $p$ and $p'$, so the success rate to send a message from \cn{Cat} to \cn{Dan} is $p * p'$.
For simplicity, in the example in \Cref{fig:q-pi-example}, we assume that each station transmission reduces the success rate by $50\%$,
i.e., if a message is transmitted from $a$ to $r_1$, it has $50\%$ chance that the message is lost.

In QAM, sending a message can be viewed as a sequence of transitions as $\to^{*}\xrightarrow{a}$,
where we might first execute several internal communication ($\to^{*}$, referring to $\tau$ transitions in classical process algebra),
with a final external communication $\xrightarrow{a}$, meaning that two parties are communicated through a message $a$.
When running a QAM program, there might be different ways where it can transit, which forms different sequences of transmissions that represent different possible ways of communication. It is also possible that some sequences only contain the internal communication part without the final external communication, which represents failure tries that there are some bottlenecks preventing the communication moving forward, because we want to allow different parties send messages via the same network at the same time period; however, this might cause message transition bottlenecks due to limited resources in different nodes.

Since quantum qubit information has lifetime, it might cause a message $a$ to lost completely if we simply queue a message at a bottleneck location until enough resources are available. Most quantum network protocols have the mechanism to bypass the message sending to another route. For example, assume that \cn{Ann} and \cn{Cat}'s messages are transmitted to $r_1$ at the same time, and $r_1$ only has a qubit resource and decide to send \cn{Cat}'s message first. Thus, \cn{Ann}'s message is queued because $r_1$ has no more resource. Then, after the life time of the \cn{Ann}'s message ends, the message sending sequence is classified as a failure try that does not have an ending external communication. To successfully send both messages, we need \cn{Ann}'s message to be bypassed through the $r_2 - r_4$ route, so that the bottleneck in node $r_1$ is resolved and both messages can be delivered on time.

In the next two sections, we introduce the modeling of quantum network protocols via QAM formally.

\begin{figure}[t]
{\small
  \[\begin{array}{llcl} 
      \texttt{Variable} & x,y \\
      \texttt{Probability} & p &\in &\Rs\\
      \texttt{Message} & m &\in& \mathbb{M}\\
    \texttt{Channel} & c &\in& \Ls\\
    \texttt{Time Stamp} & t &\in& \mathbb{T}\\
    \texttt{Label} & a &\in& \Ls \times \mathbb{M}\\
      \texttt{Singleton Action} & A & ::= & \qsend{p}{c}{m}^t \mid \qrev{c}{x} \\
      \texttt{Process} & P,Q & ::= & 0 \mid A.P \mid \parl{P}{Q} \mid \comp{P} \\
      \texttt{Relations} & R & \in & \Ls \times \Ls \times \Rs \\
      \texttt{Objects} & \Os \\
      \texttt{Custom Cell} & \varphi,\psi & ::= & \pcell{P}{n}{c} \mid \qcell{\Os}{c}\\
      \texttt{Configuration} & C & ::= & \varphi^* \qcell{R}{\cn{rel}} \qcell{t}{\cn{gt}}
                             \pcell{A}{t}{\cn{comm}} \qcell{\cn{bool}}{\cn{pred}} \psi^* \\
      \texttt{Rules} & \rules & ::= & C \longrightarrow C \\
      \texttt{Meta Function} & \Cs & ::= & \decl{\rules}{\Cs} \mid C\\
    \end{array}
  \]
}
\caption{Quantum Pi Syntax}
  \label{fig:q-pi-syntax}
\end{figure}


\subsection{Syntax} \label{sec:qamsyntax}

QAM describes multiparty communication behaviors,
whose syntax is similarity to the chemistry abstract machine \cite{BERRY1992217} and $\Pi$-calculus.
The message label that is used to communicate different parties has the form $c.m$, 
where $c$ refers to a communication channel, and $m$ is a message, possibly a quantum state.
Each single process action can be a message sending $\qsend{p}{c}{m}^t$, meaning that a message $m$ is sent through the channel $c$ with the success probability rate $p$ initialized at the time stamp $t$, and a message receipt $\qrev{c}{x}$, referring to that a message is received through the channel $c$ and represented as variable $x$.

To describe the behaviors of a network system, every QAM system comes with a configuration $C$
containing two different kinds of cells --- process and content cells.
A configuration might contain one or more process cells $\pcell{P}{n}{c}$, representing the execution maltreated process of a local node with name $c$, such as the $\cn{Ann}$ and $r_1$ nodes in \Cref{fig:q-pi-example},
and $n$ is the available qubit resource in the node.
A multi-threaded process $P$ is the program actions that node $c$ can perform, which has similar syntax as $\Pi$-calculus.
Each process might contain multiple threads competing for executions, which are separated by a parallel operation $\cn{|}$.
Each single thread is a sequence of singleton actions that ends at the unit process $0$,
and $\comp{P}$ is a replication process that repeatedly executes the process $P$, 
which is used for resending messages, explained in \Cref{sec:qamsemantics}.

In QAM, a parallel process of a message sending and receipt, $\parl{\qact{\qsend{p}{c}{m}^t}{P}}{\qact{\qrev{c}{x}}{Q}}$,
causes the process to transit to $\parl{P}{Q[m/x]}$, which is similar to the $\Pi$-calculus semantics.
On the other hand, a parallel of two message sending, $\parl{\qact{\qsend{p}{c}{m}^t}{P}}{\qsend{p'}{c'}{m'}^{t'}{P'}}$,
refers to that the two operations are competing to convey the message outside of the node, possibly to other destinations.

The content cells act as control units for storing environment information for executing a QAM system.
Users are able to define their own content cell by instantiating object types $\Os$ to different user-defined types.
In QAM, we require at least four kinds of content cells: 
1) $\qcell{R}{\cn{rel}}$ is a relation cell defining the probability value to send out one message from a node to the other;
2) $\qcell{t}{\cn{gt}}$ is a global clock cell storing the global time stamp;
3) $\pcell{A}{t}{\cn{comm}}$ is a communication cell storing the message about to communicate at time $t$;
and 4) $\qcell{\cn{bool}}{\cn{pred}}$ is a predicate cell 
determining if a message communication can happen ($\texttt{true}$ in the cell) or not ($\texttt{false}$ in the cell).

{\footnotesize
\[
\Cella{\qsend{1}{c}{m}^0}{10}{\cn{Cat}}\Cella{0}{10}{r_1}
\Cella{\qrev{c}{m}.0}{10}{\cn{Dan}} 
\qcell{\{(\cn{Cat},r_1,0.5), (r_1,\cn{Dan},0.5)\}}{\cn{rel}}
\qcell{\emptyset}{\cn{comm}}
\qcell{\texttt{true}}{\cn{pred}}
\qcell{0}{\cn{gt}}
\qcell{\texttt{fun}\;a\to 0}{\cn{at}}
\]
}

As an example of the QAM syntax, the above configuration defines the initial program state for sending a message $c.m$ from $\cn{Cat}$ to $\cn{Dan}$ via the router $r_1$, as part of the communication in \Cref{fig:q-pi-example}. Initially, the relation cell stores the connectivity between \cn{Cat}, \cn{Dan}, and $r_1$, with the success rates. 
The global time is initialized as $0$ in the \cn{gt} cell, and the predicate cell has a fixed value $\texttt{true}$, meaning that no extra condition is applied to each communication.
The message ($\qsend{1}{c}{m}^0$) sent from \cn{Cat} has an initial probability value $1$ and the time stamp $0$, referring to the time that the message is guaranteed to send out.
Node $r_0$ acts as a intermediate router so that it only contains the unit process $0$, and \cn{Dan} is waiting on receiving a message ($\qrev{c}{m}.0$). 

\begin{figure}[t]
{\small
  \begin{mathpar}

   \inferrule[GenChan]{}
       {\Cellb{\qsend{p}{c}{m}^t}{i}{a}\Cella{P}{j}{b} \qcell{\{(a,b,p)\}\cup R}{\cn{rel}}\qcell{t'}{\cn{gt}}
        \longrightarrow \Cellb{}{i-}{a}\Cella{\qsend{p}{c}{m}^{t}\texttt{|}P}{j-}{b}
              \qcell{\{(a,b,p)\}\cup R}{\cn{rel}}\qcell{t'+}{\cn{gt}} }

\ignore{
   \inferrule[GenQubit]{}
       {\Cella{P}{n,t'}{a}\qcell{t}{\cn{gt}}\longrightarrow \Cella{P}{n+,t}{a}\qcell{t}{\cn{gt}}}\;\;\texttt{when}\;t \texttt{|} \beta\wedge t' < t
}

   \inferrule[MoreTries]{}
       {\comp{P} \longrightarrow \parl{P}{\comp{P}}}
      
   \inferrule[NoTries]{}
       {\comp{P} \longrightarrow 0}

  \inferrule[PreCom]{}
      { \pcell{\qsend{p}{c}{m}^t\texttt{|} \qrev{c}{x}.P}{n}{c}\qcell{t'}{\cn{gt}}\pcell{\emptyset}{\_}{\cn{comm}}
           \longrightarrow
         \pcell{P[m/x]}{n}{c}\qcell{t'+}{\cn{gt}}\pcell{\qsend{p}{c}{m}^t}{t'}{\cn{comm}}}
                  
  \inferrule[Com]{}
      { \qcell{\qsend{p}{c}{m}^t}{\cn{comm}}\qcell{\cn{true}}{\cn{pred}}
           \xrightarrow{p.c.m}  
         \qcell{\emptyset}{\cn{comm}}\qcell{\cn{true}}{\cn{pred}} } 

  \end{mathpar}
}
\caption{Quantum Pi Semantics. $\beta$, $\mu$, and $\nu$ are globally defined for the qubit generation period, the message threshold probability, and message sending finished threshold. $\Cella{P}{n}{a}$ refers to that the $t$ in $\Cella{P}{n,t}{a}$ is omitted in the rule.}
  \label{fig:q-pi-semantics}
\end{figure}


\begin{figure}[t]
{\small
{\hspace*{-2em}
\begin{tikzpicture}[align=center,node distance=1.5cm and -1cm, thick] 
\node (1) {S$\langle\{(a,r_1,0.5), (a,r_2,0.5)\}\cup$R$\rangle$}; 
\node (2) [below left= of 1] {$\Cella{0}{9}{a}$ $\Cella{\qsend{0.5}{c}{m}|0}{9}{r_1}$... $\langle\{(r_1,r_4)\}\cup$R$\rangle$}; 
\node (3) [below right= of 1] {\text{\ \ \ \ \ \ }$\Cella{0}{9}{a}$ $\Cella{\qsend{0.5}{c}{m}|0}{9}{r_2}$..., $\ccell{\{(r_2,r_3)\}\cup\text{R}}$}; 
\node (4) [below of=2] {$\Cella{0}{8}{r_1}$ $\Cella{\qsend{0.25}{c}{m}|0}{9}{r_4}$... $\langle\{(r_4,b)\}\cup$R$\rangle$};
\node (5) [below of=3] {\text{\ \ \ \ \ \ }$\Cella{0}{8}{r_2}$ $\Cella{\qsend{0.25}{c}{m}|0}{9}{r_3}$..., $\ccell{\{(r_3,r_4)\}\cup\text{R}}$};
\node (6) [below of=4] {$\Cella{0}{8}{r_4}$ $\Cella{\qsend{0.125}{c}{m}|\qrev{c}{m}.0}{9}{b}$... $\ccell{\text{R}}$};
\node (7) [below of=5] {\text{\ \ \ \ \ \ \ \ \;}$\Cella{0}{8}{r_3}$ $\Cella{\qsend{0.125}{c}{m}|0}{9}{r_4}$..., $\ccell{\{(r_4,b)\}\cup\text{R}}$};
\node (8) [below of=6] {$\Cella{0}{9}{b}$... $\ccell{\text{R}}$};
\node (9) [below of=7] {\text{\ \ \ \ \ \ \ \ }$\Cella{0}{8}{r_4}$ $\Cella{\qsend{0.0625}{c}{m}|\qrev{c}{m}.0}{9}{b}$..., $\ccell{\text{R}}$};
\node (10) [below of=9] {\text{\ \ \ \ \ \ }$\Cella{0}{n}{b}$... $\ccell{\text{R}}$};
\draw[->] (1) -- node[midway, above left] {} (2); 
\draw[->] (1) -- node[midway, above right] {} (3); 
\draw[->] (2) -- node[midway, right] {} (4); 
\draw[->] (4) -- node[midway, right] {} (6);
\draw[->] (6) -- node[midway, right] {$0.125.c.m$} (8); 
\draw[->] (3) -- node[midway, right] {} (5); 
\draw[->] (5) -- node[midway, right] {} (7); 
\draw[->] (7) -- node[midway, right] {} (9);
\draw[->] (9) -- node[midway, right] {$0.0625.c.m$}  (10); 
\end{tikzpicture} 
}
}
\caption{Quantum Machine Transitions for \Cref{fig:q-pi-example}}
  \label{fig:q-pi-example1}
\end{figure} 

\subsection{Semantics} \label{sec:qamsemantics}

Tha QAM semantics is given based on the chemistry abstract machine framework in \Cref{fig:q-pi-semantics}.
Inside each process cell, process programs might transit to different states, such as the \rulelab{MoreTries} and \rulelab{NoTries} rules for $!P$ operations. Different cells can also interact with each other, and each QAM semantic rule only mentions the necessary cells involving in an interaction.
Transitions in a QAM rule can have either no labels, meaning that the transitions are internal; 
or a label $c.m$, referring to the message being communicated from two different parties.

There are four main tasks of the QAM semantics.
First, the semantics captures the path selection for communicating of a message between two distant parties via transmissions of the message through adjacent intermediate routers. During the procedure, QAM records the needed resource for each transmission step.
Rule \rulelab{GenChanT} and \rulelab{GenChan} describe the transmission procedure.
The former describes that when a message is firstly sent from node $a$, we look at the global time cell $\cn{gt}$ and attach the message the current time label $t$. In addition, we also attach the probability value $p$ as the probability reduction when transmitting from node $a$ to $b$; thus, the final message living in node $b$ is $\qsend{p}{c}{m}^t$.
During the process, we only 



The QAM semantics is given as a process algebra style labeled transition system listed in \Cref{fig:q-pi-semantics}.
Each transition label has either an empty label or a communication action on label $p.c.m$, where $p$ is the probability the action $c.m$ happens and $c$ is the channel name and $m$ is the actual message potentially being quantum.
Rule \rulelab{RelationUp} describes the global transitions where program execution connectivity graph is updated once upon a time,
while rule \rulelab{TimeUp} updates the global clock of the configuration.
In QAM, the semantic rules only need to mention the cells that are involved in a transition.
For example, in rule \rulelab{TimeUp}, we do not mention the cells $\qcell{\Ls}{p}$ and $\qcell{\Fs}{f}$ because they have no changes during the transition.

Rules \rulelab{GenChan}, \rulelab{GenQubit}, and \rulelab{Communication} describe the transitions happened in a program configuration.
Rule \rulelab{GenChan} represents the procedure that a message is transmitted from one location to another via a quantum channel. During the procedure, the probability for the message being successfully sent is reduced and the two locations also consume two qubits.
Here, an action $\qsend{p}{c}{m}$ is transmitted from one location, defined by the cell $\qcell{}{a}$, to another location $\qcell{}{b}$, if there is a triple relation $(a,b,p')$ in the relation cell. 
The $p'$ value represents the reduced probability factor to communicate the message between the two locations, i.e.,
the probability value $p$ in action $\qsend{p}{c}{m}$ is accumulated to become $p\cdot p'$.
Each transmission requires the consumption of two qubits: each cell consumes one qubits.
This is why $a$ and $b$'s result cells ($\Cellb{}{i-}{a}$ and $\Cellb{}{j-}{b}$) have the qubit value $i$ and $j$ being reduced, respectively.
Here, the $...$ in cell $\Cellb{}{i-}{a}$ represents that we only care about the left hand side action in a process cell regarding the rest of the content.
In the example transition in \Cref{fig:q-pi-example1} that starts from the configuration $\texttt{S} \qcell{\texttt{R}}{r}$ above,
the transitions from the top to the left and right are both applied a \rulelab{GenChan} rule.
On the left hand side, the action $c!<m>$ is transmitted from cell $a$ to cell $r_1$, and the probability value is reduced to $0.5$.
Rule \rulelab{GenQubit} generates a new qubit in a location to help transmit messages from one location to another.

Rule \rulelab{Communication} defines the behavior of consuming a message by a local process in a node cell.
In the parallel composition $\qsend{p}{c}{m}\texttt{|} \qrev{c}{x}.P$, action $\qsend{p}{c}{m}$ is waiting to be consumed on channel $c$, 
$\qrev{c}{x}$ is a receipt binding of channel $c$, and $P$ is the following process to consume the action.
After the transition, the waiting action is removed, and the remaining process $P$ is executed by substituting variable $x$ with the massage $m$. The transition is a non-empty one and the parallel composition communicates on the action $p.c.m$.

An example machine is given in \Cref{fig:q-pi-example} and \Cref{fig:q-pi-example1}.
The protocol of the definition follows the QPASS model \cite{10.1145/3387514.3405853}. 
The example contains six distinct computer locations and we want to send messages from location $a$ to $b$.
Locations $r_1$, $r_2$, $r_3$, and $r_4$ are intermediate routers. 
\Cref{fig:q-pi-example} represents the connectivity of different computers. 
\Cref{fig:q-pi-example1} provides the transition computation tree of two possible ways of sending a message from $a$ to $b$.
At the end of each path in the computation tree in \Cref{fig:q-pi-example1}, the left hand path communicates an action $0.125.c.m$, while the right hand path communicates an action $0.0625.c.m$; thus, the left hand path is selected according to the QPass model because it has a higher successful rate to send out the message.



